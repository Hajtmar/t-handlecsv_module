
\usemodule[handlecsv]
\usemodule[handlecsv-tools]

\def\mypreviousrow{\readline{\the\numexpr(\recurselevel-1)}}
\def\mycurrentrow{\readline{\recurselevel}}
\def\mynextrow{\readline{\the\numexpr(\recurselevel+1)}}

\unsetheader
\setsep{;}

\opencsvfile{mail-sample.csv}

\starttext


\startbuffer[textloop]
\cA\ (before \csvcell['A',\thenumexpr{\numline+1}], current is \cA) was born on  \cB.    
\csvcell['A',\thenumexpr{\numline+1}] was born \csvcell['B',\thenumexpr{\numline+1}].
\crlf
\stopbuffer

\startbuffer[text]
\mycurrentrow\cA\ (before \mynextrow\cA) was born on  \mycurrentrow\cB.\crlf
\stopbuffer

\startbuffer[othersolution]
\csvcell['A',\recurselevel]\ (before \csvcell['A',\thenumexpr{\recurselevel+1}])    
was born on  \csvcell['B',\recurselevel].\crlf
\stopbuffer


And classic loop solution:

\opencsvfile{mail-sample.csv}
\doloop{\ifnotEOF\getbuffer[textloop]\nextrow\else\exitloop\fi}


Dorecurse cycle solution:

\dorecurse{\numrows}{
\getbuffer[text]
}


Another dorecurse cycle solution:

\dorecurse{\numrows}{
\getbuffer[othersolution]
}


\printall

\csvreport

\printline

\stoptext





 

  
  

