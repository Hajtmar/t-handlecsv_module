\usemodule[handlecsv]

% for your testing
\def\zeroing#1#2{\ctxlua{context(string.rep("0",math.log('#1')/math.log(10)-math.log('#2'+0.1)/math.log(10)+1) ..'#2')}}

\def\zeroedlineno#1#2{\ctxlua{context(string.rep("0",math.log('#1')/math.log(10)-math.log('#2'+0.1)/math.log(10)+1) ..'#2')}} % #1 must be greater than #2

\def\zeroedlineno{\ctxlua{context(string.rep("0",math.floor(math.log(10*thirddata.handlecsv.numrows())/math.log(10))-math.floor(math.log(thirddata.handlecsv.linepointer())/math.log(10)+1)) .. thirddata.handlecsv.linepointer())}}

\def\zeroedlinenox% \unexpanded cannot be used here
   {\ctxlua{context(string.rep( "0",
    (tostring(thirddata.handlecsv.numrows())):len() -
    (tostring(thirddata.handlecsv.linepointer())):len()) ..
    thirddata.handlecsv.linepointer())}}
    
    

% ??? why so complicated \def\AddName{\ctxlua{context(thirddata.handlecsv.getcellcontent(1,thirddata.handlecsv.gCurrentLinePointer[thirddata.handlecsv.getcurrentcsvfilename()]))}}
\def\AddName{\cA}

\startbuffer[single-certificates]
\executesystemcommand{context --runs=1 --purgeall --result=certificate-\zeroedlinenox-\AddName.pdf --arguments="MainLinePointer={\lineno}" sample-cert.tex}
%\zeroedlinenox -- \lineno\crlf
\stopbuffer

\starttext
\opencsvfile{participants.csv}
\startitemize[n]
\doloopif{\cA}{~=}{}{\item \cA -- \getbuffer[single-certificates]}
%\doloopif{\cA}{~=}{}{\getbuffer[single-certificates]}
%\dorecurse{100}{\zeroing{100}{\recurselevel}\crlf}
%\dorecurse{20}{\zeroing{100}{\recurselevel}\crlf}
%\dorecurse{50}{\zeroing{1000}{\recurselevel}\crlf}
%\stopitemize
\stoptext
