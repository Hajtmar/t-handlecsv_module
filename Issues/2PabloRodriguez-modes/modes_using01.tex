% Hi Jaroslav,
% 
% I know I have to give you an outline (or even better, a draft) on the
% new handlecsv. I have been busy this previous months (no vacation for
% me yet :(), but I hope I will find time this weekend.
% 
% I have another question about the old t-scancsv.lua. Here is a minimal
% sample:

\usemodule[handlecsv]
%\usemodule[scancsv]

\enablemode[mymode]
%\disablemode[mymode]



\starttext

\startbuffer[text-content]
\cA\hfill\cB\crlf
\stopbuffer

\startbuffer[text-content-mode-enabled]
mode enabled: \cA\hfill\cB\crlf
\stopbuffer

\starttext
\startbuffer[text-content-mode-disabled]
mode disabled: \cB\hfill\cA\crlf
\stopbuffer


Loop and modes:

\opencsvfile{mail-sample.csv}
\doloop{%
	\startmode[mymode]%
		\getbuffer[text-content-mode-enabled]%
	\stopmode%
	\startnotmode[mymode]%
		\getbuffer[text-content-mode-disabled]%
	\stopnotmode%
\ifnotEOF\nextrow\else\exitloop\fi%
}


Another example:

\opencsvfile{mail-sample.csv}
\doloop{%
	\doifmodeelse{mymode}{\cA}{\cB} -- \ %
	\doifmodeelse{mymode}{\getbuffer[text-content-mode-enabled]}{\getbuffer[text-content-mode-disabled]}%
	\ifnotEOF\nextrow\else\exitloop\fi%
}

And next another example:

\opencsvfile{mail-sample.csv}
\doloop{%
	\doifnotmode{mymode}{\getbuffer[text-content-mode-disabled]}
	\doifmode{mymode}{mode is enaabled}
	\ifnotEOF\nextrow\else\exitloop\fi%
}




Attention! This loop cycle is wrong!!!! (last round is not realized)

\opencsvfile{mail-sample.csv}
\doloop{%
	\ifnotEOF\getbuffer[text-content]\nextrow
		\else\exitloop
	\fi
}




Right realizing loop is :

\opencsvfile{mail-sample.csv}
\doloop{%
	\ifnotEOF\getbuffer[text-content]\nextrow
		\else\getbuffer[text-content]\exitloop
	\fi
}

or: 

\opencsvfile{mail-sample.csv}
\doloop{% we assume, that file has more than one row
\getbuffer[text-content]
	\ifnotEOF\nextrow
		\else\exitloop
	\fi
}



Another testings:

A) 

\setfiletoscan{mail-sample.csv}
\opencsvfile
\doloop{
\getbuffer[text-content]
\ifnotEOF\nextrow\else\exitloop\fi
}


B)

\def\lineaction{\getbuffer[text-content]}
\opencsvfile{mail-sample.csv}
\filelineaction			
			

C) 

\opencsvfile{mail-sample.csv}
\dorecurse{\numrows}{
\getbuffer[text-content]
\nextrow
}
		
\stoptext


% Being the contents of mail-sample.csv:
% 
%     "ME";11/10/14
%     "You";05/02/14
%     "He";03/03/54
%     "She";03/03/12
%     "It";03/03/14
% 
% Is there any way that the loop can ignore a row when a mode is enabled
% in t-scancsv.lua?
% 
% I mean, there are the \doifmode, \doifnotmode and \doifnotmodeelse
% commands, but I don’t know how to mix them with \doloop.
% 
% Many thanks for your help,
% 
% 
% Pablo
