\usemodule[handlecsv]
\opencsvfile{numbers.csv}


\def\mujproblem{1}

\def\notEOF{%
\directlua{
	if thirddata.handlecsv.getcurrentlinepointer() > thirddata.handlecsv.gNumRows[thirddata.handlecsv.getcurrentcsvfilename()] then context([[0]])
  else context([[1]])
  end	
}1}

\starttext



 \def\testik{
 \directlua{
 		if thirddata.handlecsv.getcurrentlinepointer() > thirddata.handlecsv.gNumRows[thirddata.handlecsv.getcurrentcsvfilename()] then
      context([[\global\EOFtrue]])
      context([[\global\notEOFfalse]])
   else
   	  context([[\global\EOFfalse]])
      context([[\global\notEOFtrue]])
   end	
 }
 }
 


\startbuffer[xtable]
\startxrow %
\startxcell %
   A
%\linepointer  -- \Romannumerals{\getcsvcell['A',\recurselevel]}% 
% 	\linepointer  -- \Romannumerals{\cA}%
\stopxcell %
\stopxrow %
\stopbuffer

% 
% 
% 
% 
% 


 
\starttext 

I have no clue, why this example not work. It seems, that problem is in collision with startxtable macro.

\resetlinepointer

\startxtable[option=stretch, split=yes]

\doloop{
\if\notEOF%
	\startxrow %
  \startxcell %
   A
%\linepointer  -- \Romannumerals{\getcsvcell['A',\recurselevel]}% 
% 	\linepointer  -- \Romannumerals{\cA}%
\stopxcell %
  \startxcell B\stopxcell   
\stopxrow%
\nextrow\else\exitloopnow\fi
}
%
\stopxtable%



\stoptext
