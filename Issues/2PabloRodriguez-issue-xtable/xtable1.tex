


% The problem here is that \testtrue is not expandable and not protected and 
% therefore fragile. It is defined as \let\iftrue\iftrue. In an expandable 
% context like \edef or \typeout the \let assignment is ignored, and both 
% if-switches are expanded. Because there is only one \fi (which is taken as 
% part of \iftrue) the compiler complains about the missing \fi.
% http://tex.stackexchange.com/questions/35807/newif-conditional-causing-problem-in-typeout-or-edef

\usemodule[handlecsv]
\opencsvfile{numbers.csv}


\def\refreshvalues{\linepointer -- \ifnum\linepointer<5\notEOFfalse\EOFtrue A\else\notEOFtrue\EOFfalse B\fi}


\def\testik{
\directlua{
	if thirddata.handlecsv.getcurrentlinepointer() > thirddata.handlecsv.gNumRows[thirddata.handlecsv.getcurrentcsvfilename()] then
     context([[\global\EOFtrue]])
     context([[\global\notEOFfalse]])
  else
  	  context([[\global\EOFfalse]])
     context([[\global\notEOFtrue]])
  end	
}
}




\startbuffer[myxtable]
\startxrow %
\startxcell %
% 	\linepointer  -- \Romannumerals{\getcsvcell['A',\recurselevel]}% 
% 	\linepointer  -- \Romannumerals{\cA}%
 \linepointer  -- {\cA}%
\stopxcell %
\stopxrow %

\stopbuffer



\starttext



I have no clue, why this example not work. It seems, that problem is in collision with startxtable macro.

% The problem here is that \testtrue is not expandable and not protected and 
% therefore fragile. It is defined as \let\iftrue\iftrue. In an expandable 
% context like \edef or \typeout the \let assignment is ignored, and both 
% if-switches are expanded. Because there is only one \fi (which is taken as 
% part of \iftrue) the compiler complains about the missing \fi.
% http://tex.stackexchange.com/questions/35807/newif-conditional-causing-problem-in-typeout-or-edef




%\refreshvalues

\refreshvalues
\resetlinepointer

\startxtable
\doloop{%
\ifnum\linepointer>10\exitloop
\else  
\getbuffer[myxtable]
\nextrow
\fi
}%
\stopxtable %


xxxxxxxxxxxx

\startxtable
  \startxrow
    \startxcell one \stopxcell
    \startxcell two \stopxcell
  \stopxrow
  \startxrow
    \startxcell alpha \stopxcell
    \startxcell beta  \stopxcell
  \stopxrow
\stopxtable

xxxxxxxxxxxx


because this example is partly working:


\resetlinepointer

% remark \startxtable[option=stretch, split=yes]
\doloop{
\ifnotEOF %
\getbuffer[myxtable]
\nextrow\else
\exitloop%
\fi
}%

% remark  \stopxtable%

Working example without doloop 

\resetlinepointer
\startxtable[option=stretch, split=yes]
 \dorecurse{\numrows}{
 \startxrow%
  \startxcell
	\Romannumerals{\csvcell['A',\recurselevel]}
  \stopxcell%
\stopxrow%
}
\stopxtable


OR next working example with doloop but without ifnotEOF condition 


\resetlinepointer
\startxtable[option=stretch, split=yes]
\doloop{\ifnum\recurselevel<\thenumexpr{\numrows+1}
  \startxrow
  \startxcell
	\Romannumerals{\csvcell['A',\recurselevel]}
  \stopxcell
\stopxrow
\nextrow
\else
\exitloop
\fi
}

\stopxtable





\stoptext


% \resetlinepointer
% \bTABLE
% 
% \doloop{
% \ifnotEOF%
% \bTR\bTD\Romannumerals{\csvcell['A',\recurselevel]}\eTD\eTR
% \nextrow\else\exitloop%
% \fi
% }%
% 
% \eTABLE
