\usemodule[handlecsv]
\usemodule[handlecsv-extra]



% -- POZN 	nemělo by raději být gColNames v řádku 672??  viz		returnparametr=thirddata.handlecsv.gColumnNames[csvfile][column]

% Toto bylo již zakomponováno do t-handlecsv.lua library dne 23.2.2018
%\startluacode
%-- Substitute text in cell content of specified CSV file with other text
%function thirddata.handlecsv.substitutecontentofcellof(csvfile,column,row,whattoreplace,substitution)
%  local csvfile=thirddata.handlecsv.handlecsvfile(csvfile)
%  local column=thirddata.handlecsv.gColNames[csvfile][column]
%  local whattoreplace=tostring(whattoreplace)
%  local substitution=tostring(substitution)
%  return thirddata.handlecsv.getcellcontentof(csvfile,column,row):gsub(whattoreplace,substitution)
%end
%
%-- Substitute text in cell content of current CSV file with other text
%function thirddata.handlecsv.substitutecontentofcell(column,row,whattoreplace,substitution)
%  local csvfile=thirddata.handlecsv.getcurrentcsvfilename()
%  local column=thirddata.handlecsv.gColNames[csvfile][column]
%  return thirddata.handlecsv.substitutecontentofcellof(csvfile,column,row,whattoreplace,substitution)
%end
%
%-- Substitute text in cell content of current row of current CSV file with other text
%function thirddata.handlecsv.substitutecontentofcellofcurrentrow(column,whattoreplace,substitution)
%  local row=thirddata.handlecsv.linepointer()
%  return thirddata.handlecsv.substitutecontentofcell(column,row,whattoreplace,substitution)
%end
%\stopluacode
%
%
%% Substitution of text #2 in cell content by text #3. Substitution is done in the current column of column #1 (number, XLS name or cX name)
%\def\Xreplacecontentin#1#2#3{\ctxlua{context(thirddata.handlecsv.substitutecontentofcellofcurrentrow('#1','#2','#3'))}}%

%\def\replacecontentin#1#2#3{\ctxlua{context(thirddata.handlecsv.getcellcontent(thirddata.handlecsv.xls2ar('#1'),thirddata.handlecsv.linepointer()):gsub("#2","#3"))}}%
%\def\Xcontentreplace#1#2#3{\ctxlua{context(thirddata.handlecsv.getcellcontent(thirddata.handlecsv.gColNames[thirddata.handlecsv.getcurrentcsvfilename()]['#1'],thirddata.handlecsv.linepointer()):gsub("#2","#3"))}}%


%
%% Zatím nezařazeno do knihovny <t-handlecsv.lua>, ale zařazeno do pomocné knihovny <t-handlecsv-unused-yet.lua>.
%\startluacode
%function thirddata.handlecsv.addleadingcharacters(character, tonumberortext, width)
%-- Add leading characters to number to align with the width
%   local strcharacter=tostring(character)
%   local strnumberortext=tostring(tonumberortext)
%   strnumberortext = string.rep(strcharacter, width-#strnumberortext)..strnumberortext
%   return strnumberortext -- It returns a strange result unless the leading character is just one.
%end
%
%function thirddata.handlecsv.addleadingzeros(tonumberortext, width)
%-- Add leading zeros to number to align with the width
%   return thirddata.handlecsv.addleadingcharacters(0, tonumberortext, width)
%end
%
%function thirddata.handlecsv.addzeros(tonumber)
%-- Add leading zeroes depending on the number of rows
%    local width=string.len(tostring(thirddata.handlecsv.numrows()))
%    return thirddata.handlecsv.addleadingzeros(tonumber, width)
%end
%
%\stopluacode
%
%\def\addleading#1#2#3{\ctxlua{context(thirddata.handlecsv.addleadingcharacters('#1','#2','#3'))}}
%\def\addzeros#1#2{\ctxlua{context(thirddata.handlecsv.addleadingzeros('#1','#2'))}}
%\def\zeroed#1{\ctxlua{context(thirddata.handlecsv.addzeros('#1'))}}
%\def\zeroedlineno{\ctxlua{context(string.rep( "0",(tostring(thirddata.handlecsv.numrows())):len() - (tostring(thirddata.handlecsv.linepointer())):len()) .. thirddata.handlecsv.linepointer())}} % from Pablo
%

\startbuffer[example]

1. \replacecontentin{A}{\\\\}{_} %subst. \\ to x in colmn A of current row

\def\lbreak{\\\\} % use macro

2. \replacecontentin{A}{\lbreak}{x}  %the same

3. \replacecontentin{1}{\\\\}{ }  %subst. \\ to space in 1th colmn of current row

4. \replacecontentin{cA}{\\\\}{ } %subst. \\ to space in \cA colmn of current row


Unusual uses:

5. \replacecontentin{A}{John\\\\}{\\bgroup\\bf\\red David \\egroup }
 
6. \replacecontentin{1}{John\\\\Wayne}{Wayne \\bgroup\\bf\\blue John\\egroup } 

\def\gettextfrom[#1]{\replacecontentin{#1}{\\\\}{  }}% Your macro

7. \gettextfrom[1], \gettextfrom[A], \gettextfrom[cA]  

\stopbuffer


\startbuffer[single-card]

This examples:

\typebuffer[example]

Gets theese results:

\getbuffer[example]


\stopbuffer

\starttext
\opencsvfile{cards.csv}


%zeroedlineno: \zeroedlineno --- \cA 

\dorecurse{3}{
\replacecontentin{cA}{\cA}{ccc \cA ccc }\nextrow
 
}


\startluacode
context(tostring(thirddata.handlecsv.substitutecontentofcellof("cards.csv","cA",1,"John","JOHN")))
context(tostring(thirddata.handlecsv.substitutecontentofcell("cA",1,"John","JOHNOVO")))
context(thirddata.handlecsv.substitutecontentofcellofcurrentrow("A","John","Náhrada"))
\stopluacode



\doloopif{\cA}{~=}{}{\getbuffer[single-card]}

\startluacode
for i=1, 10 do
context (thirddata.handlecsv.addzeros(i))
context("\\crlf")
end
\stopluacode


\hairline

 {\tt
\addleading{X}{Hello}{15}

\addleading{X}{}{15}

\addleading{Y}{He}{15}

\addleading{-}{Hello Hello}{15} 


\addleading{Q}{1}{15} 

\addleading{9}{1234}{15} 

\addleading{_}{\\cA}{15} % fail

\addleading{<}{Jonathan Wayne}{15}

\addleading{\\ }{\numline}{15} 

\addleading{\\percent}{\numline}{15} 

\addleading{.}{\numline}{15} 

\addzeros{\\cA}{15} % fail

\addzeros{\cB}{15}

\zeroed{\numline}

\zeroed{\linepointer}

\zeroed{12}

\zeroed{\cB}

\zeroed{\\cA}  % fail
}

\hairline



\stoptext
