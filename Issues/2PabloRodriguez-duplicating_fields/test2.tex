    \setuppagenumbering[location=]
    \usemodule[hl2t-scancsv.lua]


\def\readnthline#1{%
\startluacode
thirddata.scancsv.opencsvfile(thirddata.scancsv.gFile2Scan)
local numrows = thirddata.scancsv.getnumrows(thirddata.scancsv.gFile2Scan)
local numlineparametr=0
if thirddata.scancsv.gCSVHeader then 
	numlineparametr=tonumber('#1')+1
else 	
	numlineparametr=tonumber('#1')
end

local line = ""
local outputline = ""
local csvline={}

local n=0
for line in io.lines(thirddata.scancsv.gFile2Scan) do
   n=n+1
   if thirddata.scancsv.gCSVHeader and n==1 then end
   if n==numlineparametr then outputline=line;break end
end

for i=1, thirddata.scancsv.gNumCols do
	thirddata.scancsv.fillvaluesintomacros(i,'')
end

if outputline ~= nil and outputline ~= ""  then
 csvline=thirddata.scancsv.parsecsvdata(outputline)
 local numcols = 0
	for _ in pairs(csvline) do numcols = numcols + 1 end
	 	for i = 1, numcols do 	%-- create an associative array: the index column names
	 	  %-- Experimental row:
			thirddata.scancsv.gCSVField[thirddata.scancsv.ar2xls(i)]=csvline[i] %-- Fill the column field from the act. line data
	 	  %--thirddata.scancsv.gCSVField[i]=csvline[i] -- not used
			thirddata.scancsv.gCSV[thirddata.scancsv.gCSV[i]]=csvline[i] %-- values are data from a CSV row of the table column
		  thirddata.scancsv.fillvaluesintomacros(i,csvline[i])
	end %-- for
end %  if outputline ~= nil then 
\stopluacode
}

    \unexpanded\def\lineaction{
		\previousName -- \Name\crlf
		
		-
		
		\nextrow
		
		\previousName -- \Name\crlf
    
		------
		
		}


    
    \setsep{;}
       
    \starttext
    
    \setheader
    \opencsvfile{dates-table.csv}

		SETHEADER - hlavička existuje:
		
		\def\zobraz{\cA-\Numline, \cB-\Name,\cC-\Begin,\cD-\End, PREV:\previousNumline}
		\def\zobraztwo{\cA, \cB, \cC, \cD, PREV:\previouscA}

    2. \readnthline{2} --> \zobraz,\previousNumline
    
		1. \readnthline{1}  --> \zobraz
		
		0. \readnthline{0}  --> \zobraz
		
		11. \readnthline{11}  --> \zobraz
		
		3. \readnthline{3}  --> \zobraz
		
		4. \readnthline{4}  --> \zobraz
		
		7. \readnthline{7}  --> \zobraz
		
		5. \readnthline{5}  --> \zobraz
	
	Změna:
	
	RESETHEADER - hlavička neexistuje:
		
		\resetheader
		\opencsvfile{dates-table2.csv}
		
		2. \readnthline{2} \zobraz 
    
		1. \readnthline{1} \zobraz
		
		0. \readnthline{0} \zobraz
		
		11. \readnthline{11} \zobraztwo
		
		3. \readnthline{3} \zobraztwo
		
		4. \readnthline{4} \zobraztwo
		
		7. \readnthline{7} \zobraztwo
		
		5. \readnthline{5} \zobraztwo
		
		
          %\filelineaction
          
%        \dorecurse{3}{
%        \previousName -- \Name\crlf
%        \nextrow xxx
% 			 \previousName -- \Name\crlf
% 			 \nextrow
% 			 }   






    \stoptext


file = assert(io.open("dates-table.csv", "r"))
for line in file:lines() do context(line) end
file:close()
