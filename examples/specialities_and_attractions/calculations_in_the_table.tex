

\input ../incl_handlecsv_file.tex % \usemodule[t-handlecsv_opt.lua]

\def\eval#1{\directlua{context(tonumber(#1))}}
\def\commatodecimal#1{\directlua{context(string.gsub('#1', "," , "."))}}

% Original macro did not round
%\def\priceforallitems{
%    \ctxlua{%
%        context(tostring(tonumber(
%        [==[\commatodecimal{\columncontent['Unit Price']}]==]*
%        [==[\commatodecimal{\columncontent['Order Quantity']}]==]))
%        )
%    }
%}

\begingroup
  \catcode`\%=12\relax
  \gdef\luatemplate{[==[%%.%df]==]}
\endgroup

\def\roundto#1#2{%
  \ctxlua{%
    local number = tonumber("\luaescapestring{#1}")
    local places = tonumber("\luaescapestring{#2}")
    local fmt = string.format(\luatemplate, places)
    context(string.format(fmt, number))
  }%
}

\def\priceforallitems{%
  \ctxlua{%
    local product =
        tonumber([==[\commatodecimal{\columncontent['Unit Price']}]==])
        *
        tonumber([==[\commatodecimal{\columncontent['Order Quantity']}]==])
    tex.sprint("\\roundto{" .. product .. "}{2}")
  }%
}

\setupTABLE[background=color,backgroundcolor=white, offset=3pt]
\setupTABLE[row][first][background=color,backgroundcolor=lightgray, style=bold]
\setupTABLE[row][last][background=color,backgroundcolor=lightgray, style=bold]
\setupTABLE[c][1][style=\tfx, width=8cm]
\setupTABLE[c][2][align=middle]
\setupTABLE[c][3][alignmentcharacter={.}, aligncharacter=yes, align=middle]
\setupTABLE[c][4][alignmentcharacter={.}, aligncharacter=yes, align=middle, style=\ss]

\def\firsttableline{
    \bTR
        \bTD Product name\eTD
        \bTD Quantity\eTD
        \bTD Unit Price\eTD
        \bTD Items price\eTD
    \eTR
}

\def\lasttableline{
    \bTR
        \bTD[nc=3]TOTAL PRICE\eTD
        \bTD\roundto{\eval{\totalprice}}{2}\eTD
    \eTR
}


\def\lineaction{%
    \expanded{
        \bTR
            \bTD\columncontent['Product Name']\eTD
            \bTD\columncontent['Order Quantity'] \eTD
            \bTD\commatodecimal{\columncontent['Unit Price']}\eTD
            \bTD\priceforallitems\eTD
        \eTR
    }
    \eaddto\totalprice{+\priceforallitems}% ADD into
}


\startbuffer[tablerows]
    \setlinepointer{\recurselevel}
    \expanded{
    \bTR
        \bTD\columncontent['Product Name']\eTD
        \bTD\columncontent['Order Quantity'] \eTD
        \bTD\commatodecimal{\columncontent['Unit Price']}\eTD
        \bTD\priceforallitems\eTD
    \eTR
    }
    \eaddto\totalprice{+\priceforallitems}% ADD into
\stopbuffer



\starttext

\setheader
\setsep{;}
\catcode`\#=12 %CSV file contains # characters (i.e. TeX problematic character)


\opencsvfile{superstore_sales_table.csv}

\title{Using \type{\dorecurse} cycle}

\def\totalprice{0}

\bTABLE[split=yes]
\firsttableline
\dorecurse{\numrows}{
	\setlinepointer{\recurselevel}
	\lineaction
	\nextrow
    }
    \lasttableline
\eTABLE

\hairline

\subject{Buffers using}

\bTABLE[split=yes]
\firsttableline
\dorecurse{\numrows}{%
	\getbuffer[tablerows]
    }
    \lasttableline
\eTABLE




\title{Using \type{\filelineaction} cycle}

\def\totalprice{0}

\bTABLE[split=yes]
    \firsttableline
    \filelineaction
    \lasttableline
\eTABLE


\stoptext

