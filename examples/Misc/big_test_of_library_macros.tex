\input ../incl_handlecsv_file.tex

\def\typeexample[#1]#2#3{% buffname, title, description
\subject{#2}
#3\par
The following source code:

\blank[big]

\typebuffer[#1]

\blank[2*big]

produces the following output result:

\blank[big]

\startlines
\getbuffer[#1]
\stoplines

\blank[big]

\hairline
}





\def\lastname{\dosingleempty\dolastname}

\def\dolastname[#1]%
   {\iffirstargument%
	 		arg. #1
    \else
    	arg. nic
   \fi
}


%\unexpanded\def\lineaction{\numrows \cA, \cB, \cC, \cD, \cE, \cF, \cG, \cH, \cI. \crlf}


\unexpanded\def\dumpcontents{\lineno: %
	\ifissetheader %
	\Lastname\ \Firstname\ -- \Email\crlf
	\else\cA\ \cB\ -- \cI \crlf
	\fi
% 	\ifnotsetheader \Lastname, \Firstname, \Flag, \MonThu, \ThuPM, \Fri, \Sat, \Sun, \Email. \crlf
% 	\else \cA, \cB, \cC, \cD, \cE, \cF, \cG, \cH, \cI. \crlf
% 	\fi
}

\unexpanded\def\DumpContents{\lineno:%
% for lide-specialheadernames.csv  table. This table has a special (TeX problematic) header
	\ifissetheader\columncontent['Příjmení'] -- \Prijmeni, \columncontent['Jméno'] -- \Jmeno, \columncontent['Příznak'], \columncontent['123'] -- \IIIIII (MonThu), \columncontent['!?*'] -- \xxx (ThuPM), \Fri, \Sat, \Sun, \Email. \crlf
	\else\cA, \cB, \cC, \cD, \cE, \cF, \cG, \cH, \cI. \crlf
	\fi
% 	\ifnotsetheader \Lastname, \Firstname, \Flag, \MonThu, \ThuPM, \Fri, \Sat, \Sun, \Email. \crlf
% 	\else \cA, \cB, \cC, \cD, \cE, \cF, \cG, \cH, \cI. \crlf
% 	\fi
}

\unexpanded\def\lineaction{LINEACTION: \dumpcontents}
\unexpanded\def\action{ACTION: \dumpcontents}


\setheader

\setsep{;}
\opencsvfile{cities.csv}


\setsep{,}
% \unsetsep
\opencsvfile{lide.csv}
%\opencsvfile{lide-withoutheader.csv}
%\opencsvfile{lide-specialheadernames.csv}

\starttext


%\csvreport%{lide.csv}


\blank[big]


\title{Working with columns predefinded macros}

Predefined macros are:

\starttyping
\colname
\xlscolname
\indexcolname
\columncontent
\numberxlscolname
\stoptyping

\blank[big]

Notice: Last open CSV file is so called current CSV file. Specific macros are used to work with the current file (as see bellow).

\blank[big]


\startbuffer[macro.columns.example1]
 1 \type{\colname[2]}: \colname[2]
 2. \type{\xlscolname[5]}: \xlscolname[5]
 3. \type{\numberxlscolname['F']}: \numberxlscolname['F']
 4. \type{\indexcolname['Firstname']}: \indexcolname['Firstname']
 5. \type{\indexcolname['B']}: \indexcolname['B']
 6. \type{\indexcolname['cC']}: \indexcolname['cC']
 7. \type{\indexcolname['2']}: \indexcolname['2']
 8. \type{\indexcolname[5]}: \indexcolname[5]
 9. \type{\columncontent[2]}: \columncontent[2]
10. \type{\columncontent['2']}: \columncontent['2']
11. \type{\columncontentof[lide.csv][2]}: \columncontentof[lide.csv][2]
12. \type{\columncontentof[lide.csv]['2']}: \columncontentof[lide.csv]['2']
13. \type{\columncontentof[lide.csv]['Firstname']}: \columncontentof[lide.csv]['Firstname']
14. \type{\columncontent['A']}: \columncontent['A']
15 \type{\columncontent['cA']}: \columncontent['cA']
16. \type{\columncontent['Firstname']}: \columncontent['Firstname']
17. \type{\columncontent['firstname']}: \columncontent['firstname']
18. \type{\columncontent['123']}: \columncontent['123']
19. \type{\columncontent['!?*']}: \columncontent['!?*']
20. \type{\columncontentof[cities.csv][2]}: \columncontentof[cities.csv][2]
21. \type{\columncontentof[cities.csv]['Name']}: \columncontentof[cities.csv]['Name']
\stopbuffer



\typeexample[macro.columns.example1]{Examples of using predefined columns macros}{}

\type{\columncontent['š23cv']}: \columncontent['š23cv']  \% not exist this column

\type{\columncontent['Příjmení']}: \columncontent['Příjmení']  \% not exist this column

\type{\columncontent['Prijmeni']}: \columncontent['Prijmeni']  \% not exist this column

\type{\columncontent['Jméno']}: \columncontent['Jméno']  \% not exist this column

\type{\columncontent['Jmeno']}: \columncontent['Jmeno']  \% not exist this column




\blank

\hairline

Dorecurse example 1:

\starttyping
\dorecurse{\numrows}{
\recurselevel:
\readline{\recurselevel}
\columncontent['Firstname'] -- \columncontent['Lastname'] -- \columncontent[3]
\crlf}
\stoptyping

\blank[big]

\dorecurse{\numrows}{\recurselevel: \readline{\recurselevel} \columncontent['Firstname'] -- \columncontent['Lastname'] -- \columncontent[3] \crlf }

 \blank

\hairline

Dorecurse example 2:

\dorecurse{9}{\edef\actualcolname{\colname[\recurselevel]} \recurselevel. column: \colname[\recurselevel] -- \indexcolname[\actualcolname]\crlf}



Using Luacodes:

Column names:

\startluacode
	for i=1,thirddata.handlecsv.gNumCols['lide.csv'] do
		context('colname '..i..': \\backslash '..thirddata.handlecsv.gColumnNames['lide.csv'][i]..' (\\backslash c'..thirddata.handlecsv.ar2colnum(i)..')\\crlf')
	end
\stopluacode

Table column names:

\startluacode
context(thirddata.handlecsv.gColumnNames['lide.csv'][1]..' [1]\\crlf')
context(thirddata.handlecsv.gColumnNames['lide.csv'][2]..' [2]\\crlf')
context(thirddata.handlecsv.gColumnNames['lide.csv'][3]..' [3]\\crlf')
\stopluacode

Column names macros: (\backslash colname[i]\ and\ \backslash xlscolname[i])

\dorecurse{10}{\recurselevel: \colname[\recurselevel] and \xlscolname[\recurselevel]\crlf}


\blank[big]

\hairline


\title{Working with \type{\addto} (\type{\eaddto}) macro: }

% Example of \addto (\eaddto) macro:

% init empty macro
\def\temporary{}



Content of \type{\temporary} macro is now: <\temporary>

\dorecurse{\numrows}{%
	\readline{\recurselevel}
	\eaddto\temporary{\Firstname, }%
}

\blank[big]

Content of \type{\temporary} macro is now (all names): <\temporary>

\blank[big]

\hairline





\title{Working with misc macros: }



Macros:

\type{\numrows}: \numrows

\type{\numcols}: \numcols

\type{\csvfilename}:  \csvfilename\  (synonym is \type{\currentcsvfilename}:  \currentcsvfile)

\type{\numrowsof[cities.csv]}: \numrowsof[cities.csv]

\type{\numcolsof[cities.csv]}: \numcolsof[cities.csv]



\blank[big]







\title{Working with table cells }


Table cells:

\startluacode
context(thirddata.handlecsv.gTableRows['lide.csv'][1][1]..' [first row, first column]\\crlf')
context(thirddata.handlecsv.gTableRows['cities.csv'][1][2]..' [first row, second column]\\crlf')
context(thirddata.handlecsv.gTableRows['cities.csv'][2][3]..' [second row, third column]\\crlf')
\stopluacode


\blank[3*big]

\hairline

Table cell contents by TeX macro:

\blank[big]


column 1, row 1 ie column \colname[1], row 1 : \csvcell[1,1]

column 2, row 3 ie column \colname[2], row 3 :   \csvcell[2,3]

column 9, row 5  ie column \colname[9], row 5 :  \csvcell[9,5]


'Lastname' column (ie first column) and first row:

 \type{\csvcell['Lastname',1] -->} \csvcell['Lastname',1]

'Firstname' column (ie second column) and first row:

 \type{csvcell['Firstname',1] -->} \csvcell['Firstname',1]

'Email' column (ie nineth column) and fifth row:

\type{\csvcell['Email',5] -->} \csvcell['Email',5]


\blank[big]


\blank[3*big]
\hairline

Excel notation of column names:

\blank[big]

\type{\csvcell['A',1] -->} \csvcell['A',1]

\type{\csvcell['B',3]  -->} \csvcell['B',3]

\type{\csvcell['I',5] -->} \csvcell['I',5]


\blank[big]


List of ten rows of 'Firstname' column (\type{\dorecurse} loop using):

\startbuffer[recurseexample]
\dorecurse{10}{\type{\Firstname[}\recurselevel\type{]}:
	\csvcell['Firstname',\recurselevel] \crlf
}
\stopbuffer

\typebuffer[recurseexample]

\getbuffer[recurseexample]


\blank[3*big]

\hairline

Undefined and defined cells:

\blank[big]


\type{\csvcell['Is',5] -->} \csvcell['Is',5]

\type{\csvcell['I',51] -->} \csvcell['I',51]

\type{\csvcell['I',0]  -->} \csvcell['I',0]

\type{\csvcell['I',50] -->} \csvcell['I',50]


\blank[3*big]
\hairline

Use macro references to row and column:

\blank[big]


\def\myrow{13}
\def\mycolumn{'A'}

\type{\csvcell[\mycolumn,\myrow] -->} \csvcell[\mycolumn,\myrow]


\def\myrow{31}
\def\mycolumn{'Firstname'}

\type{\csvcell[\mycolumn,\myrow] -->} \csvcell[\mycolumn,\myrow]



\page

List of all CSV table by use cycle:
\blank[big]

\bTABLE
\dorecurse{\numrows}{
	\bTR
	\dorecurse{\numcols}{\bTD \csvcell[##1,#1] \eTD}
	\eTR
}
\eTABLE


\page


List of all CSV table by use cycle (other version):

\bTABLE
    \dorecurse{\numrows}{\bTR
        \dorecurse{\numcols}{\bTD \csvcell[\currentTABLEcolumn,\currentTABLErow] \eTD}
    \eTR}
\eTABLE





\blank[3*big]


Write all header names:

\blank[big]

\dostepwiserecurse{1}{\numcols}{1}{
\csvcell[\recurselevel,0],\ }


\blank[3*big]
\hairline


\part{Cycles}


\title{DOLOOP FROM - TO CYCLES}



\startbuffer[example1]
\doloopfromto{5}{8}{\dumpcontents}
\stopbuffer
\typeexample[example1]{DOLOOP FROM - TO cycle}{}





\startbuffer[example2]
\doloopforall
\stopbuffer
\typeexample[example2]{DOLOOP FOR ALL cycles}{Do lineaction macro for all lines}





\startbuffer[example3]
\doloopforall{\dumpcontents}
\stopbuffer
\typeexample[example3]{DOLOOPACTION cycles}{Do specific \type{\dumpcontents} action for all lines}



\title{DOLOOPACTION cycles}



\startbuffer[example4]
\doloopaction % implicit use \lineaction macro
\stopbuffer

\typeexample[example4]
{\type{\doloopaction} macro}
{Do specific macro \type{\lineaction} for all lines}



\startbuffer[example5]
\doloopaction{\action} % use \action macro for all lines of open CSV file
\stopbuffer
\typeexample[example5]{DO SPECIFIC ACTION (for all lines):}{DO LINEACTION (for all lines):}



\subject{DO SPECIFIC ACTION (for first lines):}

\startbuffer[example6]
\doloopaction{\action}{4} % use \action macro for first 4 lines

\stopbuffer


\typebuffer[example6]

\getbuffer[example6]


%\def\be{b*}
%\def\ee{*e}
%
%\def\typeexample[#1][#2]\be#3\ee{
%\subject{#1}
%\startbuffer[buffexample#1]
%#3
%\stopbuffer
%
%\typebuffer[buffexample#1]
%
%\getbuffer[buffexample#1]
%
%}
%
%
%\typeexample[1][DO SPECIFIC ACTION (for first lines)]\be
%\doloopaction{\action}{4} % use \action macro for first 4 lines
%\ee
%
%
%

\subject{DO SPECIFIC ACTION (for lines from - to specific lines):}


\startbuffer[c7]
\doloopaction{\action}{4} % use \action macro for first 4 lines

\stopbuffer


\typebuffer[c7]

\getbuffer[c7


]



 OK: \doloopaction{\action}{2}{5} % use \action macro for lines from 2 to 5


 FILELINEACTION:

 Do for all lines of opening table

 OK: \filelineaction


 Do for all lines of specific table name

 OK: \filelineaction{lide.csv}


DOLOOPIF CYCLES:

%\doloopif{value1}{[compare_operator]}{value2}{macro_for_doing} % [compareoperators] <, >, ==(eq), ~=(neq), >=, <=, in, until, while





AAAAAAAAAAAAAAAAAA

\resetlinepointer
\dorecurse{5}{\recurselevel: \readline{\recurselevel}\dumpcontents\nextline}

\resetsernumline
\dorecurse{5}{\recurselevel: \readline[\recurselevel]\dumpcontents\nextline}


\readline[7]

\dumpcontents

SSSSSSSSSSSSS

\readline

\dumpcontents

xxxxxxxxxxxxxxx


1
\startluacode
thirddata.handlecsv.readline(1)
\stopluacode
\dumpcontents

5
\startluacode
thirddata.handlecsv.readline(5)
\stopluacode
\dumpcontents

0
\startluacode
thirddata.handlecsv.readline(0)
\stopluacode
\dumpcontents



nic
\startluacode
thirddata.handlecsv.readline(thirddata.handlecsv.gCurrentLinePointer['lide.csv'])
\stopluacode
\dumpcontents



SSSSSSSSSSS

DOLOOP CYCLE:
 \resetlinepointer
 \doloop{
  \linepointer:
  \readline
  \lineaction\nextline
  \ifnum\linepointer>20\exitloop\fi
 }


\resetlinepointer
\doloop{
 \linepointer:
 \readline
 \lineaction\nextline
 \ifEOF\exitloop\fi
}




 XXX
\setlinepointer{7}
\doloop{%
\ifnum\linepointer<9
 \linepointer:
 \readline
 \lineaction
 \nextline
 \else
 \exitloop
\fi
}


 XXX
\resetlinepointer
\doloop{%
\ifnum\linepointer<9
 \linepointer:
 \readline
 \lineaction
 \nextline
 \else
 \exitloop
\fi
}


% \doloop
%   {Some kind of typesetting punishment \par
%    \ifnum\pageno>5 \exitloop \fi}

DOLOOP CYCLE:DOLOOP CYCLE:DOLOOP CYCLE:DOLOOP CYCLE:DOLOOP CYCLE:


nic \readline


0 \readline{0}

100 \readline{100}

6 \readline{6}

100 \readline{100}

50 \readline{50}

51 \readline{51}

'a' \readline{'a'}

a \readline{a}



\dumpcontents




\startluacode
function test(opt)
--  if (interfaces.tolist(opt) > 0)
--%  	then context('kladne')
--% 	else context('zaporne')
--%  end
    context("%s",interfaces.tolist(opt))
end

interfaces.definecommand {
    name = "test",
    arguments = {
        { "option", "list" },
    },
    macro = test,
}

\stopluacode

test: \test[5]

test \test .....


\opencsvfile{lide.csv}

And now \type{\filelineaction}

\doloopfromto{1}{5}{\lineaction}

\filelineaction


\lastname

\lastname[3]



11111111111111111111111111111111111111111111111111111111111

\def\rowfrommacro#1{\the\numexpr(#1+0)}

\csvcell['A',\rowfrommacro{\numcols}]

\csvcell['A',\linepointer]

\csvcell['A',\numrows]


\startluacode
for i=1,200 do
 context(i..' - '..thirddata.handlecsv.ar2xls(i)..' - '..thirddata.handlecsv.xls2ar(thirddata.handlecsv.ar2xls(i))..'\\crlf')
end
\stopluacode








\stoptext



--~ \startluacode
--~ function test(opt_1, opt_2, arg_1)
--~     context.startnarrower()
--~     context("options 1: %s",interfaces.tolist(opt_1))
--~     context.par()
--~     context("options 2: %s",interfaces.tolist(opt_2))
--~     context.par()
--~     context("argument 1: %s",arg_1)
--~     context.stopnarrower()
--~ end

--~ interfaces.definecommand {
--~     name = "test",
--~     arguments = {
--~         { "option", "list" },
--~         { "option", "hash" },
--~         { "content", "string" },
--~     },
--~     macro = test,
--~ }
--~ \stopluacode

--~ test: \test[1][a=3]{whatever}

--~ \startluacode
--~ local function startmore(opt_1)
--~     context.startnarrower()
--~     context("start more, options: %s",interfaces.tolist(opt_1))
--~     context.startnarrower()
--~ end

--~ local function stopmore(opt_1)
--~     context.stopnarrower()
--~     context("stop more, options: %s",interfaces.tolist(opt_1))
--~     context.stopnarrower()
--~ end

--~ interfaces.definecommand ( "more", {
--~     environment = true,
--~     arguments = {
--~         { "option", "list" },
--~     },
--~     starter = startmore,
--~     stopper = stopmore,
--~ } )
--~ \stopluacode

--~ more: \startmore[1] one \startmore[2] two \stopmore one \stopmore


