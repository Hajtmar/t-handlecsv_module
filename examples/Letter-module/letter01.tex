\input ../incl_handlecsv_file.tex

\setupexternalfigures[location=default]
%\setuplabeltext[cz][figure= ] % musí být ty mezery


\mainlanguage[cz]
\usemodule[letter]
%\setupletteroptions[backgroundimage=mill]

\defineletterelement[layer][head][mylogo]%
   {\framed[background=logohead,height=50mm,frame=off,align=left]%
     {\hlavicka}}

\setupletterlayer[head]
   [x=0mm,
    y=0mm,
   alternative=mylogo]


\input lib-lua

\def\datumodeslani{26. 4. 2010}

\setupletterstyle[bodyfont=10pt, whitespace=small, alternative=doublesided]
% Nastaveni adresniho pole
\setupletterstyle[address][
	style=sansbold,
	preset=lefttop,
	hoffset=14cm%,
	%voffset=7cm,
	]




\setupletterstyle[subject][style=\bfb, align=middle]
\setupletterstyle[reference][style=\ssx,list={cislojednaci, spisovaznacka, misto, datum}]

\setuplabeltext[
	letter:Date={Datum:},
	letter:misto={Místo:},
	letter:cislojednaci={Čís.jedn.:},
	letter:spisovaznacka={Spis.zn.:}]



\startbuffer[dopis]
	\let\SpZn\cA
	\let\CisJedn\cB
	\let\Prijmeni\cC
	\let\Jmeno\cD
	\let\Pohl\cE
	\let\DatNar\cF
	\let\Ulice\cG
	\let\Obec\cH
	\let\PSC\cI
	\let\PrijmeniZZ\cJ
	\let\JmenoZZ\cK
	\let\UliceZZ\cL
	\let\ObecZZ\cM
	\let\PSCZZ\cN
	\let\BP\cO
	\let\BZ\cP
	\let\BS\cQ
	\let\VH\cR
	\let\Poradi\cS



\startletter[
	%fromname={\hlavicka},
	toname={{\red\Jmeno\ \Prijmeni}},
	toaddress={{\red\Ulice\\ \PSC\ \Obec}},
	date={\datumodeslani},
	misto={Město},
	subject={\midaligned{R O~Z~H O~D N U~T Í}},
	closing={\podpisreditele},
	cislojednaci={\red\CisJedn},
	spisovaznacka={\red\SpZn},
	]



\blank[medium]

Ředitel školy, jehož činnost vykonává Název školy, Ulice, 112 32 Město,
jako příslušný správní orgán podle § 165 odst. 2 písm. f) zákona č.
561/2004 Sb., o~předškolním, základním, středním, vyšším odborném a~jiném
vzdělávání (školský zákon), ve znění pozdějších změn a~doplňků, rozhodl
v~souladu s~§ 59, § 60, § 60a, § 60b a~§ 61 téhož zákona, vyhláškou č. 671/2004
Sb., kterou se stanoví podrobnosti o~organizaci přijímacího řízení ke
vzdělávání ve středních školách, ve znění pozdějších změn a~doplňků, a~§~1
odst. 2 zákona č. 500/2004 Sb., správní řád, ve znění pozdějších změn
a~doplňků, takto:



\blank[medium]

{\uchazec}

\blank[big]

\centerline{{\bfa není přijat\a}}

\blank[big]

ke vzdělávání ve~škole, jehož činnost vykonává Název školy, Ulice, 112 32 Město,
od 1.\,9.\,2010 do prvního ročníku nižšího stupně osmiletého
studia, do oboru vzdělání 94-13-K/81 Škola, do denní formy vzdělávání.



\blank[big]

\centerline{{\bf O~d ů v~o~d n ě n í :}}

\blank[medium]

Ředitel školy v~souladu s~§ 60 odst. 3 zákona č. 561/2004 Sb., o~předškolním,
základním, středním, vyšším odborném a~jiném vzdělávání (školský zákon), ve
znění pozdějších změn a~doplňků, stanovil, že do oboru vzdělání uvedeného ve
výroku rozhodnutí bude přijato nejvýše 30 uchazečů.

V~přijímacím řízení byli uchazeči hodnoceni v~souladu s~kritérii přijímacího
řízení vydanými ředitelem gymnázia dne 15.1. 2010 pod č.j. 3/2010.

Výsledné hodnocení uchazeče bylo v~souladu se stanovenými kritérii vyjádřeno
počtem bodů určeným níže uvedeným způsobem:



\blank[medium]

{\bf Body za průměrný prospěch (BP)} se rozumí bodové ohodnocení výsledků klasifikace uchazeče
určené podle vzorce $\rm BP = (6 - \rm P_{4/1} - \rm P_{4/2} - \rm P_{5/1})\times 10$,
kde $\rm P_{4/1},\ P_{4/2} \ a \ P_{5/1}$ je
průměrný prospěch na vysvědčení za 1. pololetí 4. ročníku, za 2. pololetí 4.
ročníku a~za 1. pololetí 5. ročníku základní školy zaokrouhlený na dvě
desetinná místa.

{\bf Body za přijímací zkoušku (BZ)} se rozumí harmonizované skóre dosažené
v~testu obecných studijních předpokladů určené podle kritérií hodnocení testu
stanovených dodavatelem testu - společností www.scio.cz, s.r.o.

{\bf Body za soutěže (BS)} se rozumí body, které jsou přiděleny uchazeči,
který se umístil do třetího místa v~okresním nebo vyšším kole některé ze
soutěží uvedených ve Věstníku MŠMT sešit 8 pro školní rok 2009/2010 pod č.j.
12 702/2009-51 a~označených A1, A2, A3, A4, A5, A6, A7, A8, A10, A11, A13,
nebo v~odpovídajících soutěžích uvedených ve Věstníku MŠMT sešit 8 pro školní
rok 2008/2009 pod č.j.~11~453/2008-51, takto:

BS = 10 bodů za 1. místo v~okresním nebo vyšším kole soutěže,\crlf
BS = 7 bodů za 2. místo v~okresním nebo vyšším kole soutěže,\crlf
BS = 4 body za 3. místo v~okresním nebo vyšším kole soutěže.\par

Umístil-li se uchazeč ve více soutěžích nebo více kolech jedné soutěže,
získává body pouze za tu soutěž nebo kolo soutěže, ve kterém dosáhl nejlepšího
výsledku. Umístění v~soutěži muselo být doloženo diplomem nebo jeho úředně
ověřenou kopií nejpozději v~den konání přijímací zkoušky.


{\bf Výsledným bodovým hodnocením (VH)}  se
rozumí počet bodů, který je součtem bodů za průměrný prospěch, bodů za
přijímací zkoušku a~bodů za soutěže.

{\bf Podmínkou přijetí} uchazeče ke vzdělávání ve výše uvedeném oboru vzdělání bylo
podle stanovených kritérií dosažení výsledného bodového hodnocení {\bf alespoň 30
bodů}.

\blank[medium]

\page[preference]

{\bf Hodnocení uchazeče} bylo vyjádřeno tímto bodovým ziskem:

Body za průměrný prospěch BP = {{\red\BP}} bodů\crlf
Body za přijímací zkoušku BZ = {{\red\BZ}} bodů\crlf
Body za soutěže BS = {{\red\BS}} bodů

{\bf Výsledné bodové hodnocení uchazeče VH = {{\red\VH}} bodů}\crlf

Uchazeč ziskem {{\red\VH}} bodů splnil podmínku pro přijetí stanovenou v~kritériích
přijímacího řízení a~v~pořadí uchazečů podle výsledků hodnocení přijímacího
řízení se umístil na {{\red\bf\Poradi}}. místě. Toto pořadí bylo určeno tak, že uchazeči byli
seřazeni sestupně podle výsledného bodového hodnocení a~v~případě rovnosti
výsledného bodového hodnocení více uchazečů byli tito vzájemně seřazeni
sestupně podle bodů za přijímací zkoušku. Protože podmínky přijímacího řízení
splnilo více uchazečů, než kolik lze přijmout, v~souladu s~§ 60 odst. 14
zákona č. 561/2004 Sb., o~předškolním, základním, středním, vyšším odborném
a~jiném vzdělávání (školský zákon), ve znění pozdějších změn a~doplňků,
rozhodovalo pořadí uchazečů podle výsledků hodnocení přijímacího řízení.
Uchazeč tedy nebyl přijat ke vzdělávání ve střední škole, neboť byla dána
přednost uchazečům, kteří se lépe umístili v~pořadí podle výsledků hodnocení
přijímacího řízení.


\blank[3*big]


\centerline{{\bf P o~u~č e n í~~ o~~~o~d v~o~l á n í :}}

\blank[medium]

Proti rozhodnutí o~nepřijetí uchazeče ke vzdělávání ve střední škole se lze
odvolat podle zákona č. 500/2004 Sb., správní řád, ve znění pozdějších změn
a~doplňků, v~souladu s~§ 60 odst. 19 zákona č. 561/2004 Sb., o~předškolním,
základním, středním, vyšším odborném a~jiném vzdělávání (školský zákon), ve
znění pozdějších změn a~doplňků, do 3 pracovních dnů od doručení tohoto
rozhodnutí k~Odboru školství, mládeže a~tělovýchovy Krajského úřadu
Olomouckého kraje. Odvolání se podává u~ředitele Školy, Adresa, Ulice, PSČ Město.

\blank[3*big]

\stopletter


\page
\stopbuffer


\def\lineaction{\getbuffer[dopis]\page}

\starttext
\setheader

\opencsvfile{letters01.csv}
\doloop{\ifnotEOF\getbuffer[dopis]\nextrow\else\exitloop\fi} %OK

% NEBO :
%\filelineaction{8lneprijat-poradi.csv}{3}{5}

\stoptext

