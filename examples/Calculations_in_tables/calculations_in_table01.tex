% https://mailman.ntg.nl/pipermail/ntg-context/2003/001390.html

\input ../incl_handlecsv_file.tex

\def\eval#1{\directlua{context(tonumber(#1))}}
\def\commatodecimal#1{\directlua{context(string.gsub('#1', "," , "."))}}
\def\priceforallitems{\ctxlua{context(tostring(tonumber([==[\commatodecimal{\columncontent['Unit Price']}]==]*[==[\commatodecimal{\columncontent['Order Quantity']}]==])))}}

\setupTABLE[background=color,backgroundcolor=white, offset=3pt]
\setupTABLE[row][first][background=color,backgroundcolor=lightgray, style=bold]
\setupTABLE[row][last][background=color,backgroundcolor=lightgray, style=bold]
\setupTABLE[c][1][style=\tfx, width=8cm]
\setupTABLE[c][2][align=middle]
\setupTABLE[c][3][alignmentcharacter={.}, aligncharacter=yes, align=middle]
\setupTABLE[c][4][alignmentcharacter={.}, aligncharacter=yes, align=middle, style=\ss]

\def\firsttableline{
\bTR
\bTD Product name\eTD
\bTD Quantity\eTD
\bTD Unit Price\eTD
\bTD Items price\eTD
\eTR
}

\def\lasttableline{
\bTR
\bTD[nc=3]TOTAL PRICE\eTD
\bTD\eval{\totalprice}\eTD
\eTR
}


\def\lineaction{%
\expanded{
\bTR
\bTD\columncontent['Product Name']\eTD
\bTD\columncontent['Order Quantity'] \eTD
\bTD\commatodecimal{\columncontent['Unit Price']}\eTD
\bTD\priceforallitems\eTD
\eTR
}
\eaddto\totalprice{+\priceforallitems}% ADD into
}

\startbuffer[tablerows]
\setlinepointer{\recurselevel}
\expanded{
\bTR
\bTD\columncontent['Product Name']\eTD
\bTD\columncontent['Order Quantity'] \eTD
\bTD\commatodecimal{\columncontent['Unit Price']}\eTD
\bTD\priceforallitems\eTD
\eTR
}
\eaddto\totalprice{+\priceforallitems}% ADD into
\stopbuffer



\starttext

\setheader
\setsep{;}
\catcode`\#=12 %CSV file contains # characters (i.e. TeX problematic character)


\opencsvfile{superstore_sales_smalltable.csv}


\title{Using \type{\dorecurse} cycle}



\def\totalprice{0}

\bTABLE[split=yes]
\firsttableline
\dorecurse{\numrows}{
	\setlinepointer{\recurselevel}
	\lineaction
	\nextrow
}
\lasttableline
\eTABLE

\hairline

\subject{using buffers}

\bTABLE[split=yes]
\firsttableline
\dorecurse{\numrows}{%
	\getbuffer[tablerows]
}
\lasttableline
\eTABLE




\title{Using \type{\filelineaction} cycle}

\def\totalprice{0}

\bTABLE[split=yes]
\firsttableline
\filelineaction
\lasttableline
\eTABLE

\stoptext







\hairline

\columncontent['Product Name']

\columncontent['Order Quantity']

\commatodecimal{\columncontent['Unit Price']}

\ctxlua{context(thirddata.handlecsv.getcellcontent('Unit Price',6))}



%%%%%%%%%%%%%%%%%%%%%%
   \newtoks\everysafeexpanded

   \long\def\safeexpanded#1%
     {\begingroup
      \the\everysafeexpanded\long\xdef\@@expanded{\noexpand#1}%
      \endgroup
      \@@expanded}
\def\somethingsimon{blabla}

\bTABLE
\bTR\safeexpanded{\bTD\Aumlaut}\eTD\eTR
\bTR\safeexpanded{\bTD\noexpand\somethingsimon\Aumlaut}\eTD\eTR
\eTABLE


\def\bTN#1\eTN{\bTD\digits#1\relax\eTD}   % will be in next release

will give you digit alignment

\bTABLE
\bTR\bTD one \eTD\bTN ~25,== \eTN\bTN ~~0,00  \eTN\eTR
\bTR\bTD two \eTD\bTN 125,-- \eTN\bTN `00,00  \eTN\eTR
\eTABLE

see supp-num.tex

%D \starttabulatie[|l|l|l|]
%D \NC \type{.} \NC , . \NC comma or period \NC \NR
%D \NC \type{,} \NC , . \NC comma or period \NC \NR
%D \NC \type{@} \NC \NC invisible space \NC \NR
%D \NC \type{_} \NC \NC invisible space \NC \NR
%D \NC \type{/} \NC \NC invisible sign \NC \NR
%D \NC \type{-} \NC $-$ \NC minus sign \NC \NR
%D \NC \type{+} \NC $+$ \NC plus sign \NC \NR
%D \NC \type{s} \NC \NC invisible high sign \NC \NR
%D \NC \type{p} \NC $\positive$ \NC high plus sign \NC \NR
%D \NC \type{m} \NC $\negative$ \NC high minus sign \NC \NR
%D \NC \type{n} \NC $\negative$ \NC high minus (negative) sign \NC \NR
%D \NC \type{=} \NC $\zeroamount$ \NC zero padding \NC \NR
%D \stoptabulatie

other alignments are discussed there


