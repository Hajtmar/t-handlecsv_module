\input ../incl_handlecsv_file.tex % \usemodule[t-handlecsv_opt.lua] % my option to switch versions

\setuppapersize[A3,landscape]

\setheader

\opencsvfile{ctx-doc.csv}

\def\typeandgetbuffers#1{
{\bf Running  macro:}\hbox{\typebuffer[#1]}
%\typebuffer[#1]
\blank[big]
{\bf Result:}\getbuffer[#1]\crlf%
\hairline
\blank[big]
}

\def\typeandgetanotherbuffers#1{
\getbuffer[#1]\crlf%
\hairline
\blank[big]
}

\def\processbuff#1#2{\hbox{\typebuffer[#1]}%
\noindent {{\bf\type{Code}}}:\type{#2}\crlf%
\noindent {\em Result}:#2\crlf%
\bigskip\hrule\bigskip%
}

% operators: <, >, ==, ~=, >=, <=, eq, neq, in, ~in, repeatuntil, whiledo

\starttext

%\setupalign[flushleft,stretch=never]


\hooksoff

AAA

1.

\bfilehook
\efilehook

2.

\resethooks

3.

\bfilehook
\efilehook

4.

\def\bfilehook{{\bf AAAAA}}
\def\efilehook{{\bf BBBBB}}

5.

\bfilehook
\efilehook

6.

\resethooks
%\resetoriginalhooks

7.

\bfilehook
\efilehook





\def\mojemakro{AA}

\startluacode
  local mojemakrocontent = token.get_macro("mojemakro") or ""
  if content == "" then
    tex.sprint("Macro \\type{\\mojemakro} is empty.")
  else
    tex.sprint("Macro \\type{\\mojemakro} expands to: " .. mojemakrocontent)
  end
\stopluacode


\doloopfromto{1}{6}{nic}
\def\NIC{NIC}
\doloopfromto{1}{6}{\NIC}



\doloopif{1}{==}{1}{nic}%

\doloopif{1}{==}{1}{\NIC}%


POKUS

\thinrule

\hookson


\opencsvfile{ctx-doc.csv}

\def\bfilehook{bTABLE }
\def\efilehook{eTABLE }
\def\blinehook{bTR}
\def\elinehook{eTR}
\def\bch{bTD }
\def\ech{eTD}

\def\lineaction{\cA}
\doloopforall


\hooksoff


\thinrule

NEXT

\doloopfornext{3}{\NIC}





\thinrule

% Cycle (Loop) Macros:
%   \doloopfromto{<start>}{<end>}{<action>}
%      - Executes <action> for each row from <start> to <end>.
%   \doloopforall, \doloopforall{<action>}
%      - Executes <action> for all rows (defaults to \lineaction if omitted).
%   \doloopaction, \doloopaction{<action>}, \doloopaction{<action>}{<end>},
%      \doloopaction{<action>}{<start>}{<end>}
%      - Executes <action> for a specified range of rows.
%   \doloopif{<val1>}{<operator>}{<val2>}{<action>}
%      - Executes <action> on rows satisfying the comparison condition.
%      (Synonym: \doloopifnum for numeric comparisons.)
%   \doloopuntil{<val1>}{<val2>}{<action>}
%      - Repeats <action> until <val1> equals <val2> (alias: \repeatuntil).
%   \doloopwhile{<val1>}{<val2>}{<action>}
%      - Repeats <action> while <val1> equals <val2> (alias: \whiledo).
%   \filelineaction, \filelineaction{<file>}
%      - Executes \lineaction for all rows of the specified CSV file.
%   \doloopfornext{<n>}{<action>}
%      - Executes <action> for the next <n> rows starting from the current row.


doloopfromto{<start>}{<end>}{<action>}

\doloopfromto{2}{4}{\cA xxx \cB zzz \crlf}


doloopforall{<action>}

\doloopforall{\cA xxx \cB zzz \crlf}


\def\lineaction{mmmm \cB xxx \cA zzz \crlf}

doloopaction

\doloopaction


doloopaction{<action>}

\doloopaction{\cA xxx \cB zzz \crlf}

doloopaction{<action>}{<end>}

\doloopaction{\cA xxx \cB zzz \crlf}{4}

doloopaction{<action>}{<start>}{<end>}

\doloopaction{\cA xxx \cB zzz \crlf}{3}{6}



\startbuffer[A)]
\doloopif{\cA}{==}{5}{\cB\cA}%
\stopbuffer
\typeandgetbuffers{A)}


\startbuffer[A)]
\setlinepointer{0}
\doloopif{\cA}{==}{}{\cB\cA}%
\stopbuffer
\typeandgetbuffers{A)}


\startbuffer[A)]
\setlinepointer{0}
\doloopif{\cA}{~=}{}{\cB\cA}%
\stopbuffer
\typeandgetbuffers{A)}

\startbuffer[A)]
\dorecurse{5}{not a break before:\doloopif{\cA}{~=}{}{\cB\cA}\blank}%
\stopbuffer
\typeandgetbuffers{A)}

\startbuffer[A)]
\dorecurse{1}{not a break before:\doloopif{\cA}{~=}{}{\cB \cA}\blank}
\stopbuffer
\typeandgetbuffers{A)}

\startbuffer[A)]
\dorecurse{1}{not a break before:\doloopif{\cB}{==}{colors-mkiv.pdf}{\cB\cA}\blank}
\stopbuffer
\typeandgetbuffers{A)}


\startbuffer[A)]
\dorecurse{1}{not a break before:\doloopif{\cB}{~=}{colors-mkiv.pdf}{\cB\cA}\blank}
\stopbuffer
\typeandgetbuffers{A)}


Compare numbers:

\startbuffer[A)]
\dorecurse{1}{not a break before:\doloopif{\cA}{==}{3}{\cB\cA}}
\stopbuffer
\typeandgetbuffers{A)}


\startbuffer[A)]
\dorecurse{1}{not a break before:\doloopif{\cA}{~=}{3}{\cB\cA}}
\stopbuffer
\typeandgetbuffers{A)}


\startbuffer[A)]
\dorecurse{1}{not a break before:\doloopif{\cA}{>=}{3}{\cB\cA}}
\stopbuffer
\typeandgetbuffers{A)}


\startbuffer[A)]
\dorecurse{1}{not a break before:\doloopif{\cA}{<=}{3}{\cB\cA}}
\stopbuffer
\typeandgetbuffers{A)}


\startbuffer[A)]
\dorecurse{1}{not a break before:\doloopif{\cA}{>}{3}{\cB\cA}}
\stopbuffer
\typeandgetbuffers{A)}

\startbuffer[A)]
\dorecurse{1}{not a break before:\doloopif{\cA}{<}{3}{\cB\cA}}
\stopbuffer
\typeandgetbuffers{A)}

\startbuffer[A)]
\dorecurse{1}{not a break before:\doloopif{\cA}{<>}{3}{\cB\cA}}
\stopbuffer
\typeandgetbuffers{A)}


\startbuffer[A)]
\dorecurse{1}{not a break before:\doloopif{\cA}{=<}{3}{\cB\cA}}
\stopbuffer
\typeandgetbuffers{A)}

\startbuffer[A)]
\dorecurse{1}{not a break before:\doloopif{\cA}{=>}{3}{\cB\cA}}
\stopbuffer
\typeandgetbuffers{A)}


\type{\dorecurse{1}{\doloopif{\cA}{<}{3}{\cB\cA}}}

\doloopif{\cA}{<}{3}{\cB\cA}





\blank[2*big]



\startbuffer[B)]
{\ssa Operators testing:}\crlf

\processbuff{Testing operátor '<':}{\dorecurse{1}{\doloopif{\cA}{<}{3}{\cB\cA}}}

\processbuff{Testing operátor '>':}{\dorecurse{1}{\doloopif{\cA}{>}{3}{\cB\cA}}}%

\processbuff{Testing operátor '==':}{\dorecurse{1}{\doloopif{\cA}{==}{3}{\cB\cA}}}%

\processbuff{Testing operátor '~=':}{\dorecurse{1}{\doloopif{\cA}{~=}{3}{\cB\cA}}}

\processbuff{Testing operátor '>=':}{\dorecurse{1}{\doloopif{\cA}{>=}{3}{\cB\cA}}}

\processbuff{Testing operátor '<=':}{\dorecurse{1}{\doloopif{\cA}{<=}{3}{\cB\cA}}}

\processbuff{Testing operátor '=<':}{\dorecurse{1}{\doloopif{\cA}{=<}{3}{\cB\cA}}}

\processbuff{Testing operátor '=>':}{\dorecurse{1}{\doloopif{\cA}{=>}{3}{\cB\cA}}}

\processbuff{Testing operátor '<>':}{\dorecurse{1}{\doloopif{\cA}{<>}{3}{\cB\cA}}}

\processbuff{Testing operátor 'eq':}{\dorecurse{1}{\doloopif{\cA}{eq}{3}{\cB\cA}}}

\processbuff{Testing operátor 'neq':}{\dorecurse{1}{\doloopif{\cA}{neq}{3}{\cB\cA}}}

\processbuff{Testing operátor 'in':}{\dorecurse{1}{\doloopif{mkiv}{in}{\cB}{\cB\cA}}}

\processbuff{Testing operátor '\type{~}in':}{\dorecurse{1}{\doloopif{i}{~in}{\cB}{\cB\cA}}}

\processbuff{Testing operátor 'repeatuntil':}{\setlinepointer{0}\dorecurse{1}{\doloopif{\cA}{repeatuntil}{3}{\cB\cA}}}

\processbuff{Testing operátor 'whiledo':}{\setlinepointer{0}\dorecurse{1}{\doloopif{\cA}{whiledo}{1}{\cB\cA}}}

\processbuff{Testing 'doloopuntil':}{\setlinepointer{0}\doloopuntil{2}{\cA}{\cB\cA}}

\processbuff{Testing 'repeatuntil':}{\setlinepointer{0}\dorecurse{1}{\repeatuntil{\cA}{3}{\cB\cA}}}

\processbuff{Testing 'doloopwhile':}{\setlinepointer{0}\dorecurse{1}{\doloopwhile{\cA}{1}{\cB\cA}}}

\processbuff{Testing 'whiledo':}{\setlinepointer{0}\dorecurse{1}{\whiledo{\cA}{1}{\cB\cA}}}

\stopbuffer

\typeandgetanotherbuffers{B)}


\thinrule

\resethooks
\hooksoff

\opencsvfile{ctx-doc.csv}



\resetnumline
\def\lineaction{\expanded{\bTR\bTD\cA\eTD\bTD\cB\eTD\eTR}}

\def\emptyrow#1{\bTR\bTD[nc=2,width=.5\textwidth] \type{#1} \eTD\eTR}




\bTABLE
\bTR\bTD ID\eTD\bTD PDF\eTD\eTR

%http://wiki.contextgarden.net/System_Macros/Loops_and_Recursion

\emptyrow{\doloopfromto{3}{5}{\expanded{\bTR\bTD\cA\eTD\bTD\cB\eTD\eTR}}}
\doloopfromto{3}{5}{\expanded{\bTR\bTD\cA\eTD\bTD\cB\eTD\eTR}}


\emptyrow{\doloopif{\cA}{<}{4}{\expanded{\bTR\bTD\cA\eTD\bTD\cB\eTD\eTR}}}
\doloopif{\cA}{<}{4}{\expanded{\bTR\bTD\cA\eTD\bTD\cB\eTD\eTR}}


\emptyrow{\doloopif{\cA}{<}{4}{\lineaction}}
\doloopif{\cA}{<}{4}{\lineaction}


\emptyrow{\doloopaction}
\doloopaction



\emptyrow{\doloopaction{\expanded{\bTR\bTD\cA\eTD\bTD\cB\eTD\eTR}}}
\doloopaction{\expanded{\bTR\bTD\cA\eTD\bTD\cB\eTD\eTR}}


\emptyrow{\doloopaction{\expanded{\bTR\bTD\cA\eTD\bTD\cB\eTD\eTR}}{3}}
\doloopaction{\expanded{\bTR\bTD\cA\eTD\bTD\cB\eTD\eTR}}{3}


\emptyrow{\doloopaction{\expanded{\bTR\bTD\cA\eTD\bTD\cB\eTD\eTR}}{2}{4}}
\doloopaction{\expanded{\bTR\bTD\cA\eTD\bTD\cB\eTD\eTR}}{2}{4} % OK

\eTABLE

\page

\bTABLE
\bTR\bTD ID\eTD\bTD PDF\eTD\eTR

\emptyrow{\setnumline{4}\doloop{\lineaction\nextrow\ifnum\recurselevel>2 \exitloop \fi}}
\setnumline{4}
\doloop{\lineaction\nextrow\ifnum\recurselevel>2 \exitloop \fi}


\emptyrow{\doloopforall}
\doloopforall



\emptyrow{\setnumline{2}\doloop{\ifEOF\exitloop\else\lineaction\nextrow\fi}}
\setnumline{2}
\doloop{\ifEOF\exitloop\else\lineaction\nextrow\fi}


\emptyrow{\setnumline{2}\doloop{\ifnotEOF\lineaction\readrow\else\exitloop\fi}}
\setnumline{2}
\doloop{\ifnotEOF\lineaction\readrow\else\exitloop\fi}


\eTABLE

\thinrule

\closecsvfile{ctx-doc.csv}
\unsetheader
\opencsvfile{ctx-doc.csv}
\filelineaction


\closecsvfile{ctx-doc.csv}
\setheader
\opencsvfile{ctx-doc.csv}
\filelineaction

\thinrule


\closecsvfile{ctx-doc.csv}
\hookson
\unsetheader
\opencsvfile{ctx-doc.csv}


% Vše zařizuje zapnuté hookování...
\def\bfilehook{\bTABLE}
\def\efilehook{\eTABLE}
\def\blinehook{\bTR}
\def\elinehook{\eTR}

\doloopfromto{1}{5}{\expanded{\bTD \cA\eTD \bTD \cB\eTD}}

\thinrule

\closecsvfile{ctx-doc.csv}
\setheader
\hooksoff
\opencsvfile{ctx-doc.csv}

\doloopfornext{3}{\expanded{ \cA \cB}}


\thinrule

\stoptext


