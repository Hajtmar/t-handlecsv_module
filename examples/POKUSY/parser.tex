

\starttext

\startluacode
local crap = [[
first,second,third,fourth
"1","2","3","4"
"a","b","c","d"
"foo","bar""baz","boogie","xyzzy"
]]


local mycsvsplitter = utilities.parsers.rfc4180splitter()
local list = mycsvsplitter(crap,true) 
inspect(list)


--tex.print(list[3][2])

function ParseCSVLine(line,sep)
	local mycsvsplitter = utilities.parsers.rfc4180splitter{
	    separator = sep,
	    quote = '"',
	    strict=true,
	}
	local list = mycsvsplitter(line) inspect(list) 
	return list[1]
end

--local vysledek=ParseCSVLine('AAA:BBB:CCC:DDD',':')

--tex.print(vysledek[2])

\stopluacode

 
\def\nazevpredmetu#1{\ctxlua{context(ParseCSVLine('#1',':')[2])}}%
 
 
\nazevpredmetu{AAA:BBB:CCC:DDD}
 
\ctxlua{context(ParseCSVLine('AAA:BBB:CCC:DDD',':')[1])}
 

% local mycsvsplitter = utilities.parsers.rfc4180splitter()
% 
%
% 
% local list, names = mycsvsplitter(crap,true)   inspect(list) inspect(names)
% 
% 
% -- local list, names = mycsvsplitter(crap)        inspect(list) inspect(names)
% 
% tex.print(list[2])



\stoptext
