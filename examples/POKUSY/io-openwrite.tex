
%\usemodule[/Users/hajtmar/OneDrive/Dokumenty/1da/ConTeXt/ConTeXt-Tests/my_way/ScanCSV/t-handlecsv_module/t-handlecsv]
%\usemodule[/Users/hajtmar/OneDrive/Dokumenty/1da/ConTeXt/ConTeXt-Tests/my_way/ScanCSV/t-handlecsv_module/t-handlecsv-extra]


\usemodule[handlecsv]
\usemodule[handlecsv-extra]


\starttext
\opencsvfile{mail-sample.csv}
\opencsvfile{io-openwrite.csv}

Type all rows of currently opened (io-openwrite.csv) file with initial order:

\doloopforall{\cA\crlf}

% now create temporary (or permanent) file of arbitrary name in reverse order from fil
\writefileinreverseorderfromto{io-openwrite.csv}{tempfile.tmp}
\writecurrfileinreverseorderto{io-openwrite-reverse.csv} % current file is io-openwrite.csv

After opening temporary file (tempfile.tmp) reverse ordered file:

\opencsvfile{tempfile.tmp} % open temporary file

\doloopforall{\cA\crlf} % type all rows of temporary file

\deletefile{tempfile.tmp} % delete temporary file (is not necessary)


Now type reverse order of current processed CSV file (io-openwrite.csv):

\setcurrentcsvfile[io-openwrite.csv] % set to current processed table
\reverseorder % change order of current table

\doloopforall{\cA\crlf} % type all rows, now in reverse order


\reverseorder % change order of current table again (back to initial order)

After change into reverse order of current processed CSV file (io-openwrite.csv) again are rows of table:

\doloopforall{\cA\crlf} % type all rows in reverse order


Reverse order of another opened table (mail-sample.csv):

\reverseorderof{mail-sample.csv}
\setcurrentcsvfile[mail-sample.csv] % set to current processed table

% another possibility is
% \setcurrentcsvfile[mail-sample.csv] % set to current processed table
% \reverseorder % change order of current table

\doloopforall{\cA\crlf} % type all rows in reverse order


Change it back (mail-sample.csv):

\reverseorderof{mail-sample.csv} % change it back
\doloopforall{\cA\crlf} %

\stoptext

