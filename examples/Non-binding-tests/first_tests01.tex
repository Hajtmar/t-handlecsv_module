\usemodule[handlecsvcells]








\starttext

\opencsvfile{numbers.csv}

\unsetheader
\opencsvfile{letters.csv}


\getcsvcell[letters.csv](1,5)


\getcsvcell[letters.csv]('A',5)


\csvfilename

\getcsvcell[numbers.csv](1,9)

\csvfilename

\hairline 

\setsep{,}
\setheader
\opencsvfile{lide.csv}

\csvfilename
%\numrows


\setsep{;}
\setheader
\opencsvfile{countries.csv}

\csvfilename
%\numrows

\numrows[letters.csv]

%\csvfilename -- \numrows


\opencsvfile{countries.csv}

%\csvfilename -- \numrows

\opencsvfile{countries.csv}

%\csvfilename -- \numrows



%\csvfilename -- \numrows -- \numrows[] --


\opencsvfile{letters.csv}

%\csvfilename -- \numrows -- \numrows[]



1 \numrows -- \numrows[]

2 \numrows[lide.csv]

3 \numrows[countries.csv]

4 \numrows[letters.csv]



%\dorecurse{\number\numrows\relax}{\recurselevel \crlf}


%\dorecurse{\numrows[lide.csv]}{\recurselevel \crlf}


%
%1: \getcsvcell[lide.csv](1,9)
%
%A: \getcsvcell[lide.csv]("A",9)
%
%cA: \getcsvcell[lide.csv]("cA",9)
%
%Lastname: \getcsvcell[lide.csv]("Lastname",9)
%
%Last name: \getcsvcell[lide.csv]("Last name",9)
%
%Lastxname: \getcsvcell[lide.csv]("Lastxname",9)
%
%
%Fitstname: \getcsvcell[lide.csv]("Firstname",9)
%
%Firstxname: \getcsvcell[lide.csv]("Firstxname",9)
%
%
%numrows lide: \numrows{lide.csv}
%
%\opencsvfile{letters.csv}
%
%current opened filename: \csvfilename
%
%current opened filename numrows: \numrows

%\dorecurse{\numrows{lide.csv}}{\getcsvcell[lide.csv]("Lastname",\recurselevel)}
%
%
%\dorecurse{\numrows{numbers.csv}}{\getcsvcell[numbers.csv]("B",\recurselevel)}
%
%xxx 
%
%\dorecurse{\numrows{lide.csv}}{\recurselevel. \getcsvcell[numbers.csv](1,\recurselevel) -- \getcsvcell[numbers.csv]('B',\recurselevel) -- \getcsvcell[numbers.csv]('C',\recurselevel) -- \getcsvcell[numbers.csv](7,\recurselevel)\crlf }
%
%xxx 
%
%
%\dorecurse{\numrows{letters.csv}}{\getcsvcell[letters.csv]("B",\recurselevel)}
%
%\dorecurse{\numrows{letters.csv}}{\getcsvcell[letters.csv](2,\recurselevel)}
%
%
%
%
%numrows lide.csv: \numrows{lide.csv}
%
%\csvfilename
%
%numcols lide.csv: \numcols{lide.csv}
%
%numrows numbers.csv: \numrows{numbers.csv}
%
%numrows: \numrows{letters.csv}
%
%\csvfilename
%
%numcols numbers.csv: \numcols{numbers.csv}
%
%numrows letters.csv: \numrows{letters.csv}
%
%numrows countries.csv: \numrows{countries.csv}
%
%numcols letters.csv: \numcols{letters.csv}
%
%numcols: \numcols
%
%countries: \numrows{countries.csv}
%
%\dorecurse{4}{
%
%\recurselevel: \dorecurse{\recurselevel}{\recurselevel x}
%
%}

\stoptext




\def\docommand[#1][#2]%
  {\if!#1!\else#1\fi
  \if!#2!\else#2\fi}

\def\command%
  {\dodoubleargument\docommand}


\def\Docommand[#1]%
  {\if!#1!AAA\else#1\fi}

\def\Command%
  {\dosingleargument\Docommand}
  
  
  
\starttext

1. \command[222][333]

2. \command[222]

3. \command


1. \Command[aaa]

2. \Command



\def\ctnum#1{\ctxlua{if type(tonumber('#1'))=='number' then context('#1') else context(-1) end}}%

\def\Ctnum#1{\ctxlua{if type(tonumber('#1'))=='number' then context('#1') else context(thirddata.handlecsvcells.numrows('#1')) end}}%


\ctnum{5}

\ctnum{aaa}



\optdef\Dumrows[\ctxlua{context(thirddata.handlecsvcells.numrows(''))}]{
	\ifnum\ctnum{\opt}>0 \opt%
	\else\ctxlua{context(thirddata.handlecsvcells.numrows('\opt'))}%
	\fi%
}%

\optdef\Wumrows[\ctxlua{context(thirddata.handlecsvcells.numrows(''))}]{
	\ctnum{\opt}
}%

\optdef\Numrows[\ctxlua{context(thirddata.handlecsvcells.numrows(''))}]{\ctxlua{if type(tonumber('\opt'))=='number' then context('\opt') else context(thirddata.handlecsvcells.numrows('\opt')) end}}%

