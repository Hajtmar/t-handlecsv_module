% Šustek:

\def\pocetradku{\futurelet\mujtoken\pocetradkutest}
\def\pocetradkutest{\ifx\mujtoken[\expandafter\pocetradkusparametrem\else\pocetradkubezparametru\fi}
\def\pocetradkusparametrem[#1]{\message{Makro volano s parametrem "#1".}}
\def\pocetradkubezparametru{\message{Makro volano bez parametru.}}

\def\command{\futurelet\commandtoken\docommand}
\def\docommand{\ifx\commandtoken[\expandafter\commandpar\else\commandnopar\fi}
\def\commandpar[#1]{Parametr je #1.}
\def\commandnopar{Bez parametru}




% Olšák:
% http://petr.olsak.net/opmac-tricks.html

\def\isnextcharA{\the\toks\ifx\tmp\next0\else1\fi\space}
\long\def\isnextchar#1#2#3{\begingroup\toks0={\endgroup#2}\toks1={\endgroup#3}%
   \let\tmp=#1\futurelet\next\isnextcharA
}

\def\sdef#1{\expandafter\def\csname#1\endcsname}

\def\optdef#1[#2]{%
   \def#1{\def\opt{#2}\isnextchar[{\csname oA:\string#1\endcsname}{\csname oB:\string#1\endcsname}}%
   \sdef{oA:\string#1}[##1]{\def\opt{##1}\csname oB:\string#1\nospaceafter\endcsname}%
   \sdef{oB:\string#1\nospaceafter}%
}
\def\nospaceafter#1{\expandafter#1\romannumeral-`\.}



\def\mojemakro{Ahoj}

\optdef\mojemakro [default] #1{opt=\opt, 1=#1}

\optdef\mojedruhemakro[default] #1 #2 {opt=opt, 1=#1, 2=#2.}

  
  
\starttext

\mojemakro

\mojedruhemakro




1. \command[xxx]

2. \command




%1. \dorecurse{\Command[5]}{\recurselevel\crlf}

%2. \dorecurse{\Command}{\recurselevel\crlf}

\stoptext

Subject: Problem with macro with optional parameter
Hello ConTeXist.
I'm not able to solve a macros problem with one optional argument. I need the macro to return a value that can be used as a dorecurse loop parameter.
Is there possibility create macro with optional parameter and parameter is concurrently inside in braces? (in nonsquare brackets)? 
I've tried to experiment with examples from the wiki, but I'm not clear about it.

Thanx for help.
Jaroslav Hajtmar

Here is minimal example:



Dobrý den.
Rád bych zase po čase  "provětral" tuto konferenci dotazem na možnost definice plainových maker s volitelným(i) parametrem(y).
Ačkoliv plaintex prakticky nepoužívám (s výjimkou kompilace starých věcí z archívu), tak mám většinu maker v ConTeXtu plainových.
ConTeXt řeší možnost maker s volitelnými parametry, rád bych se však zeptal v této konferenci, zda lze, popř. jak lze v plainu definovat makro s volitelným (více volitelnými) parametry. Koukal jsem do TBN a na podobnou věc jsem nenarazil. Dále bych rád upozornil, že ačkoliv jsem schopen svůj problém řešit makry různých názvů (některé bez parametru jiné s parametry), tak mne jisté důvody (kompatibilita se staršími verzemi mých dokumentů) vedou k tomu, abych se pokusil použít řešení pomocí volitelných parametrů.

Rád bych měl tedy makro s názvem např.: \pocetradku, které by vracelo něco jíného než makro \pocetradku{<jmenosouboru>} resp. možná lépe pro mne \pocetradku[<jmenosouboru>].
Makro \pocetradku by vracelo počet řádků jakéhosi defaultního - v paměti aktuálního souboru, tzn. bylo by ekvivalentní s makrem \pocetradku[], zatímco \pocetradku[<jmenosouboru>] by vracelo počet řádků nějakého jiného souboru. Při experimentování jsem narazil meze svých znalostí plainu a proto bych se rád obrátil na konferenci. Může mne někdo odkázat na nějaké relevantní zdroj popř. mi nějak poradit? 
Na internetu jsem našel nějaké podovné věci dělané v LaTeXu, to je pro mne však nepoužitelné... 
Rád přijmu i to, že mi někdo rozmluví můj problém řešit zrovna takto .... :-).


Díky za případné tipy a rady

Zdraví
Jarda Hajtmar


 


