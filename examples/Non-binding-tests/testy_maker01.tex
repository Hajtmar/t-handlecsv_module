\usemodule[handlecsv] % use new module

  
\starttext

\setheader


\opencsvfile{cities.csv}

\opencsvfile{selectedcountries.csv}

\opencsvfile{selectedcountries.csv}

\opencsvfile{selectedcountries.csv}

\opencsvfile{cities.csv}

\opencsvfile{superstore_sales_smalltable.csv}

\opencsvfile{cities.csv}

\opencsvfile{superstore_sales_smalltable.csv}

\type{\setcurrentcsvfile{selectedcountries.csv}} \setcurrentcsvfile[selectedcountries.csv]   %- stejné jako \setfiletoscan{cities.csv}

\type{\numrows:} \numrows

\type{\numrowsof[selectedcountries.csv]:} \numrowsof[selectedcountries.csv]

\type{\numrowsof[cities.csv]:} \numrowsof[cities.csv]


\type{\numcols:} \numcols

gNumCols selectedcountries.csv:  \ctxlua{context(thirddata.handlecsv.gNumCols['selectedcountries.csv'])}

gNumCols selectedcountries.csv:  \ctxlua{context(thirddata.handlecsv.gNumCols['cities.csv'])}

handlecsv.handlecsvfilename()):  \ctxlua{context(thirddata.handlecsv.setgetcurrentcsvfile())}


\type{\numcolsof[selectedcountries.csv]:} \numcolsof[selectedcountries.csv]

\type{\numcolsof[cities.csv]:} \numcolsof[cities.csv]

\type{\linepointer \csvfilename} \csvfilename -- \linepointer 

\hairline

\type{\setfiletoscan{cities.csv}} \setfiletoscan{cities.csv}  %- stejné jako \setcurrentcsvfile{cities.csv}
%\opencsvfile

\type{\numrows:} \numrows

\type{\numrowsof[selectedcountries.csv]:} \numrowsof[selectedcountries.csv]

\type{\numrowsof[cities.csv]:} \numrowsof[cities.csv]

\type{\numcols:} \numcols

\type{\numcolsof[selectedcountries.csv]:} \numcolsof[selectedcountries.csv]

\type{\numcolsof[cities.csv]:} \numcolsof[cities.csv]

\type{\linepointer \csvfilename} \csvfilename -- \linepointer 

\hairline



\type{\numcols:} \numcols


\type{\csvfilename:} \csvfilename



\type{\citiescA:} \citiescA

\type{\cA, \cB, \cC:} \cA, \cB, \cC

\type{\linepointer \csvfilename} \csvfilename -- \linepointer 

\type{\nextrow: } \nextrow

\type{\cA, \cB, \cC:} \cA, \cB, \cC

\type{\linepointer \csvfilename} \csvfilename -- \linepointer 

\type{\readline} \readline

\type{\cA, \cB, \cC:} \cA, \cB, \cC

\type{\readline[selectedcountries.csv]} \readline[selectedcountries.csv]

\type{\nextrow: } \nextrow

\type{\cA, \cB, \cC:} \cA, \cB, \cC

\type{\linepointer \csvfilename} \csvfilename -- \linepointer 

\type{\nextline: } \nextline

\type{\cA, \cB, \cC:} \cA, \cB, \cC




\dorecurse{\numrows}{
\recurselevel -- \linepointer
}

\hairline

selectedcountries.csv : \numrowsof[selectedcountries.csv]

%\resetlinepointerof[selectedcountries.csv]

\opencsvfile{cities.csv}

\dorecurse{\numrowsof[selectedcountries.csv]}
{

 \recurselevel -- \linepointerof[selectedcountries.csv] 

%\setlinepointerof[selectedcountries.csv]

linepointer: \ctxlua{context(thirddata.handlecsv.gCurrentLinePointer['selectedcountries.csv'])}

numrows: \ctxlua{context(thirddata.handlecsv.gNumRows['selectedcountries.csv'])}

COUNTRIES

cA \cA -- colA  \colA -- colA recurselevel \colA[\recurselevel]\crlf 
cB \cB -- colB \colB -- colB recurselevel \colB[\recurselevel]\crlf 
cC \cC -- colC \colC -- colC recurselevel \colC[\recurselevel]

\edef\CA{\cA}
\edef\COLA{\colA}

CA COLA {\CA--\COLA}


\ifx\CO\COLA Hurá \else PRD \fi

\doifelse{\cA}{\colA}{Hurá 1 ...}{Sorry jako 1 !}

\doifelse{\cA}{5}{Hurá 2 ...}{Sorry jako 2 !}

\doifelse{5}{\cA}{Hurá 2 ...}{Sorry jako 2 !}

\doifelse{5}{\colA}{Hurá 3 ...}{Sorry jako 3 !}

\doifelse{\colA}{5}{Hurá 4 ...}{Sorry jako 4 !}

CITIES

%\cA, \cB, \cC


\nextrow

%\nextrowof[selectedcountries.csv]

}

\hairline

\def\OK{AAA}

\def\AAA{AAA}
\def\aaa{AAA}
\def\Aaa{\OK}

\doifelse{\AAA}{\aaa}{OK}{not OK}

\doifelse{\AAA}{\Aaa}{OK}{not OK}

\def\AAA[#1]{#1}

\def\aaa{AAA}

\doifelse{\AAA[AAA]}{\aaa}{OK}{not OK}





\hairline

\type{\resetlinepointer:} \resetlinepointer

\type{\linepointer:} \linepointer

\type{\setlinepointer{5}:} \setlinepointer{5}

\type{\linepointer:} \linepointer



\type{\resetlinepointerof[selectedcountries.csv]:} \resetlinepointerof[selectedcountries.csv]

\type{\linepointerof[selectedcountries.csv]:} \linepointerof[selectedcountries.csv]

\type{\setlinepointerof[selectedcountries.csv]{7}:} \setlinepointerof[selectedcountries.csv]{7}

\type{\linepointerof[selectedcountries.csv]:} \linepointerof[selectedcountries.csv]


\hairline

\def\GV[#1,#2,#3]{
\ctxlua{context(thirddata.handlecsv.getcellcontentof('#1',#2,#3))}
}

\def\GW[#1,#2]{
\ctxlua{context(thirddata.handlecsv.getcellcontent(#1,#2))}
}

\dorecurse{10}{
\GV[cities.csv,2,\recurselevel]\crlf
}

\hairline

\setcurrentcsvfile[selectedcountries.csv]
\dorecurse{10}{
\GW[1,\recurselevel]\crlf
}

\hairline
\setcurrentcsvfile[cities.csv]
\dorecurse{10}{
\GW[2,\recurselevel]\crlf
}

XXX

\opencsvfile{superstore_sales_smalltable.csv}

\columncontent['Product Name']

XXX

\def\GCN#1{
\ctxlua{context(thirddata.handlecsv.gOpenFiles[#1])}
}

\ctxlua{
count = 0
for k,v in pairs(thirddata.handlecsv.gOpenFiles) do
     count = count + 1
end
context(count)
}

\ctxlua{
context(thirddata.handlecsv.isopenfile('superstore_sales_smalltable.csv'))
}

\ctxlua{
context(thirddata.handlecsv.isopenfile('superstore_sales_smalltable.cv'))
}



*\GCN{'superstore_sales_smalltable.csv'}*

gColNames: {\ctxlua{context(thirddata.handlecsv.gColNames['superstore_sales_smalltable.csv']['Product Name'])}}

gColNames 2: {\ctxlua{context(thirddata.handlecsv.gColNames[thirddata.handlecsv.getcurrentcsvfilename()]['Product Name'])}}

Columncontent['Product Name'] \columncontent['Product Name']

columncontent['Product Name'] \columncontent['Product Name']

linepointer: {\ctxlua{context(thirddata.handlecsv.linepointer())}}

getcellcontent 1:  {\ctxlua{context(thirddata.handlecsv.getcellcontent(thirddata.handlecsv.gColNames['superstore_sales_smalltable.csv']['Product Name'],thirddata.handlecsv.linepointer()))}}

getcellcontent 2: {\ctxlua{context(thirddata.handlecsv.getcellcontent(thirddata.handlecsv.gColNames[thirddata.handlecsv.getcurrentcsvfilename()]['Product Name'],thirddata.handlecsv.linepointer()))}}




get number of open files: {\ctxlua{context(thirddata.handlecsv.getnumberofopencsvfiles())}



\hairline

\readlineof[cities.csv]{5}

\cA -- \citiescA

\cB -- \citiescB

\hairline


\readline

\cB

\nextrow

\readline

\cB


\readline{5}

\cA, \cB, \cC

\readline{6}

\cA, \cB, \cC


\readlineof[selectedcountries.csv]{6}


\selectedcountriescA -- \cA

\selectedcountriescB -- \cB

\type{\cC} ??? \cC % na tohle dát pozor, protože se při menším počtu sloupců nevyresetuje hodnota starých maker


\currentcsvfile

\resetlinepointer

\cA

\cB

\opencsvfile{selectedcountries.csv}

\cA, \cB, \cC

\hairline
 
 \getcsvcellof[selectedcountries.csv]['Abbrev',5]


\getcsvcellof[cities.csv]['CountryAbbrev',7]

\cB


\hairline

\opencsvfile{selectedcountries.csv}

\opencsvfile{cities.csv}

\readlineof[cities.csv]{5}

cities - 5: \cA, \cB, \cC



\readlineof[selectedcountries.csv]{5}

5 - countries:\cA, \cB, \cC


\opencsvfile{cities.csv}

\readline{7}

cities 7: \cA, \cB, \cC


\hairline

\directlua{context(thirddata.handlecsv.getcurrentcsvfilename())}

selectedcountries.csv

\directlua{context(thirddata.handlecsv.gCurrentlyProcessedCSVFile)}


\opencsvfile{selectedcountries.csv}

%\readlineof[,5]

\cA, \cB, \cC



%columnA: \columnA{6} 

%\columnB{}

%DEFINICE: \definice{6}


%DEFINICE: \definice{1}

%DEFINICE: \definice{8}

columnA 1 \columnA{1}

columnA 6  \columnA{6}

column B 4 \columnA{4} \columnB{4}

column  \columnA{} \columnB{}

\nextrowof[selectedcountries.csv]

column  \columnA{} \columnB{}

\nextrowof[selectedcountries.csv]

column  \columnA{} \columnB{}

\nextrowof[selectedcountries.csv]

column  \columnA{} \columnB{}

\nextrowof[selectedcountries.csv]

column bez zavoeres  \columnA \columnB



column CCC: 
BBB\columnA{AAA}BBB
AAA\columnB{BBB}AAA

colB \colB

colB 6 \colB[6]

\resetlinepointer

\dorecurse{\numrows}{
\recurselevel -- \linepointer -- \columnA{}, \columnB{}\crlf

%\edef\mymacro{\columnB{\recurselevel}}
\edef\mymacro{\getcsvcell['cA',\recurselevel]}

\edef\abbrevcountry{\Abbrev}

\def\US{US}

\if\mymacro\US OK \else NOT \fi \crlf

\doif{\getcsvcell['cA',\recurselevel]}{\US}{equal content}

\doifsamestringelse{\mymacro}{US}{equal content}{not equal content}

\doifinstringelse{\mymacro}{US}{equal content}{not equal content}

\doifincsnameelse{\mymacro}{US}{equal content}{not equal content}



%\recurselevel -- \columnB{} -- \test -- \doifelse{\test}{\US}{OK}{Not Ok} ---- \crlf
%\recurselevel -- \columnB{9} -- \test -- {\columnB{9}} -- \doifsamestringelse{\test}{\US}{OK}{Not Ok} ---- \crlf

\nextrow
}


XXXXX

\setcurrentcsvfile[cities.csv]
\resetlinepointerof[cities.csv]

\edef\abbrevcountry{\Abbrev}

\dorecurse{\numrowsof[cities.csv]}{% For all cities

\recurselevel: \CountryAbbrev[\recurselevel], 
\citiescE, 
\colE[\recurselevel] -- \colE, 
\columnE{\recurselevel},




\doif{\getcsvcellof[cities.csv]['CountryAbbrev',\recurselevel]}{US}{\recurselevel -- \Name\crlf } % if city is in country then write it, else nothing
% NOT OK \expdoif{\citiesCountryAbbrev[\recurselevel]}{US}{Huráááááááá \recurselevel -- \Name\crlf } % if city is in country then write it, else nothing
% NOT OK \expdoif{\columnE{\recurselevel}}{US}{Huráááááááá \recurselevel -- \Name\crlf } % if city is in country then write it, else nothing
%OK \expdoif{\citiescE}{US}{Huráááááááá \recurselevel -- \Name\crlf } % if city is in country then write it, else nothing
% NOT OK \expdoif{\colE[\recurselevel]}{US}{Huráááááááá \recurselevel -- \Name\crlf } % if city is in country then write it, else nothing
% NOT OK \expdoif{\colE}{US}{Huráááááááá \recurselevel -- \Name\crlf } % if city is in country then write it, else nothing
\nextrowof[cities.csv]
}



%\dorecurse{\numrowsof[cities.csv]}{% For all cities
\resetlinepointerof[cities.csv]

%{\bf \getcsvcellof[cities.csv]['CountryAbbrev',\recurselevel]:}

%\columnE{\recurselevel}


\define[1]\ListEntry{
\doloopif{\cE}{==}{#1}{Huráááááááá \getcsvcellof[cities.csv]['CountryAbbrev',\linepointer] -- \Name\crlf}
}

%\doloopif{\cE}{==}{\getcsvcellof[cities.csv]['CountryAbbrev',\recurselevel]}{Huráááááááá \getcsvcellof[cities.csv]['CountryAbbrev',\recurselevel] -- \recurselevel -- \Name\crlf}  % 
\processcommalist[CN,TH,IQ,JP]{\ListEntry}

%}


% DefiniceA :\definiceA{5}:


\startluacode
context(thirddata.handlecsv.getcellcontentof('cities.csv',0,0))
\stopluacode


\stoptext
































\dorecurse{5}{ % For all countries from selectedcountries.csv file DO:

}


\hairline

\opencsvfile{selectedcountries.csv}

\cA
 
\getcsvcellof[selectedcountries.csv]['Abbrev',5]


\getcsvcellof[cities.csv]['CountryAbbrev',7]

\cB



\startluacode

local row="145"
local testrow=123

	if type(row)=='string' then
		local testrow=tonumber(row)
		if testrow==nil then
		  row=0
		else
		   row=testrow
		end
	end
	 context(row)

context("\\crlf")	

context(testrow)

				for i=1,thirddata.handlecsv.gNumRows['cities.csv'] do
					for j=1,thirddata.handlecsv.gNumCols['cities.csv'] do
				context(thirddata.handlecsv.gTableRows['cities.csv'][i][j])
				local stri=tostring(i)
				local strj=tostring(j)
--				thirddata.handlecsv.gTableRows['cities.csv'][stri][strj]=thirddata.handlecsv.gTableRows['cities.csv'][i][j]
				end
				end



\stopluacode
