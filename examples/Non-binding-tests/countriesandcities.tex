\usemodule[handlecsv]
%
%\setupTABLE[split=yes,location={middle,lohi}, height=4.5mm]
%\setupTABLE[row][first][background=color,backgroundcolor=gray,foregroundcolor=black,height=8mm, align={middle,lohi}]
%\setupTABLE[row][last][background=color,backgroundcolor=gray,foregroundcolor=black,height=8mm,align={middle,lohi}]
%\setupTABLE[c][1][background=color,backgroundcolor=gray,align=middle, width=7mm]
%\setupTABLE[c][2][align={right, lohi}, width=4cm]
%\setupTABLE[c][3][align={middle, lohi}, width=2.5cm]
%\setupTABLE[c][4][align={right, lohi}, width=7cm]
%
%
%
%  
%\startbuffer[tableaction]
%\expanded{\bTR\bTD\SerNum\eTD\bTD\Name\eTD\bTD\Population\eTD\bTD\Altitude\eTD\eTR}%
%\stopbuffer
%
%
%
%\startbuffer{xxx}
%\bTABLE
%\bTR\bTD\ssx SerNum \eTD\bTD\ssx Name\eTD\bTD\ssx Population\eTD\bTD\ssx Altitude\eTD\eTR
%
%\bTR\bTD-- \eTD\bTD\ssx -- \eTD\bTD -- \eTD\bTD -- \eTD\eTR
%\eTABLE
%
%\page[no]	
%
%\stopbuffer
%


  
\starttext

\setheader
\opencsvfile{cities.csv}
\opencsvfile{countries.csv}


  	
\blank[big]

\startitemize[n]
\dorecurse{\numrowsof[countries.csv]}{%
\edef\abbrevcountry{\getcsvcellof[countries.csv]['Abbrev',\recurselevel]}%
\item {\bf \getcsvcellof[countries.csv]['Country',\recurselevel]}
\startitemize[n]
\dorecurse{\numrowsof[cities.csv]}%
{\doifelse{\getcsvcellof[cities.csv]['CountryAbbrev',\recurselevel]}{\abbrevcountry}{\item \getcsvcellof[cities.csv]['Name',\recurselevel] }{}% %\if\abbrevcity\recurselevel: \getcsvcellof[cities.csv]('Name',\recurselevel)\fi	
}%
\stopitemize
\blank[big]
}%
\stopitemize



\stoptext



%\if\getcsvcellof[countries.csv]('Abbrev',\recurselevel]\getcsvcellof[cities.csv]('CountryAbbrev',] \CountryAbbrev\CountryAbbrev\tableaction\fi
%\doloopif{\CountryAbbrev}{==}{\env{CountryAbbrev}}{\tableaction}


\getcsvcellof[letters.csv][1,5]

\getcsvcellof[letters.csv]['A',5]

\getcsvcellof[numbers.csv][1,9]

\hairline 

\setsep{,}
\setheader
\opencsvfile{lide.csv}


1: \getcsvcellof[lide.csv][1,9]

A: \getcsvcellof[lide.csv]["A",9]

cA: \getcsvcellof[lide.csv]["cA",9]

Lastname: \getcsvcellof[lide.csv]["Lastname",9]

Last name: \getcsvcellof[lide.csv]["Last name",9]

Lastxname: \getcsvcellof[lide.csv]["Lastxname",9]


Fitstname: \getcsvcellof[lide.csv]["Firstname",9]

Firstxname: \getcsvcellof[lide.csv]["Firstxname",9]

\dorecurse{\numrows[lide.csv]}{\getcsvcellof[lide.csv]["Lastname",\recurselevel]}


\dorecurse{\numrows[numbers.csv]}{\getcsvcellof[numbers.csv]["B",\recurselevel]}

xxx 

\dorecurse{\numrows[lide.csv]}{\recurselevel. \getcsvcellof[numbers.csv][1,\recurselevel] -- \getcsvcellof[numbers.csv]['B',\recurselevel] -- \getcsvcellof[numbers.csv]['C',\recurselevel] -- \getcsvcellof[numbers.csv][7,\recurselevel]\crlf }

xxx 


\dorecurse{\numrows[letters.csv]}{\getcsvcellof[letters.csv]["B",\recurselevel]}

\dorecurse{\numrows[letters.csv]}{\getcsvcellof[letters.csv][2,\recurselevel]}




numrows lide.csv: \numrows[lide.csv]

numcols lide.csv: \numcols{lide.csv}

numrows numbers.csv: \numrows[numbers.csv]

numcols numbers.csv: \numcols{numbers.csv}

numrows letters.csv: \numrows[letters.csv]

numcols letters.csv: \numcols{letters.csv}

numcols: \numcols


\stoptext





\midaligned{\ssb List of biggest cities of \env{Country} (\env{CountryAbbrev})}

\blank[big]

\setheader 
\opencsvfile{cities.csv}

\bTABLE
\bTR\bTD\ssx SerNum \eTD\bTD\ssx Name\eTD\bTD\ssx Population\eTD\bTD\ssx Altitude\eTD\eTR
\doloopif{\CountryAbbrev}{==}{\env{CountryAbbrev}}{\tableaction}
\bTR\bTD-- \eTD\bTD\ssx -- \eTD\bTD -- \eTD\bTD -- \eTD\eTR
\eTABLE

\page[no]	