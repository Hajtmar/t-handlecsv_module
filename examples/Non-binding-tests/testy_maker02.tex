\input ../incl_handlecsv_file.tex % \usemodule[t-handlecsv_opt.lua]


\starttext

\setheader
\enablemode[XXL]
\opencsvfile{cities.csv}
\disablemode[XXL]
\opencsvfile{selectedcountries.csv}
\enablemode[XXL]
\opencsvfile{cities.csv}
\disablemode[XXL]

1. \currentcsvfile

\setcurrentcsvfile[selectedcountries.csv]

2. \currentcsvfile

\opencsvfile{cities.csv}

3. \currentcsvfile

\setcurrentcsvfile[selectedcountries.csv]

4. \currentcsvfile

\setcurrentcsvfile[cities.csv]

5. \currentcsvfile
\hairline
\hairline
\hairline

\type{\cA, \cB, \cC: } \cA, \cB, \cC

\nextrow

\type{\cA, \cB, \cC (after \nextrow): } \cA, \cB, \cC


%\resetlinepointer

\setlinepointer{82}

\type{\colA, \colB, \colC: } \colA, \colB, \colC

\nextrow

\type{\colA, \colB, \colC (after \nextrow): } \colA, \colB, \colC


\type{\colA[1], \colA[2], \colA[3], \colB[1], \colB[2], \colB[3]: } \colA[1], \colA[2], \colA[3], \colB[1], \colB[2], \colB[3]

\type{\cA, \cB, \cC (after \setlinepointer{82}): } \cA, \cB, \cC

\nextrow

\type{\cA, \cB, \cC (after \nextrow): } \cA, \cB, \cC


\type{\cA[1], \cA[2], \cA[3], \cB[1], \cB[2], \cB[83]: } \cA[1], \cA[2], \cA[3], \cB[1], \cB[2], \cB[83]


\type{\SerNum, \Name, \Population: } \SerNum, \Name, \Population

\nextrow

\type{\colSerNum, \colName, \colPopulation (after nextrow): } \colSerNum, \colName, \colPopulation

\type{\colSerNum[6], \colName[2], \colPopulation[82]: } \colSerNum[6], \colName[2], \colPopulation[82]





\hairline
\hairline
\hairline

\stoptext
