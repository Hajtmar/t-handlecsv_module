
\input ../incl_handlecsv_file.tex % \usemodule[t-handlecsv_opt.lua]
\setheader
\opencsvfile{cities.csv}


\setupTABLE[split=yes,location={middle,lohi}, style=\ssx]
\setupTABLE[row][first][background=color,backgroundcolor=gray,foregroundcolor=black,height=8mm, align={middle,lohi}]
\setupTABLE[row][last][background=color,backgroundcolor=gray,foregroundcolor=black,height=8mm,align={middle,lohi}]
\setupTABLE[c][1][background=color,backgroundcolor=gray,align=middle, width=15mm,style=\ssxx]
\setupTABLE[c][2][width=4cm,style=\ssxx]
\setupTABLE[c][3][width=2.5cm,style=\ssxx]
\setupTABLE[c][4][width=7cm,style=\ssxx]


\def\Name{\getcsvcellof[cities.csv]['Name',\recurselevel]}
\def\SerNum{\getcsvcellof[cities.csv]['SerNum',\recurselevel]}
\def\Population{\getcsvcellof[cities.csv]['Population',\recurselevel]}
\def\Altitude{\getcsvcellof[cities.csv]['Altitude',\recurselevel]}
\def\numcountries{\numrowsof[selectedcountries.csv]}
\def\numcities{\numrowsof[cities.csv]}


\startbuffer[tableaction]
\expanded{\bTR\bTD\SerNum \eTD\bTD\Name \eTD\bTD\Population\eTD\bTD\Altitude\eTD\eTR}%
\stopbuffer


\starttext

\title{Example of together using of two opened CSV files at the same time}




CSV file \type{selectedcountries.csv} consist of information about countries, file \type{cities.csv} consist of information about cities.
Relation is mediated through the country abbreviation field \type{Abbrev}  (in \type{selectedcountries.csv}) and field \type{CountryAbbrev} (in \type{cities.csv} file).

\blank[big]

% Open two CSV files at one time together



\readlineof[cities.csv]{5}


xxx

\cA

zzzz

\readline
\cA

xxx

\readline{5}
\cA, \cB, \cC

zzz

\resetlinepointer
\cA

\cB

\opencsvfile{selectedcountries.csv}

\cA

 \getcsvcellof[selectedcountries.csv]['Abbrev',5]


\getcsvcellof[cities.csv]['CountryAbbrev',7]

\cB



Numcities \numcities

Numcountries: \numcountries

\opencsvfile{selectedcountries.csv}

\opencsvfile{cities.csv}

1111111
\readlineof[cities.csv]{5}
\cA, \cB, \cC

222222

\readlineof[selectedcountries.csv]{5}
\cA, \cB, \cC

3333333

\opencsvfile{cities.csv}
\readline{7}
\cA, \cB, \cC



qqqqq


\directlua{context(thirddata.handlecsv.getcurrentcsvfilename())}


\directlua{context(thirddata.handlecsv.gCurrentlyProcessedCSVFile,"xxx")}

qqqq

\opencsvfile{selectedcountries.csv}

%\readlineof[,5]

\cA, \cB, \cC

\setcurrentcsvfile[cities.csv]

\dorecurse{\numcountries}{ % For all countries from selectedcountries.csv file DO:
   \edef\abbrevcountry{\getcsvcellof[selectedcountries.csv]['Abbrev',\recurselevel]}

	{\bf Cities of \getcsvcellof[selectedcountries.csv]['Country',\recurselevel]:}
	\resetlinepointerof[cities.csv]

	\blank[big]

	\bTABLE
	\bTR\bTD Id \eTD\bTD Name\eTD\bTD Population\eTD\bTD Altitude\eTD\eTR
	 \resetlinepointer
		\dorecurse{\numcities}{% For all cities
			%\doifelse{\getcsvcell[cities.csv]['CountryAbbrev',\recurselevel]}{\abbrevcountry}{\getbuffer[tableaction] }{} % if city is in country then write it, else nothing
			\expdoif{\getcsvcellof[cities.csv]['CountryAbbrev',\recurselevel]}{\abbrevcountry}{\getbuffer[tableaction] } % if city is in country then write it, else nothing
		\nextrow
		}
	\bTR\bTD-- \eTD\bTD\ssx -- \eTD\bTD -- \eTD\bTD -- \eTD\eTR
	\eTABLE
	\blank[big]
}


\hairline
\opencsvfile{selectedcountries.csv}

%\cA

 \getcsvcellof[selectedcountries.csv]['Abbrev',5]


\getcsvcellof[cities.csv]['CountryAbbrev',7]

%\cB



\stoptext
