%\usepath[{/Users/hajtmar/Documents/1da/GitHub-projekty/t-handlecsv_module/,}]
%\usemodule[t-handlecsv-opt_by_ai.lua] % poslední verze

\input ../incl_handlecsv_file.tex
\usemodule[/Users/hajtmar/Documents/1da/GitHub-projekty/t-handlecsv_module/] % poslední verze



\starttext



\setsep{;}
\setheader
\opencsvfile{countries.csv}

\csvfilename
\numrows

\numrows[letters.csv]

\csvfilename -- \numrows


\opencsvfile{countries.csv}

\csvfilename -- \numrows


\opencsvfile{countries.csv}

\csvfilename -- \numrows


\csvfilename -- \numrows -- \numrows[] --


\opencsvfile{letters.csv}

\csvfilename -- \numrows -- \numrows[]




\dorecurse{\numrows}{\recurselevel \crlf}







Otevírám CSV soubor letters.csv\par

\opencsvfile{letters.csv}\par

Počet řádků načtených ze souboru: \numrows\par





LETTERS.CSV

\opencsvfile{letters.csv}

\ctxlua{tex.sprint("Current file: ["..thirddata.handlecsv.getcurrentcsvfilename().."]")}

TEST:

\ctxlua{
  local f = thirddata.handlecsv.getcurrentcsvfilename() or "nil"
  local n = thirddata.handlecsv.gNumRows[f] or "nil"
  tex.sprint("File: ["..f.."], rows: ["..n.."]")
}



NUMROWS: \numrows


tex.sprint

\ctxlua{tex.sprint("["..thirddata.handlecsv.numrows().."]")}

SSSSSSS


dorecurse\{numrows\}\{recurselevel \\crlf\}

\dorecurse{\numrows}{\readline\recurselevel -- yyyyy \cA -- \cB \crlf\nextrow}




%\dorecurse{\numrows}{\recurselevel \crlf}


T E S T

Otevírám CSV soubor letters.csv\par

\opencsvfile{letters.csv}\par

\ctxlua{
  local f = thirddata.handlecsv.getcurrentcsvfilename() or "none"
  local n = thirddata.handlecsv.gNumRows[f] or "nil"
  tex.sprint("Aktuální soubor: ["..f.."], počet řádků: ["..n.."]")
}
\par



\csvreport


\stoptext


