\usemodule[handlecsv]

\setuppagenumbering[state=stop]
%\setupbodyfont[regular, 8pt]

%http://wiki.contextgarden.net/TABLE
\setupTABLE[split=yes,location={middle,lohi}, height=6mm]
\setupTABLE[row][first][background=color,backgroundcolor=gray,foregroundcolor=black,height=8mm, align={middle,lohi}]
\setupTABLE[row][last][background=color,backgroundcolor=gray,foregroundcolor=black,height=8mm,align={middle,lohi}]
\setupTABLE[c][1][background=color,backgroundcolor=gray,align=middle, width=20mm]
\setupTABLE[c][2][align={right, lohi}, width=4cm]
\setupTABLE[c][3][align={middle, lohi}, width=2.5cm]
\setupTABLE[c][4][align={right, lohi}, width=7cm]


\startbuffer[rowoftable]
\expanded{\bTR\bTD\SerNum\eTD\bTD\Name\eTD\bTD\Population\eTD\bTD\Altitude\eTD\eTR}%
\stopbuffer
	
	
\starttext
\blank[big]

\setheader 
\opencsvfile{cities.csv}
\opencsvfile{selectedcountries.csv}

\dorecurse{\numrowsof[selectedcountries.csv]}{

\readline{\recurselevel}
\midaligned{\ssb List of biggest cities of \Country -- \Abbrev}

\blank[big]

\edef\abbrevcountry{\Abbrev}

\bTABLE
\bTR\bTD\ssx SerNum \eTD\bTD\ssx Name\eTD\bTD\ssx Population\eTD\bTD\ssx Altitude\eTD\eTR

\resetlinepointerof[cities.csv]
\dorecurse{\numrowsof[cities.csv]}{% For all cities
	\doif{\getcsvcellof[cities.csv]['CountryAbbrev',\recurselevel]}{\abbrevcountry}{\getbuffer[rowoftable] } % if city is in country then write it, else nothing
	\nextrowof[cities.csv]
}

\bTR\bTD-- \eTD\bTD\ssx -- \eTD\bTD -- \eTD\bTD -- \eTD\eTR
\eTABLE
\nextrow
\page	
}


\stoptext
