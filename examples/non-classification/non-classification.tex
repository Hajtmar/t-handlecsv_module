% DATA.CSV file:
%Surname;Firstname;Beh:Behaviour:B:0;Eng:English:S:1;His:History:S:2;Geo:Geography:S:3;M:Mathematics:S:4;Ph:Physics:S:5;Chem:Chemistry:S:6;Bi:Biology:S:7;Gym:Gymnastics:S:8
%Armstrong;Martin;1;2;2;1;N;2;1;1;N
%Jones;Bill;1;N;2;2;N;3;5;2;1
%Aladin;Adam;1;2;2;3;3;3;2;2;1
%Smith;John;1;2;3;2;3;2;2;1;1
%Johnson;Veronika;1;N;2;5;N;3;N;2;1
%Williams;Olga;1;1;2;3;3;1;2;2;1
%Brown;Martina;1;2;2;1;3;2;2;1;1
%Cruz;Barbora;1;2;2;1;3;5;2;1;1
%Martinez;Romana;1;1;1;1;3;1;2;2;N
%Taylor;Phillip;1;1;2;2;3;3;3;2;1
%Thomas;Alena;1;2;2;1;2;5;3;1;1
%Moore;Hanah;1;N;2;3;N;2;3;N;1
%Garcia;Ann;1;2;2;2;3;1;2;2;1
%Davis;Sarka;1;3;3;3;3;3;3;2;1
%Miller;Johana;1;1;1;1;3;1;2;2;1
%Kvapil;Barbora;1;1;1;1;5;1;1;1;1
%Mark;Bob;1;2;3;5;4;2;3;5;1
%Nimmrichter;Tomas;1;1;2;1;1;1;2;2;1
%Dunlop;Jane;1;N;2;1;N;2;3;1;N
%Eres;Hana;1;1;1;1;2;1;1;1;1
%Hummer;Paul;1;1;2;1;2;2;1;1;1


% Minimal example: The cycle goes through all the marks of students and lists students who have difficulty learning.
% In our school we use this script to generate letters to inform parents about the need for students with lack of evaluation to completing their marks on particular subjects.

%\usedirectory[/Users/Hajtmar/Onedrive/Dokumenty/1da/ConTeXt/ConTeXt-Tests/my_way/ScanCSV/t-handlecsv_module/]
%\usemodule[handlecsv]
\input ../incl_handlecsv_file.tex

\unexpanded\def\subjectabbreviation#1{\ctxlua{context(thirddata.handlecsv.ParseCSVLine('#1',':')[1])}}%


\def\subjectstesting#1{%
 \def\listofsubjects{}%
 \dostepwiserecurse{3}{11}{1}{\if\csvcell[\recurselevel,\linepointer]#1 \eaddto\listofsubjects{\subjectabbreviation{\csvcell[\recurselevel,0]}, }\fi}%
}

\unexpanded\def\tableaction#1{
\subjectstesting{#1}
\ifx\listofsubjects\empty\addtonumline{-1}\else
\expanded{\bTR\bTD\numline\eTD \bTD\cA\ \cB \eTD\bTD\listofsubjects\eTD}%
\fi
}


\starttext

\setheader
\opencsvfile{data.csv}

\subject{The list of pupils with a lack of evaluation (N):}

\bTABLE
\doloopfromto{1}{\numrows}{\tableaction{N}}
\eTABLE

\subject{List of students with insufficient marks (5) :}

\bTABLE
\doloopfromto{1}{\numrows}{\tableaction{5}}
\eTABLE

\stoptext
