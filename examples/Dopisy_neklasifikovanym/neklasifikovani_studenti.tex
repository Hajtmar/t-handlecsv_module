\mainlanguage[cz]            % Podle typografických pravidel 
\setuppapersize[A4][A4]      % Velikost papíru bude A5 a bude na stránce opět o téže velikosti. 
\setupcolors[state=start]    % Zapne používání barev.
\setuppagenumbering[state=stop]
\setuplayout[width=middle, topspace=1cm, header=1cm, footer=1cm]

\enablemode[prvnipololeti]
%\enablemode[druhepololeti]

\def\skolnirok{2015/16}




\doifmodeelse{prvnipololeti}
{\def\pololeti{první}\def\pololeticislici{1.}}
{\def\pololeti{druhé}\def\pololeticislici{2.}}


\usedirectory[c:/1da/ConTeXt/ConTeXt-Tests/my_way/ScanCSV/t-handlecsv_module/]
\usemodule[handlecsv]

\long\def\addto#1#2{\expandafter\def\expandafter#1\expandafter{#1#2}}
\long\def\eaddto#1#2{\edef#1{#1#2}} 

\unexpanded\def\zkratkapredmetu#1{\ctxlua{context(thirddata.handlecsv.ParseCSVLine('#1',':')[1])}}%

\def\neklpredmety{%
\def\NeklPredmety{}
\dostepwiserecurse{24}{58}{1}{\if\csvcell[\recurselevel,\linepointer]N \eaddto\NeklPredmety{\zkratkapredmetu{\csvcell[\recurselevel,0]}, }\fi}% 
\dostepwiserecurse{75}{85}{1}{\if\csvcell[\recurselevel,\linepointer]N \eaddto\NeklPredmety{\zkratkapredmetu{\csvcell[\recurselevel,0]}, }\fi}% 
}



\input lib-lua.tex

\unexpanded\def\lineaction{
\let\Trida\cA
\let\Prijmeni\cE
\let\Jmeno\cF
\neklpredmety
\ifx\NeklPredmety\empty\addtonumline{-1}\else
\numline -- \Jmeno\ \Prijmeni\  (\Trida):  \NeklPredmety\crlf 
\fi
} 


\unexpanded\def\tableaction{
\neklpredmety
\ifx\NeklPredmety\empty\addtonumline{-1}\else
\let\Trida\cA
\let\Prijmeni\cE
\let\Jmeno\cF
\expanded{\bTR\bTD\numline\eTD \bTD\Jmeno\ \Prijmeni \eTD\bTD (\Trida)\eTD\bTD\NeklPredmety \eTD}%
\fi
}


\starttext

\setheader

\doifmodeelse{prvnipololeti}
{\setfiletoscan{export-1pololeti.csv}}
{\setfiletoscan{export-2pololeti.csv}}
\opencsvfile

\subject{Neklasifikovaní žáci --  šk. r. \skolnirok, \pololeticislici\ pololetí} 


% \resetnumline
% \doloopfromto{1}{\numrows}{
% \lineaction
% }
% 
% 
% \filelineaction



\blank[3*big]


\bTABLE
\doloopfromto{1}{\numrows}{\tableaction}
\eTABLE



\stoptext
