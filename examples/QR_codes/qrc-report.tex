

\input ../incl_handlecsv_file.tex

\usepath[{/Users/hajtmar/OneDrive/Dokumenty/1da/GitHub-projekty/t-handlecsv_module,qrkody,qrc-shell,qrc-shell/qrkody}]
\setupexternalfigures[directory={qrkody,qrc-shell,qrc-shell/qrkody}] % POZOR! Nesmí být mezera na konci.



%\usemodule[/Users/hajtmar/OneDrive/Dokumenty/1da/ConTeXt/ConTeXt-Tests/my_way/ScanCSV/t-handlecsv_module/t-handlecsv] % poslední verze
%\usemodule[/Users/hajtmar/OneDrive/Dokumenty/1da/ConTeXt/ConTeXt-Tests/my_way/ScanCSV/t-handlecsv_module 2/t-handlecsv (kopie 2)]
%\usemodule[/Users/hajtmar/OneDrive/Dokumenty/1da/GitHub-projekty/t-handlecsv_module/t-handlecsv-optbyai] % poslední verze

%\usemodule[t-handlecsv] % poslední verze
%\usemodule[t-handlecsv-optbyai] % poslední verze



\setheader
% \setsep{;}

\newdimen\QRSize
\QRSize=2.5cm % Výchozí velikost QR kódu, lze změnit

%
% TATO MAKRA JSOU NYNÍ PLNĚ FUNKČNÍ! DALŠÍ ZPŮSOB GENEROVÁNÍ QR KÓDŮ JE NÁSLEDUJÍCÍ:
% v terminálu spustit zsh skript:
% qrfromcsv.sh ../test-cstex.csv --ccol=QRURL  (je na cestě v SUPPORT/BASH/)
% v CSV souboru jsou mj i data pro generování QR kódů. Výstupem jsou soubory test-cstex_qr<Id>.png, kde <Id> je záhlaví sloupce v CSV souboru
% přičemž data pro vytváření QR kódu jsou ve defaultně ve sloupci QRcontent, nicméně přepínačem --ccol=QQRUL je možné vybrat jakýkoliv sloupec,
% který bude sloužit jako obsah pro vygenerování QR kódu. Název obrázku s QR kódem tj. test-cstex_qr<Id>.png je odvozen od názvu CSV souboru s daty
% tj. v tomto případě test-cstex.csv.
% zobrazení obrázků s QR kódy by již neměl být problém, protože ke všem by měl mít ConTeXt cestu viz \usepath a \setupexternalfigures výše (lze dodat cesty..)
% stačí se na obrázek odvolat v handlecsv modulu takto     \externalfigure[test-cstex_qr\Id.png][width=\QRSize] nebo
% pomocí makra \viewqr{test-cstex_qr\Id.png}.
%




% Specializované makro pro použití s handlecsv (bez handlecsv nedává použití smysl)
\def\genqr#1{%
    \directlua{os.execute([[qrencode -o qrkody/qrcode\Id.png -s 10 "#1"]])}
    \externalfigure[qrcode\Id.png][width=\QRSize]%
}


% Jednoúčelové zcela obecné makro, které  vezme obsah z #1 udělá z něj QR kód
% a ten uloží do obrázku s názvem a cestou v #2
\def\GENQR#1#2{%
    \directlua{os.execute("qrencode -o #2 -s 10 #1")}%'
    \externalfigure[#2][width=\QRSize]%
}



% Funguje tak, že v případě existence parametru #2 načte QRkod obrázku v #2
% pokus je  #2={} pak vygeneruje a zobrazí QR kód.
\def\Genqr#1#2{%
  % Nejprve ověříme, zda je první parametr (#1) neprázdný
  \if:#1:
    % Pokud #1 je prázdný, rozhodujeme se podle #2
    \if:#2:
      % Oba parametry jsou prázdné → prázdný rámeček
      \framed[width=\QRSize, height=\QRSize, frame=on, align=middle]{}
    \else
      % Pokud #1 je prázdné, ale #2 obsahuje soubor, načteme obrázek
       \externalfigure[#2][width=\QRSize]
    \fi
  \else
    % #1 není prázdné → pokud #2 obsahuje soubor, načteme ho místo generování QR
    \if:#2:
      % #1 obsahuje hodnotu, ale #2 je prázdné → generujeme QR kód
        % \directlua{os.execute("qrencode -o qrkody/qrcode\Id.png -s 10 #1")}%
        % \externalfigure[qrkody/qrcode\Id.png][width=\QRSize]
        \directlua{os.execute("qrencode -o qrkody/qrcode_by_GenQr.png -s 10 #1")}%
        \externalfigure[qrkody/qrcode_by_GenQr.png][width=\QRSize]
    \else
      % #1 i #2 jsou neprázdné →  vytvoříme QR kód s obsahem a uložíme jej do souboru s daným názvem v #2
    %  \externalfigure[#2][width=\QRSize]
     \GENQR{#1}{#2}
    \fi
  \fi
}


% Pouze pro nalezení souboru s obrázkem na cestě + zobrazení QR kódu
\define[1]\viewqr{%
  \doiffileelse{#1}{%
    \externalfigure[#1][width=\QRSize]%
  }{%
    \framed[width=\QRSize, height=\QRSize, frame=on, align=middle]{}
  }%
}



% Zobrazení QR kódu:
% \viewqr{qrc_by_shell_qrfromcsv_sh1.png}\page
% \viewqr{qrc_by_shell_qrfromcsv_sh5.png}\page


%% Funkční kód:
% 1 \Genqr{}{ahoj123.png}\page         % načte QR kód z obrázku ahoj123.png
% 2 \Genqr{AAA}{qrkody/xyz.png}\page   % vytvoří QR kód z AAA a uloží jej do qrkody/xyz.png
% 3 \Genqr{}{}\page                    % Prázdný rámeček - není co řešit
% 4 \Genqr{BBB}{}\page                 % Vytvoří kód z BBB a uloží jej do defaultního QR kódu "qrcode_by_GenQr.png" a ten načte
%
%
%% Vytváření kódů, jejich ukládání a  načítání (zobrazování)
% 5 \GENQR{AAA}{qrkody/ahoj.png}\page
% 6 \GENQR{BBB}{qrkody/ahoj.png}\page
% 7 \GENQR{CCC}{ahoj123.png}\page
% 8 \GENQR{DDD}{qrkody/alibaba.png}\page
%


% Zadávané hodnoty:
\newcount\countofhor\countofhor=3
\newcount\countofvert\countofvert=6

\showframe[edge]


% Zbytek se už vypočítává
\definepapersize[BC][width=\dimexpr(\printpaperwidth/\countofhor),height=\dimexpr(\printpaperheight/\countofvert)]

\setuppapersize [BC][A4]

\setuppaper [topspace=0pt,backspace=0pt,bottomspace=0pt,cutspace=0pt, dx=0pt,dy=0pt,nx=\countofhor,ny=\countofvert]

\setuplayout [page] [topspace=0pt,backspace=0pt,bottomspace=0pt,cutspace=0pt, location=fit]
\setuplayout [topspace=0pt, backspace=0pt, height=fit, width=fit, location=middle, marking=on, leftmargin=0pt, rightmargin=0pt, header=0pt, footer=0pt]
\setuparranging [XY]


\setuppagenumbering [alternative=doublesided,location=, state=stop]




\startbuffer[test-cstex]
\vbox to \vsize{
  \vfil
  \hbox{}
  \hbox to \hsize{\hfil\ssa\bf Test CsTeX - \Id \hfil}
  \hbox to \hsize{\hfil {\genqr{\QRURL}}\hfil}
  \blank[-3mm]
  \hbox to \hsize{\hfil\ssx\bf\Prijmeni\ \Jmeno\ (\Trida)\hfil}
  \hbox to \hsize{\hfil\ssxx Uchovejte tento privátní QR kód v bezpečí!\hfil}
  \vfil
}
\nextrow
\stopbuffer






\starttext

\enablemode[test-cstex]


\startmode[test-cstex]
\catcode`\%=12
    \opencsvfile{test-cstex.csv}
    \dorecurse{\numrows}{\getbuffer[test-cstex]}
\stopmode



\stoptext




