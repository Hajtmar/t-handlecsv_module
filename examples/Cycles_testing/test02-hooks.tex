\usemodule[handlecsv]
%\usemodule[../t-scancsv_ConTeXt_module/last_versions/scancsv-2015-06-02]
%\usemodule[scancsv-2015-06-02]



\def\bfilehook{BF}%
\def\efilehook{EF}%
\def\blinehook{BL}%
\def\elinehook{EL}%
\def\bch{BC}
\def\ech{EC}

% Transaction_date,Product,Price,Payment_Type,Name,City,State,Country,Account_Created,Last_Login,Latitude,Longitude
\unexpanded\def\clineaction{%
% \let\Product=\cB%
% \let\Price=\cC% 
% \let\City=\cF% 
% \let\State=\cG% 
% \let\Country=\cH% 
{\numline}\ -  \sernumline - \linepointer -- (\lineno): \Product\ -- \Price, \City, \State, \Country% \Latitude - \Longitude \crlf
}%
	
\unexpanded\def\lineaction{\numline-\sernumline-\linepointer--(\lineno):\hProduct\--\hPrice-\hCity-\hState-\hCountry}%	
\unexpanded\def\action{ACTION: \numline\ -  \sernumline - \linepointer -- (\lineno): \hProduct\ -- \hPrice, \hCity, \hState, \hCountry}%

\usehooks % it is necessary turn on using hooks while process CSV file
\setheader
\setsep{;}
% \unsetsep
\opencsvfile{SalesJan2009.csv} % 999 lines


\starttext


% List of all CSV table:
%
% \bTABLE
%     \dorecurse{\numrows}{\bTR
%         \dorecurse{\numcols}{\bTD \csvcell[\currentTABLEcolumn,\currentTABLErow] \eTD}
%     \eTR}
% \eTABLE
%    




% doloop cycle:
% 
% \doloop{%
% \ifnum\cC=1200\lineaction\ifEOF\exitloop\else\nextrow\fi
% \else\ifEOF\exitloop\else\nextrow\fi
% \fi%
% }%

%
%
%


% List of column names (ie all columns in 'zero row'):\crlf
% \dostepwiserecurse{1}{\numcols}{1}{\csvcell[\recurselevel,0],\  }
% 
% List of first row of table:\crlf
% \dostepwiserecurse{1}{\numcols}{1}{\csvcell[\recurselevel,1],\  }



% MODULE PREDEFINED LOOPS
%\filelineaction
%\doloopfromto{1}{50}{\lineaction}  
%\doloopforall  % implicit do \lineaction for all lines of open CSV table
%\doloopforall{\action}  % do \action macro for all lines of open CSV table



% \doloopaction{\action} % use \action macro for all lines of open CSV file
% \doloopaction{\action}{4} % use \action macro for first 4 lines
% \doloopaction{\action}{2}{5} % use \action macro for lines from 2 to 5


\filelineaction

% compare numbers
%\doloopifnum{\cC}{<}{1200}{\lineaction} % less
%\doloopifnum{\cC}{>}{1200}{\lineaction} % greather
%\doloopifnum{\cC}{==}{1200}{\lineaction} % equal
%\doloopifnum{\cC}{~=}{1200}{\lineaction} % not equal
%\doloopifnum{\cC}{>=}{1200}{\lineaction} % greather or equal
%\doloopifnum{\cC}{<=}{1200}{\lineaction} % less or equal

% compare strings
% \doloopif{\Product}{==}{Product2}{\lineaction} % compare string
% \doloopif{\Product}{eq}{Product2}{\lineaction} % == is synonym for 'eq' operator if arguments are not numbers
% \doloopif{\Product}{~=}{Product2}{\lineaction}
% \doloopif{\Product}{neq}{Product2}{\lineaction} % ~= is synonym for 'neq' operator  if arguments are not numbers
% \doloopif{uct3}{in}{\Product}{\lineaction} % testing if substring is contained inside string 
% \doloopif{Product3}{until}{\Product}{\lineaction} % List all, until the test is not met - then just quit. Repeat until satisfied. If it is not never met, will list all record
% \doloopif{\Product}{while}{Product1}{\lineaction} % List all, while the test is  met. When first which not met then just quit. If first is not met, will list no record
%\doloopuntil{Product3}{\Product}{\lineaction}
%\doloopwhile{\Product}{Product1}{\lineaction}
 


\stoptext





% 4. \doloopif{value1}{[compare_operator]}{value2}{macro_for_doing} % [compareoperators] <, >, ==(eq), ~=(neq), >=, <=, in, until, while 
% actions for rows of open CSV file which are responded of condition  
\def\doloopif#1#2#3#4{%
    \opencsvfile%
    \readline
    \bfilehook%
    \def\paroperator{#2}
    %  operator '==' is for strings comparing converted to 'eq' operator 
    \startluacode
			 if '#2'=="==" and not(type(tonumber('#1'))=='number' and type(tonumber('#3'))=='number') 
	     then context('\\def\\paroperator{eq}')  
	     end
     \stopluacode
    %  operator '~=' is for strings comparing converted to 'neq' operator
    \startluacode
		 if '#2'=="~=" and not(type(tonumber('#1'))=='number' and type(tonumber('#3'))=='number') 
     then context('\\def\\paroperator{neq}')  
     end
		 \stopluacode
    % and now process actual operator 
    \processaction[\paroperator][%
     <=>{% {number1}{<}{number2} ... Less 
     \doloop{%
    	\edef\tempiv{#4}%
      \startluacode
				if #1<#3 then context('\\blinehook \\tempiv\\elinehook') end
			\stopluacode
			%
      \readline\nextline%
   	  \ifEOF\exitloop\fi%
     }%
     },% end < ... Less
     >=>{% {number1}{>}{number2} ... Greater 
     \doloop{%
    	\edef\tempiv{#4}%
    	\startluacode
				if #1>#3 then context('\\blinehook \\tempiv\\elinehook') end
			\stopluacode
			\readline\nextline\ifEOF\exitloop\fi%
     }%
     },% end > ... Greater
     ===>{% {number1}{==}{number2} ... Equal
     \doloop{%
       \edef\tempiv{#4}%
    	 \startluacode
			  if #1==#3 then context('\\blinehook \\tempiv\\elinehook') end
			 \stopluacode
			 \readline\nextline\ifEOF\exitloop\fi%
     }%
     },% end == ... Equal
     ~==>{% {number1}{~=}{number2} ... Not Equal
     \doloop{%
       \edef\tempiv{#4}%
    	  \startluacode
			   if #1~=#3 then context('\\blinehook \\tempiv\\elinehook') end
				\stopluacode
				\readline\nextline\ifEOF\exitloop\fi%
     }%
     },% end ~= ... Not Equal
     >==>{% {number1}{>=}{number2} ... GreaterOrEqual 
     \doloop{%
       \edef\tempiv{#4}%
    	 \startluacode
			  if #1>=#3 then context('\\blinehook \\tempiv\\elinehook') end
				\stopluacode
				\readline\nextline\ifEOF\exitloop\fi%
     }
     },% end >=  ... GreaterOrEqual
     <==>{% {number1}{<=}{number2} ... LessOrEqual 
     \doloop{%
       \edef\tempiv{#4}%
       \startluacode
			   if #1<=#3 then context('\\blinehook \\tempiv\\elinehook') end
			 \stopluacode
			 \readline\nextline\ifEOF\exitloop\fi%
     }%
     },% end <= ... LessOrEqual 
     eq=>{%  command {string1}{==}{string2} is converted to command command {string1}{eq}{string2} ... string1 is equal string2 
     \doloop{%
 		 \edef\tempi{#1}%
 		 \edef\tempii{#3}%
 		 \edef\tempiii{#4}%
       \ifx\tempi\tempii\blinehook \tempiii\elinehook\fi\readline\nextline\ifEOF\exitloop\fi%
     }%
     },%  end eq
     neq=>{%  command {string1}{~=}{string2} is converted to command command {string1}{neq}{string2} ... string1 is not equal string2  
     \doloop{%
  		\edef\tempi{#1}%
  		\edef\tempii{#3}%
  		\edef\tempiii{#4}%
      \ifx\tempi\tempii\else\blinehook \tempiii\elinehook\fi\readline\nextline\ifEOF\exitloop\fi%
     }%
     },% end neq
     in=>{% {substring}{in}{string} ... substring is contained inside string 
     \doloop{%
 		 \edef\tempi{#1}%
 		 \edef\tempii{#3}%
 		 \edef\tempiii{#4}%
       \doifincsnameelse{\tempi}{\tempii}{\blinehook \tempiii\elinehook}
			 \readline\nextline\ifEOF\exitloop\fi%
     }
     }, % end in
     until=>{% {substring}{until}{string} ... % List all, until the test is not met - then just quit. Repeat until satisfied. If it is not never met, will list all record
      \doloop{
        \edef\tempi{#1}
        \edef\tempii{#3}
        \edef\tempiii{#4}
        % \ifx\tempi\tempii\exitloop\else\ifEOF\exitloop\else \blinehook \tempiii \elinehook\readline\nextline\fi\fi% recent version
        \ifx\tempi\tempii\exitloop\else\ifEOF\exitloop\else\blinehook \tempiii\elinehook\readline\nextline\fi\fi% new version
     }%
     }, % end until % the comma , is very important here!!!
     while=>{% {string}{while}{string} ... % List all, while the test is  met. When first which not met then just quit. If first is not met, will list no record
     \doloop{%
	   \edef\tempi{#1}
	   \edef\tempii{#3}
      \edef\tempiii{#4}
	   \ifx\tempi\tempii\blinehook \tempiii\elinehook\readline\nextline\else\exitloop\fi%
   	\ifEOF\exitloop\fi%
     }%
     }, % end while % the comma , is very important here!!!
     test=>{% {string}{while}{string} ... % List all, while the test is  met. When first which not met then just quit. If first is not met, will list no record
     \doloop{%
	   \edef\tempi{#1}
	   \edef\tempii{#3}
      \edef\tempiii{#4}
	   \ifx\tempi\tempii\blinehook \tempiii\elinehook\readline\nextline\else\exitloop\fi%
   	\ifEOF\exitloop\fi%
     }%
     }, % end test % the comma , is very important here!!!
    ]% end of \processaction
  \efilehook%
} % end of \doloopif
