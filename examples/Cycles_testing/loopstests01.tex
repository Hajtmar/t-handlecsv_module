\input ../incl_handlecsv_file.tex % \usemodule[t-handlecsv_opt.lua]


\unexpanded\def\action{ACTION: \numline\  (\lineno): %
	\ifissetheader\Lastname\ \Firstname\crlf
	\else\cA\ \cB \crlf
	\fi
% OR
% 	\ifnotsetheader \Lastname, \Firstname, \Flag, \MonThu, \ThuPM, \Fri, \Sat, \Sun, \Email. \crlf
% 	\else \cA, \cB, \cC, \cD, \cE, \cF, \cG, \cH, \cI. \crlf
% 	\fi
}

\unexpanded\def\lineaction{LINEACTION: \action}


\def\pokec#1{\type{#1}  \blank[big]}

\def\mezera{\blank[4*big]}

\setheader
\setsep{,}
% \unsetsep
\opencsvfile{ctm.csv}


\starttext




\section{\type{\doloopfromto} cycles}

% \doloopfromto{from}{to}{action}  --  do action "action" from line "from" to line "to" of open CSV file

\pokec{\doloopfromto{3}{8}{\action}}

\resetnumline
\doloopfromto{3}{8}{\action}

\doloopfromto{8}{8}{\action}

\doloopfromto{3}{8}{\action}

\mezera

\pokec{\doloopfromto{5}{8}{\lineaction}}

\resetnumline
\doloopfromto{5}{8}{\lineaction}
\doloopfromto{5}{8}{\lineaction}

\mezera


\pokec{\doloopfromto{1}{\numrows}{\lineaction}}

\resetnumline
\doloopfromto{1}{\numrows}{\lineaction}

ss
\doloopfromto{1}{\numrows}{\lineaction}
ss

\mezera





\section{\type{\doloopforall} cycles}

% \doloopforall  % implicit do \lineaction for all lines of open CSV table
% \doloopforall{\action}  % do \action macro for all lines of open CSV table



\subsection{do \type{\lineaction} macro for all lines of opening csv table}
\pokec{\doloopforall}

\resetnumline
\doloopforall


\subsection{do specific \type{\action} macro for all lines of opening csv table}
\pokec{\doloopforall{\action}}


\resetnumline
\doloopforall{\action}




\section{\type{\doloopaction} cycles}

% \doloopaction % implicit use \lineaction macro
% \doloopaction{\action} % use \action macro for all lines of open CSV file
% \doloopaction{\action}{4} % use \action macro for first 4 lines
% \doloopaction{\action}{2}{5} % use \action macro for lines from 2 to 5

\subsection{do \type{\lineaction} macro for all lines of opening csv table}
\pokec{\doloopaction}

\resetnumline
\doloopaction % implicit use \lineaction macro


\subsection{do specific \type{\action} macro for all lines of opening csv table}
\pokec{\doloopaction{\action}}

\resetnumline
\doloopaction{\action} % use \action macro for all lines of open CSV file


\subsection{do specific \type{\action} macro for the first few lines of opening csv table}
\pokec{\doloopaction{\action}{4} }

\resetnumline
\doloopaction{\action}{4} % use \action macro for first 4 lines


\section{\type{\doloopaction} cycles}

\subsection{do specific \type{\action} macro for specific continous range of lines (from -- to)}
\pokec{\doloopaction{\action}{2}{5}}

\resetnumline
\doloopaction{\action}{2}{5} % use \action macro for lines from 2 to 5



\section{\type{\filelineaction} cycles}

% \filelineaction  % implicit do \lineaction for all lines of current open CSV table
% \filelineaction{filename.csv}   % do \lineaction macro for all lines of specific CSV table (filename.csv)

\subsection{do \type{\lineaction} macro for all lines of opening table}
\pokec{\filelineaction}

\resetnumline
\filelineaction


\subsection{do \type{\lineaction} macro for all lines of specific \type{filename.csv} table}
\pokec{\filelineaction{ctm2.csv}}

\resetnumline
\filelineaction{ctm2.csv}


\section{\type{\doloopif} cycles}

% \doloopif{value1}{[compare_operator]}{value2}{macro_for_doing} % [compareoperators] <, >, ==(eq), ~=(neq), >=, <=, in, until, while
% and specific variations of previous macro \doloopif
% \letvalue{doloopifnum}=\doloopif %\doloopifnum{value1}{[compare_operator]}{value2}{macro_for_doing} % [compareoperators] ==, ~=, >, <, >=, <=
% \def\doloopuntil#1#2#3{\doloopif{#1}{until}{#2}{#3}}% \doloopuntil{\Trida}{3.A}{\tableaction}  % List all, until the test is not met - then just quit. Repeat until satisfied. If it is not never met, will list all records.
% \def\doloopwhile#1#2#3{\doloopif{#1}{while}{#2}{#3}}% \doloopwhile{\Trida}{3.A}{\tableaction}  % List when the test is met, other just quit.

\opencsvfile{ctm2.csv}

\doloopif{\Fri}{==}{0}{\action}



\stoptext


% \def\pd#1{\directlua{context([==[\detokenize{#1}]==])}}
%
% \pd{\numrows}
