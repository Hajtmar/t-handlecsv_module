\input ../incl_handlecsv_file.tex % \usemodule[t-handlecsv_opt.lua]

\starttext


%\ctxlua{
%xlsname='A'
%content='content when parameter is missing'
%context([[\definecomplexorsimple\]]..xlsname)
%context([[\def\simple]]..xlsname..[[{]]..content..[[}]])
%context([[\def\complex]]..xlsname..'[#1]'..[[{Neco: #1}]])
%}

\def\simpleA {content when parameter is missing}
\def\complexA[#1]{parameter #1}



%\type{\A}: \A

%\type{\A[5]}: \A[5]



\startluacode
interfaces.definecommand ('dolastname', {
    arguments = { { "option", "string" }  },
    macro = function (opt_1)
       if #opt_1>0 then context(opt_1) else context('nothing') end
    end
})
interfaces.definecommand ('lastname', {
    macro = function ()
       context.dosingleempty()
       context.dolastname()
    end
})
\stopluacode

\lastname

\lastname[5]




\startluacode
xlsname='A'
content='content when parameter is missing'
interfaces.definecommand (xlsname, {
    arguments = { { "option", "string" }  },
    macro = function (opt_1)
       if #opt_1>0 then context(opt_1) else context(content) end
    end
})
interfaces.definecommand (xlscommand, {
    macro = function ()
       context.dosingleempty()
       context.dolastname()
    end
})
\stopluacode



\stoptext


%
% Within the complex problems I need to solve the following case. I want using
% lua code to define a macro with one optional parameter in brackets. I tried it
% using commands context.setgvalue, or find something on the wiki
% (http://wiki.contextgarden.net/System_Macros/Definitions_and_Assignments), but
% each time I broke a tooth on it.
%
% Can you, please, someone advise  how something can be it done?
% Suffice some reference to solve a similar problem.
%
% Thanks Jaroslav Hajtmar
%
% Here is my minimal example:

%\starttext

% OK
% \def\lastname{\dosingleempty\dolastname}
% \def\dolastname[#1]{\iffirstargument #1\else nothing\fi}
%
% \lastname
%
% \lastname[Smith]


% Not working
% \ctxlua{
% context([[\def\firstname{\dosingleempty\dofirstname}]])
% context([[\def\dofirstname[#1]{\iffirstargument #1\else {nothing}\fi}]])
% }
%
% \firstname
%
% \firstname['Peter']


% \complexorsimple\priklad
%
% \def\simplepriklad{XXX}
%
% \def\complexpriklad[ZZZZZ]
%
%
% \priklad


% \def\priklad{}
% \complexprikladdo[setup]
% \complexprikladdo[]


% \definecomplexorsimple\command
% \def\complexcommand[#1]{parameter #1}
% \def\simplecommand {without parameter}
%
% \command
%
% \command[123]


%\starttext
%
% \ctxlua{
% xlsname='A'
% content='content when parameter is missing'
% context([[\definecomplexorsimple\]]..xlsname)
% %context([[\def\simple]]..xlsname..[[{]]..content..[[}]])
% %context([[\def\complex]]..xlsname..'[#1]'..[[{Neco: #1}]])
% }
%
% \def\simpleA {content when parameter is missing}
% \def\complexA[#1]{parameter #1}
%
%
%
% \type{\A}: \A
%
% \type{\A[5]}: \A[5]


%\stoptext



% OK
% \definecomplexorsimple\priklad
% \def\complexpriklad[#1]{Neco: #1}
% \def\simplepriklad{nic}
%
%
%
% Bez parametru: \priklad
%
% S parametrem 123: \priklad[123]
%




% \ctxlua{
% xlsname='A'
% content='Jmeno a prijmeni'
% context([[\definecomplexorsimple\]]..xlsname)
% context([[\def\complex]]..xlsname..'[#1]'..[[{Neco: #1}]])
% %context([[\def\simple]]..xlsname..'{'..content..'}')
% }


% \def\complexA[#1]{Neco: #1}
% \def\simpleA{nic}
%
% AAAAAAAAAAAAAAAAAA
%
% \A
%
%
% \A[5]
%
% AAAAAAAAAAAAAAAAAAAAAAAAAAAA


%\stoptext




% \definestartstopcommand\somecommand\e!specifier{[##1][##2]}{##3}%
%   {do #1 something #2 with #3 arg}
% which becomes:
% \def\somecommand[#1][#2]\startspecifier#3\stopspecifier%
%   {do #1 something #2 with #3 arg}











% \def\dolastname[1]\{\iffirstargument 1\else nic\fi\}
% context([[*def*]]..xlsname..[[\{*dosingleempty*do]]..xlsname..[[\}]])
% context([[def*do]]..xlsname..[[\{[!1]*iffirstargument*csvcell[']]..xlsname..[[',!1]*else ]]..content..[[*fi\}]])
% context([[\def\]]..xlsname..[[\{\dosingleempty\do]]..xlsname..[[\}]])
% context([[\def\do]]..xlsname..[[\{[!1]\iffirstargument\csvcell[']]..xlsname..[[',!1]\else ]]..content..[[\fi\}]])

