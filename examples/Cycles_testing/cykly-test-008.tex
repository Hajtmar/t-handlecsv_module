
\def\typeandgetbuffers#1#2{
Prováděné makro: {\tt\backslash#2} 
\typebuffer[#2]
\typebuffer[#1]
\blank[big]
\startlines
\getbuffer[#1]
\stoplines
\blank[big]
\hairline
}


\usemodule[handlecsv]
\usemodule[handlecsv-tools]


\def\bch{\bgroup\red\bf }
\def\ech{\egroup}


\startbuffer[tab]
\unexpanded\def\tab{
\expanded{\bTR
\bTD{\it \colname[{\indexcolname['cA']}]}:  \hcA\ - \Jmeno\eTD\bTD\hJmeno\eTD
\bTD\hPrijmeni\eTD
\eTR}
}%
\stopbuffer

\startbuffer[lineaction]
\unexpanded\def\lineaction{
{\it\colname[{\indexcolname['cA']}]}:  \hcA\ - \hJmeno\ \hPrijmeni\par
}
\stopbuffer

\startbuffer[akce]
\unexpanded\def\akce{
{\it\colname[{\indexcolname['cA']}]}: \hcA\ - \hJmeno\ \hPrijmeni\ 
(NumHod: \NumHod)\par
}
\stopbuffer

\getbuffer[tab]
\getbuffer[lineaction]
\getbuffer[akce]


\setupcolors[state=start]


\starttext

\title{Netabulkové použití} 


\setheader
\setsep{,}
\usehooks % it is necessary turn on using hooks while process CSV file 
\opencsvfile{lisajdb1a.csv}

%\printline
%\printall
% \csvreport 




%\enablemode[XXX]

%\startmode[XXX]



\startbuffer[A)]
\doloopif{\cA}{<}{5}{\akce} %OK
\stopbuffer
\typeandgetbuffers{A)}{akce}




\startitemize[n]

\sym{A)}
\startbuffer[A)]
\doloopif{\Id}{<}{5}{\akce} %OK
\stopbuffer
\typeandgetbuffers{A)}{akce}



\sym{B)}
\startbuffer[B)]
\doloopifnum{\Id}{>}{4}{\akce} %OK
\stopbuffer
\typeandgetbuffers{B)}{akce}


\sym{C)}
\startbuffer[C)]
\doloopifnum{\Id}{==}{\NumHod}{\akce} %OK
\stopbuffer
\typeandgetbuffers{C)}{akce}


\sym{D)} 
\startbuffer[D)]
\doloopifnum{\Id}{<=}{\NumHod}{\akce}
\stopbuffer
\typeandgetbuffers{D)}{akce}

\sym{E)} 
\startbuffer[E)]
\doloopifnum{\Id}{>=}{\NumHod}{\akce}
\stopbuffer
\typeandgetbuffers{E)}{akce}


\sym{F)} 
\startbuffer[F)]
\bTABLE
\bTR\bTD ID\eTD\bTD Jméno\eTD\bTD Příjmení\eTD\eTR
\doloopifnum{\Id}{>=}{5}{\tab}
\eTABLE
\stopbuffer
\typeandgetbuffers{F)}{tab}


\sym{01)}
\startbuffer[01)]
\doloopaction % implicitně používá \lineaction %OK
\stopbuffer
\typeandgetbuffers{01)}{lineaction}

	
\sym{02)}
\startbuffer[02)]
\doloopaction{\akce} %OK
\stopbuffer
\typeandgetbuffers{02)}{akce}

\sym{03)}
\startbuffer[03)]
\doloopaction{\akce}{4} %OK
\stopbuffer
\typeandgetbuffers{03)}{akce}
	
\sym{04)}
\startbuffer[04)]
\doloopaction{\akce}{2}{5} %OK
\stopbuffer
\typeandgetbuffers{04)}{akce}
	
\sym{05)}
\startbuffer[05)]
\doloopforall %OK
\stopbuffer
\typeandgetbuffers{05)}{lineaction}

	
\sym{06)}
\startbuffer[06)]
\doloopforall{\akce} %OK
\stopbuffer
\typeandgetbuffers{06)}{akce}

	
\sym{07)}
\startbuffer[07)]
\doloopif{Jan}{==}{\Jmeno}{\akce} % OK
\stopbuffer
\typeandgetbuffers{07)}{akce}
  
\sym{08)}
\startbuffer[08)]
\doloopif{Jan}{~=}{\Jmeno}{\akce} % OK
\stopbuffer
\typeandgetbuffers{08)}{akce}
	

\sym{08a)}
\startbuffer[08a)]
\doloopif{Ja}{in}{\Jmeno}{\akce} % OK
\stopbuffer
\typeandgetbuffers{08a)}{akce}


\sym{08b)}
\startbuffer[08b)]
\doloopif{Ja}{~in}{\Jmeno}{\akce} % OK
\stopbuffer
\typeandgetbuffers{08b)}{akce}


\sym{09)}
\startbuffer[09)]
\doloopuntil{\Jmeno}{Jan}{\akce} %OK
\stopbuffer
\typeandgetbuffers{09)}{akce}


% \sym{09a)}
% \startbuffer[09a)]
% \doloopwhile{\Jmeno}{Jan}{\akce} %OK
% \stopbuffer
% \typeandgetbuffers{09a)}{akce}


\sym{10)}
\startbuffer[10)]
\resetlinepointer
\dorecurse{5}{\readline\akce\nextrow} % OK
\stopbuffer
\typeandgetbuffers{10)}{akce}


\sym{10a)}
\startbuffer[10a)]
\resetlinepointer
\dorecurse{5}{\readline
{\it\colname[{\indexcolname['cA']}]}: \hcA\ - \hJmeno\ \hPrijmeni\  (NumHod: \NumHod)\par
\nextrow} % OK
\stopbuffer
\typeandgetbuffers{10a)}{vlastni}


	
\sym{11)}
\startbuffer[11)]
\resetlinepointer
\resetnumline
\doloop{
\ifnum\numline>7\exitloop
\else\readline\numline -- \lineaction\nextline
\fi} %OK
\stopbuffer
\typeandgetbuffers{11)}{lineaction}


\sym{11a)}
\startbuffer[11a)]
% Condition A AND B
\resetlinepointer
\resetnumline
\dorecurse{\numrows}{%
\readline{\recurselevel}%
\ifnum\Id>2% A condition
	\ifnum\Id<8\lineaction% B condition
	\fi%
\fi%
}%
\stopbuffer
\typeandgetbuffers{11a)}{lineaction}


\sym{11b)}
\startbuffer[11b)]
% Condition A OR B
\resetlinepointer
\resetnumline
\def\AorB{\lineaction}
\dorecurse{\numrows}{%
\readline{\recurselevel}%
\ifnum\Id=1\AorB% A condition
	\else\ifnum\Id>3\AorB\fi% B condition
\fi%
}
\stopbuffer
\typeandgetbuffers{11b)}{lineaction}


\sym{11c)}
\startbuffer[11c)]
% Condition A OR B
\resetlinepointer
\resetnumline
\def\AorB{\lineaction}
\def\condA{Jan}
\def\condB{Jaroslav}
\dorecurse{\numrows}{%
\readline{\recurselevel}%
\ifx\Jmeno\condA\AorB% A condition
	\else\ifx\Jmeno\condB\AorB\fi% B condition
\fi%
}
\stopbuffer
\typeandgetbuffers{11c)}{lineaction}



	
\sym{12)}
\startbuffer[12)]
\resetlinepointer
\doloop{\ifEOF\exitloop\else\readline\lineaction\nextline\fi} %OK
\stopbuffer
\typeandgetbuffers{12)}{lineaction}
	
\sym{13)}
\startbuffer[13)]
\resetlinepointer
\doloop{\readline
\if\Id3\exitloop
\else
\lineaction\nextline
\fi 
} %OK
\stopbuffer
\typeandgetbuffers{13)}{lineaction}

\sym{14)}
\startbuffer[14)]
\filelineaction %OK
\stopbuffer
\typeandgetbuffers{14)}{lineaction}

	
\sym{15)}
\startbuffer[15)]
\filelineaction{lisajdb2a.csv} %OK
\stopbuffer
\typeandgetbuffers{15)}{lineaction}

	
\sym{16)}
\startbuffer[16)]
\filelineaction{lisajdb1a.csv}{3} %OK
\stopbuffer
\typeandgetbuffers{16)}{lineaction}
	

\sym{17)}
\startbuffer[17)]
\filelineaction{lisajdb2a.csv}{3}{6} %OK
\stopbuffer
\typeandgetbuffers{17)}{lineaction}

	
% 	A16\sym{17a}\par
% \opencsvfile
%  \for \px=1 \to 5 \step 1 \do {\akce\nextrow} % OK
  

\sym{18)}
\startbuffer[18)]
\resetlinepointer
\resetnumline
\dostepwiserecurse{1}{6}{1}{\readline\akce\nextrow}
\stopbuffer
\typeandgetbuffers{18)}{akce}
  
  
\sym{19}
\startbuffer[19)]
\resetnumline
\doloopfromto{3}{7}{\akce}
\stopbuffer
\typeandgetbuffers{19)}{akce}

 
\stopitemize

\page



\title{Tabulkové použití} 

% Nyní ukázky použití v tabulkovém režimu. Toto použití vyžaduje expanzi uvnitř prostředí \bTABLE ... \eTABLE
% nebo použití jiných maker ... 


\startbuffer[lineaction]
\unexpanded\def\lineaction{%
	\expanded{%
		\bTR%
		\bTD\hcA\eTD%
		\bTD\hJmeno\eTD%
		\bTD\hPrijmeni\eTD%
		\eTR%
		}
}
\stopbuffer

\startbuffer[tab]
\unexpanded\def\tab{%
	\expanded{%
	\bTR%
	\bTD\hcA - \hJmeno\eTD%
	\bTD\Jmeno\eTD%
	\bTD\hPrijmeni\eTD%
	\eTR%
	}%
}%
\stopbuffer

\startbuffer[othertab]
\unexpanded\def\othertab{\expanded{\bTR\bTD\hcA - \Jmeno\eTD\bTD\hJmeno\eTD\bTD\Prijmeni\eTD\eTR}}
\stopbuffer

\startbuffer[tableaction]
\unexpanded\def\tableaction{\expanded{\bTR\bTD\hcA\eTD\bTD\hJmeno\eTD\bTD\hPrijmeni\eTD\eTR}}
\stopbuffer


\getbuffer[lineaction]
\getbuffer[tab]
\getbuffer[othertab]
\getbuffer[tableaction]


\setheader
\opencsvfile{lisajdb1a.csv}



\startitemize[n]


\sym{T1)}
\startbuffer[T1)]
\bTABLE[split=yes]
	\bTR\bTD ID\eTD\bTD Jméno\eTD\bTD Příjmení\eTD\eTR
	\doloopaction %implicit \lineaction using 
	\doloopaction{\tab}% use \tab macro in cycle
	\doloopaction{\tab}{4}%
	\doloopaction{\tab}{2}{5}%  
	\doloopaction{\othertab}%
	\doloopforall{\tab}%
\eTABLE
\stopbuffer
\typeandgetbuffers{T1)}{lineaction}



\sym{T2)}
\startbuffer[T2)]
\bTABLE%
	\bTR\bTD ID\eTD\bTD Jméno\eTD\bTD Příjmení\eTD\eTR%
	\doloopfromto{3}{7}{\lineaction}%
\eTABLE%
\stopbuffer
\typeandgetbuffers{T2)}{lineaction}



\sym{T3)}
\startbuffer[T3)]
\bTABLE%
	\bTR\bTD ID\eTD\bTD Jméno\eTD\bTD Příjmení\eTD\eTR%
	\doloopforall{\tab}%
	\doloopif{Jan}{==}{\Jmeno}{\tab}%
\eTABLE%
\stopbuffer
\typeandgetbuffers{T3)}{tab}



\sym{T4)}
\startbuffer[T4)]
\bTABLE%
	\bTR\bTD ID\eTD\bTD Jméno\eTD\bTD Příjmení\eTD\eTR
	\doloopif{Jan}{~=}{\Jmeno}{\tab}
\eTABLE
\stopbuffer
\typeandgetbuffers{T4)}{tab}


\sym{T5)}
\startbuffer[T5)]
\bTABLE%
	\bTR\bTD ID\eTD\bTD Jméno\eTD\bTD Příjmení\eTD\eTR%
	\doloopif{Jan}{==}{\Jmeno}{\tab}%
	\doloopuntil{\Jmeno}{Ondřej}{\tab}%
\eTABLE
\stopbuffer
\typeandgetbuffers{T5)}{tab}



\sym{T6)}
\startbuffer[T6)]
\bTABLE%
	\bTR\bTD ID\eTD\bTD Jméno\eTD\bTD Příjmení\eTD\eTR%
	\doloopif{Jan}{~=}{\Jmeno}{\tab}%OK	
	\doloopif{Markéta}{==}{\Jmeno}{\tab}%
\eTABLE
\stopbuffer
\typeandgetbuffers{T6)}{tab}



\sym{T7)}
\startbuffer[T7)]
\bTABLE
	\bTR\bTD ID\eTD\bTD Jméno\eTD\bTD Příjmení\eTD\eTR
	\doloopuntil{\Jmeno}{Jan}{\tab}%
	\dorecurse{5}{\readline{\recurselevel}\tab}%
\eTABLE
\stopbuffer
\typeandgetbuffers{T7)}{tab}
	
	
	
	
\sym{T8)}
\startbuffer[T8)]
\bTABLE
	\bTR\bTD ID\eTD\bTD Jméno\eTD\bTD Příjmení\eTD\eTR
	\dorecurse{5}{\readline{\recurselevel}\othertab} %OK	
\eTABLE
\stopbuffer
\typeandgetbuffers{T8)}{othertab}
 


\sym{T9)}
\startbuffer[T9)]
\resetlinepointer
\resetnumline
\bTABLE%
	\bTR\bTD ID\eTD\bTD Jméno\eTD\bTD Příjmení\eTD\eTR%
	\doloop{\readline\ifnum\numline>7\exitloop\else\lineaction\nextline\fi}%
\eTABLE
\stopbuffer
\typeandgetbuffers{T9)}{lineaction}





\sym{T10)}
\startbuffer[T10)]
\bTABLE
	\bTR\bTD ID\eTD\bTD Jméno\eTD\bTD Příjmení\eTD\eTR
	\resetlinepointer
	\resetnumline
	\doloop{\readline\ifnum\numline>7\exitloop\else\othertab\nextline\fi}%
	\resetlinepointer
	\resetnumline	
	\doloop{\readline\ifnum\numline<8\tab\nextline\else\exitloop\fi}%
\eTABLE
\stopbuffer
\typeandgetbuffers{T10)}{lineaction}

 


\sym{T11)}
\startbuffer[T11)]
\bTABLE%
	\bTR\bTD ID\eTD\bTD Jméno\eTD\bTD Příjmení\eTD\eTR%
	\dorecurse{\numrows}{%
		\readline{\recurselevel}%
		\ifnum\Id>7\othertab\fi
	}%	
\eTABLE%
\stopbuffer
\typeandgetbuffers{T11)}{lineaction}



\sym{T12)}
\startbuffer[T12)]
\bTABLE
	\bTR\bTD ID\eTD\bTD Jméno\eTD\bTD Příjmení\eTD\eTR
	\doloop{\ifEOF\exitloop\else\lineaction\nextrow\fi} %OK
\eTABLE
\stopbuffer
\typeandgetbuffers{T12)}{lineaction}
 
\sym{T13)}
\startbuffer[T13)]
\bTABLE
 	\bTR\bTD ID\eTD\bTD Jméno\eTD\bTD Příjmení\eTD\eTR
	\doloop{\lineaction\nextrow\if\Id3\exitloop\fi} %OK
\eTABLE	
\stopbuffer
\typeandgetbuffers{T13)}{lineaction}



\unexpanded\def\lineaction{\expanded{\bTR\bTD\cA\eTD\bTD\Jmeno\eTD\bTD\Prijmeni\eTD\eTR}}	


\sym{T14)}
\startbuffer[T14)]
\bTABLE
	\bTR\bTD ID\eTD\bTD Jméno\eTD\bTD Příjmení\eTD\eTR
	\doloopaction %OK provádí implicitně makro \lineaction pro všechny záznamy otevřeného CSV souboru uvnitř tabulky (netřeba \expanded) 
\eTABLE
\stopbuffer
\typeandgetbuffers{T14)}{lineaction}

	
\sym{T15)}
\startbuffer[T15)]
\bTABLE
	\bTR\bTD ID\eTD\bTD Jméno\eTD\bTD Příjmení\eTD\eTR	
	\doloopaction{\lineaction} %OK stejné jako předchozí ... makro \lineaction zadáno explicitně (netřeba \expanded)
\eTABLE
\stopbuffer
\typeandgetbuffers{T15)}{lineaction}




\sym{T16)}
\startbuffer[T16)]
\bTABLE
	\bTR\bTD ID\eTD\bTD Jméno\eTD\bTD Příjmení\eTD\eTR
	\doloopaction{\tab} %OK provede makro \tab pro všechny záznamy otevřeného CSV souboru uvnitř tabulky (netřeba \expanded)
									%\tdoloopaction{\othertab} %KIKS - nefunguje dobře, není chyba kompilace (problém s \expanded???)
\eTABLE
\stopbuffer
\typeandgetbuffers{T16)}{tab}


\sym{T17)}
\startbuffer[T17)]
\bTABLE
	\bTR\bTD ID\eTD\bTD Jméno\eTD\bTD Příjmení\eTD\eTR	
	\doloopaction{\tab}{3} %OK provede makro \tab pro záznamy 1-3 otevřeného CSV souboru uvnitř tabulky (netřeba \expanded)
\eTABLE
\stopbuffer
\typeandgetbuffers{T17)}{tab}


\sym{T18)}
\startbuffer[T18)]
\bTABLE
	\bTR\bTD ID\eTD\bTD Jméno\eTD\bTD Příjmení\eTD\eTR	
	\doloopaction{\tab}{3}{6}  %OK provede makro \tab pro záznamy 3-6 otevřeného CSV souboru uvnitř tabulky (netřeba \expanded)
 \eTABLE
\stopbuffer
\typeandgetbuffers{T18)}{tab}

	
\sym{T19)}
\startbuffer[T19)]
\bTABLE
	\bTR\bTD ID\eTD\bTD Jméno\eTD\bTD Příjmení\eTD\eTR 
  \doloopforall %OK ... ekvivalentní \tdoloopaction - udělá \lineaction pro všechny záznamy (netřeba \expanded)
\eTABLE
\stopbuffer
\typeandgetbuffers{T19)}{lineaction}

	
\sym{T20)}
\startbuffer[T20)]
\bTABLE
	\bTR\bTD ID\eTD\bTD Jméno\eTD\bTD Příjmení\eTD\eTR
	\doloopforall{\tab} %OK ... ekvivalentní \tdoloopaction - udělá \tab pro všechny záznamy (netřeba \expanded)
\eTABLE
\stopbuffer
\typeandgetbuffers{T20)}{tab}
	

\sym{T21)}
\startbuffer[T21)]
\bTABLE
	\bTR\bTD ID\eTD\bTD Jméno\eTD\bTD Příjmení\eTD\eTR  
	\doloopif{\Jmeno}{==}{Jan}{\tab} %OK provede makro \tab pro všechny záznamy CSV souboru, které vyhovují podmínce tj. rovnost prvních dvou parametrů (netřeba \expanded)
\eTABLE
\stopbuffer
\typeandgetbuffers{T21)}{tab}


\sym{T22)}
\startbuffer[T22)]
\bTABLE
	\bTR\bTD ID\eTD\bTD Jméno\eTD\bTD Příjmení\eTD\eTR  
	\doloopif{\Jmeno}{~=}{Jan}{\tab} %OK provede makro \tab pro všechny záznamy CSV souboru, které vyhovují podmínce tj. rovnost prvních dvou parametrů (netřeba \expanded)
\eTABLE
\stopbuffer
\typeandgetbuffers{T22)}{tab}

	
\sym{T23)}
\startbuffer[T23)]
\bTABLE
	\bTR\bTD ID\eTD\bTD Jméno\eTD\bTD Příjmení\eTD\eTR	
  \doloopuntil{\Jmeno}{Slávek}{\tab} %OK provede makro \tab pro všechny záznamy CSV souboru, které vyhovují podmínce tj. rovnost prvních dvou parametrů (netřeba \expanded)
\eTABLE
\stopbuffer
\typeandgetbuffers{T23)}{tab}

	
\sym{T24)}
\startbuffer[T24)]
\bTABLE
	\bTR\bTD ID\eTD\bTD Jméno\eTD\bTD Příjmení\eTD\eTR  
	\filelineaction %OK provede makro \lineaction pro všechny záznamy otevřeného CSV souboru (netřeba \expanded)
\eTABLE
\stopbuffer
\typeandgetbuffers{T24)}{lineaction}
	
	
\sym{T25)}
\startbuffer[T25)]
\bTABLE
	\bTR\bTD ID\eTD\bTD Jméno\eTD\bTD Příjmení\eTD\eTR
  \filelineaction{lisajdb1a.csv} %OK provede makro \lineaction pro všechny záznamy CSV souboru lisajdb1a.csv (netřeba \expanded)
\eTABLE
\stopbuffer
\typeandgetbuffers{T25)}{lineaction}



	
\sym{T26)}
\startbuffer[T26)]
\bTABLE
	\bTR\bTD ID\eTD\bTD Jméno\eTD\bTD Příjmení\eTD\eTR
  \filelineaction{lisajdb2a.csv}{4} %OK provede makro \lineaction pro záznamy 1-4 CSV souboru lisajdb2a.csv  (netřeba \expanded)
\eTABLE
\stopbuffer
\typeandgetbuffers{T26)}{lineaction}

	
\sym{T27)}
\startbuffer[T27)]
\bTABLE
	\bTR\bTD ID\eTD\bTD Jméno\eTD\bTD Příjmení\eTD\eTR
	\filelineaction{lisajdb2a.csv}{4}{6} %OK provede makro \lineaction pro záznamy 4-6 CSV souboru lisajdb2a.csv  (netřeba \expanded)
\eTABLE
\stopbuffer
\typeandgetbuffers{T27)}{lineaction}

	
\sym{T28)}
\startbuffer[T28)]
\bTABLE
	\bTR\bTD ID\eTD\bTD Jméno\eTD\bTD Příjmení\eTD\eTR
	\filelineaction %OK ... ekvivalentní \tdolooplineaction - provádí \lineaction pro všechny záznamy otevřeného CSV souboru 
\eTABLE
\stopbuffer
\typeandgetbuffers{T28)}{lineaction}


	
\sym{T29)}
\startbuffer[T29)]
\bTABLE
	\bTR\bTD ID\eTD\bTD Jméno\eTD\bTD Příjmení\eTD\eTR
	\filelineaction{lisajdb1a.csv}
\eTABLE
\stopbuffer
\typeandgetbuffers{T29)}{lineaction}

	
\sym{T30)}
\startbuffer[T30)]
\bTABLE
	\bTR\bTD ID\eTD\bTD Jméno\eTD\bTD Příjmení\eTD\eTR
	\filelineaction{lisajdb1a.csv}{4}
\eTABLE
\stopbuffer
\typeandgetbuffers{T30)}{lineaction}

	
\sym{T31)}
\startbuffer[T31)]
\bTABLE
	\bTR\bTD ID\eTD\bTD Jméno\eTD\bTD Příjmení\eTD\eTR
	\filelineaction{lisajdb1a.csv}{3}{7}
\eTABLE
\stopbuffer
\typeandgetbuffers{T31)}{lineaction}
 

\stopitemize

\stoptext 





SMETI:

%-----------------------------------
% http://source.contextgarden.net/syst-aux.mkiv
% OK OK OK
\def\mari{Marijana}
\edef\jm{\csvcell['Jmeno',5]}

CSVCELL \jm: TEST: \ifx\jm\mari TRUE\else FALSE \fi

\def\JMENO[#1]{\csvcell['Jmeno',#1]}
\edef\TESTJMENO{\JMENO[5]}

JMENO: \TESTJMENO\ TEST: \ifx\TESTJMENO\mari TRUE\else FALSE \fi

%-----------------------------------


\ifx\csvcell['Jmeno',5]\mari ANO1\else NE1 \fi



\def\JMENO#1{\colJmeno[#1]}

\type{\JMENO}: \JMENO


\type{\dosinglegroupempty\JMENO{5} Text}: \dosinglegroupempty\JMENO{5} Text


\type{\dosinglegroupempty\JMENO Text}: \dosinglegroupempty\JMENO Text


\type{\dosinglegroupempty\JMENO{5} Text}: \dosinglegroupempty\colJmeno{5} Text





\def\JM#1{\csvcell['Jmeno',{#1}]}

\dosinglegroupempty\JM{5}

\dosinglegroupempty\JM{7}

\dosinglegroupempty\JM{}

\csvcell['Jmeno',7]


 

sssssssssssssssssss


% \ifx\podm\mari aaa\else bbb \fi
% 
% \edef\podm{\colId}


% \podm
% 
% \ifnum\podm>2 aaa\else bbb\fi



----

\colJmeno[5]


\colJmeno

\def\mari{Marijana}

\def\podminka#1{#1}

sssssssssssssss

%\doloopif{\detokenize{\colJmeno[3]}}{==}{Marijana}{\lineaction}

\doloopfromto{1}{\numrows}{\ctxlua{if '\\colJmeno[5]'=='Marijana' then context('\\blinehook\\lineaction\\elinehook') else thirddata.handlecsv.addtonumline(-1) end}}%

