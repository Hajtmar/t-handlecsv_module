\input ../incl_handlecsv_file.tex

\def\typeexample[#1]#2#3{% buffname, title, description
\subject{#2}
#3\par
The following source code:

\blank[big]

\typebuffer[#1]

\blank[2*big]

produces the following output result:

\blank[big]

\startlines
\getbuffer[#1]
\stoplines

\blank[big]

\hairline
}



\unexpanded\def\dumpcontents{\lineno: %
	\ifissetheader %
	\Lastname\ \Firstname\ -- \Email\crlf
	\else\cA\ \cB\ -- \cI \crlf
	\fi
% 	\ifnotsetheader \Lastname, \Firstname, \Flag, \MonThu, \ThuPM, \Fri, \Sat, \Sun, \Email. \crlf
% 	\else \cA, \cB, \cC, \cD, \cE, \cF, \cG, \cH, \cI. \crlf
% 	\fi
}

\unexpanded\def\DumpContents{\lineno:%
% for lide-specialheadernames.csv  table. This table has a special (TeX problematic) header
	\ifissetheader\columncontent['Příjmení'] -- \Prijmeni, \columncontent['Jméno'] -- \Jmeno, \columncontent['Příznak'], \columncontent['123'] -- \IIIIII (MonThu), \columncontent['!?*'] -- \xxx (ThuPM), \Fri, \Sat, \Sun, \Email. \crlf
	\else\cA, \cB, \cC, \cD, \cE, \cF, \cG, \cH, \cI. \crlf
	\fi
% 	\ifnotsetheader \Lastname, \Firstname, \Flag, \MonThu, \ThuPM, \Fri, \Sat, \Sun, \Email. \crlf
% 	\else \cA, \cB, \cC, \cD, \cE, \cF, \cG, \cH, \cI. \crlf
% 	\fi
}

\unexpanded\def\lineaction{LINEACTION: \dumpcontents}
\unexpanded\def\action{ACTION: \dumpcontents}


\starttext

%\setheader
\unsetheader
\setsep{,}
% \unsetsep
%\opencsvfile{lide.csv}
\opencsvfile{lide-specialheadernames.csv}




\blank[big]


\title{Working with columns predefinded macros}

Predefined macros are:

\starttyping
\colname
\xlscolname
\indexcolname
\columncontent
\numberxlscolname
\stoptyping

\blank[big]


1. \type{\colname[2]}: \colname[2]
2. \type{\xlscolname[5]}: \xlscolname[5]
3. \type{\numberxlscolname['F']}: \numberxlscolname['F']
%4. \type{\indexcolname['Firstname']}: \indexcolname['Firstname']
5. \type{\indexcolname['B']}: \indexcolname['B']
6. \type{\indexcolname['cC']}: \indexcolname['cC']
7. \type{\columncontent[2]}: \columncontent[2]
8. \type{\columncontent['A']}: \columncontent['A']
9. \type{\columncontent['cA']}: \columncontent['cA']
10. \type{\columncontent['Firstname']}: \columncontent['Firstname']
11. \type{\columncontent['firstname']}: \columncontent['firstname']
12. \type{\columncontent['123']}: \columncontent['123']
13. \type{\columncontent['!?*']}: \columncontent['!?*']




\type{\columncontent['š23cv']}: \columncontent['š23cv']  % not exist this column

\type{\columncontent['Příjmení']}: \columncontent['Příjmení']  % not exist this column

\type{\columncontent['Prijmeni']}: \columncontent['Prijmeni']  % not exist this column

\type{\columncontent['Jméno']}: \columncontent['Jméno']  % not exist this column

\type{\columncontent['Jmeno']}: \columncontent['Jmeno']  % not exist this column



Dorecurse example 1:

\starttyping
\dorecurse{\numrows}{
\recurselevel:
\readline{\recurselevel}
\columncontent['Firstname'] -- \columncontent['Lastname'] -- \columncontent[3]
\crlf}
\stoptyping


\blank[big]

\dorecurse{\numrows}{\recurselevel: \readline{\recurselevel} \columncontent['A'] -- \columncontent['cB'] -- \columncontent[3] \crlf }


% Dorecurse example 2:

%%% \dorecurse{9}{\edef\actualcolname{\colname[\recurselevel]} \recurselevel. column: \colname[\recurselevel] -- \indexcolname[\actualcolname]\crlf}



% Luacodes

Column names:

\startluacode
	for i=1,thirddata.handlecsv.gNumCols[thirddata.handlecsv.getcurrentcsvfilename()] do
		context('colname '..i..': \\backslash '..thirddata.handlecsv.gColumnNames[thirddata.handlecsv.getcurrentcsvfilename()][i]..' (\\backslash c'..thirddata.handlecsv.ar2colnum(i)..')\\crlf')
	end
\stopluacode

Table column names:

\startluacode
context(thirddata.handlecsv.gColumnNames[thirddata.handlecsv.getcurrentcsvfilename()][1]..' [1]\\crlf')
context(thirddata.handlecsv.gColumnNames[thirddata.handlecsv.getcurrentcsvfilename()][2]..' [2]\\crlf')
context(thirddata.handlecsv.gColumnNames[thirddata.handlecsv.getcurrentcsvfilename()][3]..' [3]\\crlf')
\stopluacode

Column names macros: (\backslash colname[i]\ and\ \backslash xlscolname[i])

\dorecurse{10}{\recurselevel: \colname[\recurselevel] and \xlscolname[\recurselevel]\crlf}


\blank[big]

\hairline

\title{Working with \type{\addto} (\type{\eaddto}) macro: }

% Example of \addto (\eaddto) macro:

% init empty macro
\def\temporary{}



Content of \type{\temporary} macro is now: <\temporary>

\dorecurse{\numrows}{%
	\readline{\recurselevel}
	\eaddto\temporary{\cA, }%
}

Content of \type{\temporary} macro is now (all names): <\temporary>


\title{Working with misc macros: }



Macros:

\backslash{numrows}: \numrows

\backslash{numcols}: \numcols

\backslash{csvfilename}:  \csvfilename

\blank[big]







\title{Working with table cells }


Table cells:

\startluacode
context(thirddata.handlecsv.gTableRows[thirddata.handlecsv.getcurrentcsvfilename()][1][1]..' [first row, first column]\\crlf')
context(thirddata.handlecsv.gTableRows[thirddata.handlecsv.getcurrentcsvfilename()][1][2]..' [first row, second column]\\crlf')
context(thirddata.handlecsv.gTableRows[thirddata.handlecsv.getcurrentcsvfilename()][5][9]..' [fifth row, nineth column]\\crlf')
\stopluacode


\blank[3*big]

\hairline

Table cell contents by TeX macro:

\blank[big]



column 1, row 1 ie column \colname[1], row 1 : \csvcell[1,1]

column 2, row 3 ie column \colname[2], row 3 :   \csvcell[2,3]

column 9, row 5  ie column \colname[9], row 5 :  \csvcell[9,5]


'Lastname' column (ie first column) and first row: \backslash csvcell['Lastname',1] --> \csvcell['Lastname',1]

'Firstname' column (ie second column) and first row: \backslash csvcell['Firstname',1] --> \csvcell['Firstname',1]

'Email' column (ie nineth column) and fifth row: \backslash csvcell['Email',5] --> \csvcell['Email',5]


\blank[big]


\blank[3*big]
\hairline

Excel notation of column names:

\blank[big]

\backslash csvcell['A',1] --> \csvcell['A',1]

\backslash csvcell['B',3]  --> \csvcell['B',3]

\backslash csvcell['I',5] --> \csvcell['I',5]


\blank[big]


List of ten rows of 'Firstname' column (\backslash dorecurse loop using):

\dorecurse{10}{Firstname[\recurselevel] : \csvcell['Firstname',\recurselevel] \crlf}


\blank[3*big]

\hairline

Undefined and defined cells:

\blank[big]


\backslash csvcell['Is',5] --> \csvcell['Is',5]

\backslash csvcell['I',51] --> \csvcell['I',51]

\backslash csvcell['I',0]  --> \csvcell['I',0]

\backslash csvcell['I',50] --> \csvcell['I',50]


\blank[3*big]
\hairline

Use macro references to row and column:

\blank[big]


\def\myrow{13}
\def\mycolumn{'A'}

\backslash csvcell[\mycolumn,\myrow] --> \csvcell[\mycolumn,\myrow]


\def\myrow{31}
\def\mycolumn{'Firstname'}

\backslash csvcell[\mycolumn,\myrow] --> \csvcell[\mycolumn,\myrow]



\page

List of all CSV table by use cycle:
\blank[big]

\bTABLE
\dorecurse{\numrows}{
	\bTR
	\dorecurse{\numcols}{\bTD \csvcell[##1,#1] \eTD}
	\eTR
}
\eTABLE


\page

List of all CSV table by use cycle (other version):

\bTABLE
    \dorecurse{\numrows}{\bTR
        \dorecurse{\numcols}{\bTD \csvcell[\currentTABLEcolumn,\currentTABLErow] \eTD}
    \eTR}
\eTABLE





\blank[3*big]


Write all header names:

\blank[big]

\dostepwiserecurse{1}{\numcols}{1}{
\csvcell[\recurselevel,0],\ }


\blank[3*big]
\hairline


\part{Cycles}


\title{DOLOOP FROM - TO CYCLES}



\startbuffer[example1]
\doloopfromto{5}{8}{\dumpcontents}
\stopbuffer
\typeexample[example1]{DOLOOP FROM - TO cycle}{}





\startbuffer[example2]
\doloopforall
\stopbuffer
\typeexample[example2]{DOLOOP FOR ALL cycles}{Do lineaction macro for all lines}



\startbuffer[example3]
\doloopforall{\dumpcontents}
\stopbuffer
\typeexample[example3]{DOLOOPACTION cycles}{Do specific \type{\dumpcontents} action for all lines}



\title{DOLOOPACTION cycles}



\startbuffer[example4]
\doloopaction % implicit use \lineaction macro
\stopbuffer

\typeexample[example4]
{\type{\doloopaction} macro}
{Do specific macro \type{\lineaction} for all lines}



\startbuffer[example5]
\doloopaction{\action} % use \action macro for all lines of open CSV file
\stopbuffer
\typeexample[example5]{DO SPECIFIC ACTION (for all lines):}{DO LINEACTION (for all lines):}



\subject{DO SPECIFIC ACTION (for first lines):}

\startbuffer[example6]
\doloopaction{\action}{4} % use \action macro for first 4 lines

\stopbuffer


\typebuffer[example6]

\getbuffer[example6]



EX 6:

\typeexample[example6]
{DO SPECIFIC ACTION (for first lines)}
{\doloopaction{\action}{4} % use \action macro for first 4 lines
}



\subject{DO SPECIFIC ACTION (for lines from - to specific lines):}




\startbuffer[c7]
\doloopaction{\action}{4} % use \action macro for first 4 lines

\stopbuffer


\typebuffer[c7]

\getbuffer[c7]



%
% OK: \doloopaction{\action}{2}{5} % use \action macro for lines from 2 to 5
%
%
% FILELINEACTION:
%
% Do for all lines of opening table
%
% OK: \filelineaction
%
%
% Do for all lines of specific table name
%
% OK: \filelineaction{lide2.csv}


DOLOOPIF CYCLES:

%\doloopif{value1}{[compare_operator]}{value2}{macro_for_doing} % [compareoperators] <, >, ==(eq), ~=(neq), >=, <=, in, until, while






AAAAAAAAAAAAAAAAAA

\resetlinepointer
\dorecurse{5}{\recurselevel: \readline{\recurselevel}\dumpcontents\nextline}

\resetsernumline
\dorecurse{5}{\recurselevel: \readline[\recurselevel]\dumpcontents\nextline}


\readline[7]

\dumpcontents

SSSSSSSSSSSSS

\readline

\dumpcontents

xxxxxxxxxxxxxxx

1
\startluacode
thirddata.handlecsv.readline(1)
\stopluacode
\dumpcontents

5
\startluacode
thirddata.handlecsv.readline(5)
\stopluacode
\dumpcontents

0
\startluacode
thirddata.handlecsv.readline(0)
\stopluacode
\dumpcontents




nic

\startluacode
thirddata.handlecsv.readline(thirddata.handlecsv.gCurrentLinePointer[thirddata.handlecsv.getcurrentcsvfilename()])
\stopluacode
\dumpcontents

SSSSSSSSSSS

DOLOOP CYCLE:
% \resetlinepointer
% \doloop{
%  \linepointer:
%  \readline
%  \lineaction\nextline
%  \ifnum\linepointer>20\exitloop\fi
% }

\resetlinepointer
\doloop{
 \linepointer:
 \readline
 \lineaction\nextline
 \ifEOF\exitloop\fi
}



 XXX
\setlinepointer{7}
\doloop{%
\ifnum\linepointer<9
 \linepointer:
 \readline
 \lineaction
 \nextline
 \else
 \exitloop
\fi
}


 XXX
\resetlinepointer
\doloop{%
\ifnum\linepointer<9
 \linepointer:
 \readline
 \lineaction
 \nextline
 \else
 \exitloop
\fi
}


% \doloop
%   {Some kind of typesetting punishment \par
%    \ifnum\pageno>5 \exitloop \fi}

DOLOOP CYCLE:DOLOOP CYCLE:DOLOOP CYCLE:DOLOOP CYCLE:DOLOOP CYCLE:



nic \readline


0 \readline{0}

100 \readline{100}

6 \readline{6}

100 \readline{100}

50 \readline{50}

51 \readline{51}

'a' \readline{'a'}

a \readline{a}



\dumpcontents



\startluacode
function test(opt)
--  if (interfaces.tolist(opt) > 0)
--%  	then context('kladne')
--% 	else context('zaporne')
--%  end
    context("%s",interfaces.tolist(opt))
end

interfaces.definecommand {
    name = "test",
    arguments = {
        { "option", "list" },
    },
    macro = test,
}

\stopluacode

test: \test[5]

test \test .....



\setheader
\opencsvfile{lide.csv}

And now \backslash filelineaction

%\doloopfromto{1}{5}{\lineaction}

%\filelineaction


\Lastname

\colLastname[3]





11111111111111111111111111111111111111111111111111111111111

\def\rowfrommacro#1{\the\numexpr(#1+0)}

\csvcell['A',\rowfrommacro{\numcols}]

\csvcell['A',\linepointer]

\csvcell['A',\numrows]








\stoptext




\def\be{b*}
\def\ee{*e}

\def\typeexample[#1][#2]\be#3\ee{
\subject{#1}
\startbuffer[buffexample#1]
#3
\stopbuffer

\typebuffer[buffexample#1]

\getbuffer[buffexample#1]

}



--~ \startluacode
--~ function test(opt_1, opt_2, arg_1)
--~     context.startnarrower()
--~     context("options 1: %s",interfaces.tolist(opt_1))
--~     context.par()
--~     context("options 2: %s",interfaces.tolist(opt_2))
--~     context.par()
--~     context("argument 1: %s",arg_1)
--~     context.stopnarrower()
--~ end

--~ interfaces.definecommand {
--~     name = "test",
--~     arguments = {
--~         { "option", "list" },
--~         { "option", "hash" },
--~         { "content", "string" },
--~     },
--~     macro = test,
--~ }
--~ \stopluacode

--~ test: \test[1][a=3]{whatever}

--~ \startluacode
--~ local function startmore(opt_1)
--~     context.startnarrower()
--~     context("start more, options: %s",interfaces.tolist(opt_1))
--~     context.startnarrower()
--~ end

--~ local function stopmore(opt_1)
--~     context.stopnarrower()
--~     context("stop more, options: %s",interfaces.tolist(opt_1))
--~     context.stopnarrower()
--~ end

--~ interfaces.definecommand ( "more", {
--~     environment = true,
--~     arguments = {
--~         { "option", "list" },
--~     },
--~     starter = startmore,
--~     stopper = stopmore,
--~ } )
--~ \stopluacode

--~ more: \startmore[1] one \startmore[2] two \stopmore one \stopmore



\startluacode
for i=1,200 do
 context(i..' - '..thirddata.handlecsv.ar2xls(i)..' - '..thirddata.handlecsv.xls2ar(thirddata.handlecsv.ar2xls(i))..'\\crlf')
end
\stopluacode



function thirddata.handlecsv.texmacroisdefined(macroname) -- check whether macroname macro is defined  in ConTeXt
-- function is used to test whether the user has defined the macro \macroname. If not, it needs to define any default value
	if token.csname_name(token.create(macroname)) == macroname then
		return true
 	else
		return false
	end
end



function thirddata.handlecsv.hooksevaluation()
	for i=1,#thirddata.handlecsv.gColumnNames do
	 context(i..' - '..thirddata.handlecsv.gColumnNames[i]..', ')
	end
end



function thirddata.handlecsv.xopencsvfile(filetoscan) -- Open CSV tabule, inicialize variables
	-- otevře tabulku načte ji do globální proměnné thirddata.handlecsv.gTableRows
	-- pokud je zapnuta volba thirddata.handlecsv.gCSVHeader==true, pak do gl. proměnné thirddata.handlecsv.gColumnNames
	-- nastaví názvy sloupců ze záhlaví, pokud ne, tak se nastaví XLS notace, tj. cA, cB, cC, ...
	-- do proměnných  thirddata.handlecsv.gNumRows a  thirddata.handlecsv.gNumCols se uloží počty řádků a sloupců tabulky
	-- pokud je soubor s hlavičkou, tak se hlavičkový řádek nepočítá do počtu řádků tabulky
	-- Navíc se nadefinují ConTeXová makra  \csvfilename, \numrows a \numcols
	 thirddata.handlecsv.gNumEmptyRows=0
	 thirddata.handlecsv.gColNames={}
	 thirddata.handlecsv.gColumnNames={}
	 thirddata.handlecsv.resetlinepointer()	-- set pointer to begin table (first row)
	 thirddata.handlecsv.resetnumline()
	 local inpcsvfile=thirddata.handlecsv.handlecsvfilename(filetoscan)
	 local currentlyprocessedcsvfile = io.loaddata(inpcsvfile)
	 local mycsvsplitter = utilities.parsers.rfc4180splitter{
    	separator = thirddata.handlecsv.gCSVSeparator,
    	quote = thirddata.handlecsv.gCSVQuoter,
    	strict = true,
		}
	if thirddata.handlecsv.gCSVHeader then
	 thirddata.handlecsv.gTableRows, thirddata.handlecsv.gColumnNames = mycsvsplitter(currentlyprocessedcsvfile,true)
	 inspect(thirddata.handlecsv.gTableRows)
	 inspect(thirddata.handlecsv.gColumnNames)
	 	for i=1,#thirddata.handlecsv.gTableRows[1] do
			thirddata.handlecsv.gColNames[tostring(thirddata.handlecsv.gColumnNames[i])] = i -- for indexing use (register names of macros ie 'Firstname' etc...)
			thirddata.handlecsv.gColNames[tostring(thirddata.handlecsv.ar2xls(i))] = i -- for indexcolname macro (register names of macros ie 'A', 'B', etc...)
			thirddata.handlecsv.gColNames[tostring('c'..thirddata.handlecsv.ar2xls(i))] = i -- for indexcolname macro (register names of macros ie 'cA', 'cB', etc...)
		end
	else -- if thirddata.handlecsv.gCSVHeader
	  thirddata.handlecsv.gTableRows, thirddata.handlecsv.gColumnNames = mycsvsplitter(currentlyprocessedcsvfile)
	  inspect(thirddata.handlecsv.gTableRows)
	  thirddata.handlecsv.gColumnNames={}
	  -- ad now set column names for withoutheader situation:
		for i=1,#thirddata.handlecsv.gTableRows[1] do
		 -- OK, but not used: thirddata.handlecsv.gColumnNames[i]=thirddata.handlecsv.tmn(thirddata.handlecsv.gTableRows[1][i])
		 thirddata.handlecsv.gColumnNames[i]=tostring('c'..thirddata.handlecsv.ar2xls(i)) -- set XLS notation (fill array with XLS names of columns like 'cA', 'cB', etc.)
		 -- thirddata.handlecsv.gColumnNames[i]=tostring(thirddata.handlecsv.ar2xls(i)) -- set XLS notation (fill array with XLS names of columns  like 'A', 'B', etc.)
		 thirddata.handlecsv.gColNames[tostring(thirddata.handlecsv.ar2xls(i))] = i -- for indexcolname macro (register names of macros ie 'A', 'B', etc...)
		 thirddata.handlecsv.gColNames[tostring('c'..thirddata.handlecsv.ar2xls(i))] = i -- for indexcolname macro (register names of macros ie 'cA', 'cB', etc...)
		end
	end -- if thirddata.handlecsv.gCSVHeader

 	 thirddata.handlecsv.gNumRows=#thirddata.handlecsv.gTableRows -- Getting number of rows
  	 thirddata.handlecsv.gNumCols=#thirddata.handlecsv.gTableRows[1] -- Getting number of columns
    context([[\global\EOFfalse%]])
  	 context([[\global\notEOFtrue%]])
  	 context([[\readline%]])
  	 thirddata.handlecsv.resetmarkemptylines()
  	 thirddata.handlecsv.hooksevaluation()
end -- of thirddata.handlecsv.opencsvfile(file)

