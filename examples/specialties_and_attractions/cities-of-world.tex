
\input ../incl_handlecsv_file.tex

\def\numcountries{\numrowsof[selectedcountries.csv]}
\def\numcities{\numrowsof[cities.csv]}



\setupTABLE[split=yes,location={middle,lohi}, style=\ssx]
\setupTABLE[row][first][background=color,backgroundcolor=gray,foregroundcolor=black,height=8mm, align={middle,lohi}]
\setupTABLE[row][last][background=color,backgroundcolor=gray,foregroundcolor=black,height=8mm,align={middle,lohi}]
\setupTABLE[c][1][background=color,backgroundcolor=gray,align=middle, width=15mm,style=\ssxx]
\setupTABLE[c][2][width=4cm,style=\ssxx]
\setupTABLE[c][3][width=2.5cm,style=\ssxx]
\setupTABLE[c][4][width=7cm,style=\ssxx]



\startbuffer[tableaction]
\expanded{\bTR\bTD\SerNum \eTD\bTD\Name \eTD\bTD\Population\eTD\bTD\Altitude\eTD\eTR}%
\stopbuffer


\starttext

\title{Example of together using of two opened CSV files at the same time}

CSV file \type{selectedcountries.csv} consist of information about countries, file \type{cities.csv} consist of information about cities.
Relation is mediated through the country abbreviation field \type{Abbrev}  (in \type{selectedcountries.csv}) and field \type{CountryAbbrev} (in \type{cities.csv} file).

\blank[big]

% Open two CSV files at one time together
\setheader
\opencsvfile{cities.csv}
\opencsvfile{selectedcountries.csv}


\dorecurse{\numcountries}{ % For all countries from selectedcountries.csv file DO:
    \edef\abbrevcountry{\getcsvcellof[selectedcountries.csv]['Abbrev',\recurselevel]}

	{\bf \recursedepth.\recurselevel\  Cities of \getcsvcellof[selectedcountries.csv]['Country',\recurselevel] -- \abbrevcountry:}

	\blank[big]

	\bTABLE
	\bTR\bTD Id \eTD\bTD Name\eTD\bTD Population\eTD\bTD Altitude\eTD\eTR


		\dorecurse{\numcities}{% For all cities
		    \edef\onecountry{\getcsvcellof[cities.csv]['CountryAbbrev',\recurselevel]}
		    \edef\onecity{\getcsvcellof[cities.csv]['Name',\recurselevel]}
            \edef\Name{\getcsvcellof[cities.csv]['Name',\recurselevel]}
            \edef\SerNum{\getcsvcellof[cities.csv]['SerNum',\recurselevel]}
            \edef\Population{\getcsvcellof[cities.csv]['Population',\recurselevel]}
            \edef\Altitude{\getcsvcellof[cities.csv]['Altitude',\recurselevel]}
            % -- \abbrevcountry - \onecity - \onecountry
%%%%		\doifelse{\abbrevcountry}{\onecountry}{\expanded{\bTR\bTD\SerNum \eTD\bTD\Name \eTD\bTD\Population\eTD\bTD\Altitude\eTD\eTR}}{} % if city is in country then write it, else nothing
    		\doif{\getcsvcellof[cities.csv]['CountryAbbrev',\recurselevel]}{\abbrevcountry}{\getbuffer[tableaction] } % if city is in country then write it, else nothing
%%%%		\doif{\onecountry}{\abbrevcountry}{\getbuffer[tableaction] } % if city is in country then write it, else nothing
					}
	\bTR\bTD-- \eTD\bTD\ssx -- \eTD\bTD -- \eTD\bTD -- \eTD\eTR
	\eTABLE
	\blank[big]
}

\stoptext


\doloop{
    \ifnum \recurselevel=2
        \exitloop % exit at end of current iteration
    \fi
    \recurselevel!
}
