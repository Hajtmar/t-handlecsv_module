\mainlanguage[cz]            % Podle typografických pravidel 
\setuppapersize[A4][A4]      % Velikost papíru bude A5 a bude na stránce opět o téže velikosti. 
\setupcolors[state=start]    % Zapne používání barev.
\setuppagenumbering[state=stop]
\setuplayout[width=middle, topspace=1cm, header=1cm, footer=1cm]

%\enablemode[prvnipololeti]
\enablemode[druhepololeti]

\def\skolnirok{2014/15}




\doifmodeelse{prvnipololeti}
{\def\pololeti{první}\def\pololeticislici{1.}}
{\def\pololeti{druhé}\def\pololeticislici{2.}}


\usedirectory[c:/1da/ConTeXt/ConTeXt-Tests/my_way/ScanCSV/t-handlecsv_module/]
\usemodule[handlecsv]

\newif\ifneklasifikovanyzak

\def\zkratkapredmetu#1{\ctxlua{context(thirddata.handlecsv.ParseCSVLine('#1',':')[1])}}%

\def\neklpredmety{% Vypíše předmět zkratkou.
\neklasifikovanyzakfalse
\dostepwiserecurse{24}{58}{1}{\if\csvcell[\recurselevel,\linepointer]N \neklasifikovanyzaktrue\addtohandlecsvbuffer{\zkratkapredmetu{\csvcell[\recurselevel,0]}, }\fi}%
\dostepwiserecurse{75}{85}{1}{\if\csvcell[\recurselevel,\linepointer]N \neklasifikovanyzaktrue\addtohandlecsvbuffer{\zkratkapredmetu{\csvcell[\recurselevel,0]}, }\fi}%  
}



%\input ../texlibrary/lib-lua.tex
\input lib-lua.tex


\unexpanded\def\printaction{
\let\Trida\cA
\let\Prijmeni\cE
\let\Jmeno\cF
%\TestujNeklPredmety
\savetohandlecsvbuffer{}
\neklpredmety
\ifneklasifikovanyzak
\linepointer -- \Jmeno\ \Prijmeni\  (\Trida):\ \gethandlecsvbuffer{}\crlf 
\fi
} 


\unexpanded\def\tableaction{
\let\Trida\cA
\let\Prijmeni\cE
\let\Jmeno\cF
\savetohandlecsvbuffer{}
\neklpredmety
\ifneklasifikovanyzak
\expanded{\bTR\bTD \linepointer\eTD \bTD\Jmeno\ \Prijmeni \eTD\bTD (\Trida)\eTD\bTD\gethandlecsvbuffer{} \eTD}%
\fi
}


\starttext

\setheader

\doifmodeelse{prvnipololeti}
{\setfiletoscan{export-1pololeti.csv}}
{\setfiletoscan{export-2pololeti.csv}}
\opencsvfile

\subject{Neklasifikovaní žáci --  šk. r. \skolnirok, \pololeticislici\ pololetí} 


\doloopfromto{1}{\numrows}{
\printaction
}


\blank[3*big]


\bTABLE
\doloopfromto{1}{\numrows}{\tableaction}
\eTABLE

%\dorecurse{\numrows}{\recurselevel -- \gethandlecsvbuffer{\recurselevel}\crlf}


\stoptext


% %%%%  Nepoužitý přístup (pomocí TeXu only)
% \def\predmetzhlavicky#1:#2:#3:#4!{#2} % celé jméno předmětu z hlavičky CSV souboru (malými písmeny) např. český jazyk a literatura
% \def\vyhodnocenypredmet{\edef\predmet{\csvcell[\recurselevel,0]}\words{\expandafter\predmetzhlavicky\predmet!}}
% 
% \def\zkratkapredmetuzhlavicky#1:#2:#3:#4!{#1} % Zkratka předmětu z hlavičky CSV souboru  např. CJL
% \def\vyhodnocenazkratkapredmetu{\edef\zkratkapredmetu{\csvcell[\recurselevel,0]}\expandafter\zkratkapredmetuzhlavicky\zkratkapredmetu!}
% 
% \unexpanded\def\TestujNeklPredmety{% Testuje, zda existuje nějaké Nko.
% \neklasifikovanyzakfalse
% \dostepwiserecurse{24}{58}{1}{\if\csvcell[\recurselevel,\linepointer]N \neklasifikovanyzaktrue\fi }%
% \dostepwiserecurse{75}{85}{1}{\if\csvcell[\recurselevel,\linepointer]N \neklasifikovanyzaktrue\fi }%
% \removeunwantedspaces%
% 
% }
% 
% \def\NeklPredmety{% Vypíše předmět zkratkou.
% \dostepwiserecurse{24}{58}{1}{\if\csvcell[\recurselevel,\linepointer]N \expanded{\savetohandlecsvbuffer{\vyhodnocenazkratkapredmetu}}\addtohandlecsvbuffer{\vyhodnocenazkratkapredmetu}\vyhodnocenazkratkapredmetu,  \fi }%  \expandafter\vyhodnocenazkratkapredmetu\addtohandlecsvbuffer
% \dostepwiserecurse{75}{85}{1}{\if\csvcell[\recurselevel,\linepointer]N \vyhodnocenazkratkapredmetu, \addtohandlecsvbuffer{\vyhodnocenazkratkapredmetu}\fi }% \expandafter\vyhodnocenazkratkapredmetu\addtohandlecsvbuffer 
% }







% 
% 
% 
% \zkratkapredmetu{CJL:Cesky jazyk a literatura:P:1}
% 
% \def\zkp{CJL:Cesky jazyk a literatura:P:1}
% 
% \zkratkapredmetu{\zkp}
% 
% 
%\resethandlecsvbuffer

% \resethandlecsvbuffer
% 
% \gethandlecsvbuffer{} 
% 
% \savetohandlecsvbuffer{aaa}
% 
% \gethandlecsvbuffer{} 
% 
% \addtohandlecsvbuffer{xxx}
% 
% \gethandlecsvbuffer{} 
% 
% \addtohandlecsvbuffer{zzz asdlfkj;lfdk}
% 
% \gethandlecsvbuffer{}
% 
% \par
% 
% \gethandlecsvbuffer{} 

%\gethandlecsvbuffer{1}




% \setnumline{61}
% 
% \numline


%\gethandlecsvbuffer{71}



% \let\lineaction\printaction
% \def\recurselevel{1}
% \filelineaction

\let\Trida\cA
\let\Prijmeni\cE
\let\Jmeno\cF
\let\DatumNarozeni\cG
\let\MistoNarozeni\cH
\let\Pohlavi\cJ
\let\Mesto\cK
\let\PSC\cL
\let\Ulice\cM
\let\ZZPrijmeni\cP
\let\ZZJmeno\cQ
\let\ZZUlice\cR
\let\ZZMesto\cS
\let\ZZPSC\cT
\let\DatNar\DatumNarozeni
\let\Obec\Mesto


% \resetlinepointer
% \resethandlecsvbuffer
% \savetohandlecsvbuffer{XXX}
% \addtohandlecsvbuffer{YYY}
% \gethandlecsvbuffer{\linepointer}
% 
% 20 \gethandlecsvbuffer{20}
% 
% \linepointer  nic \gethandlecsvbuffer{}
% 
% 1 \gethandlecsvbuffer{1}
% 
% 
% \def\hlavickasloupcespredmetem{CJL:Cesky jazyk a literatura:P:1}
