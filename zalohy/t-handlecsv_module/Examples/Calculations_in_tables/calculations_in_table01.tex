\usemodule[handlecsv]

\def\eval#1{\directlua{context(tonumber(#1))}}

\def\commatodecimal#1{\directlua{context(string.gsub('#1', "," , "."))}}


\def\priceforallitems{%
\ctxlua{context(tostring(tonumber(%
[==[\commatodecimal{\columncontent['Unit Price']}]==]*[==[\commatodecimal{\columncontent['Order Quantity']}]==])))}%
% OR [==[\commatodecimal{\UnitxPrice}]==]*[==[\commatodecimal{\OrderxQuantity}]==])))}%
}


\unexpanded\def\lineaction{%
\expanded{%
\bTR
\bTD\columncontent['Product Name']\eTD
\bTD\columncontent['Order Quantity']\eTD
\bTD\commatodecimal{\columncontent['Unit Price']}\eTD 
\bTD\priceforallitems\eTD
\eTR%
}%
\eaddto\totalprice{+\priceforallitems}% ADD into 
}

\catcode`\#=12 %CSV file contains # characters (i.e. TeX problematic character)


\setupTABLE[background=color,backgroundcolor=white, offset=3pt]
\setupTABLE[row][first][background=color,backgroundcolor=lightgray, style=bold]
\setupTABLE[row][last][background=color,backgroundcolor=lightgray, style=bold]
\setupTABLE[c][1][style=\tfx]
\setupTABLE[c][2][align=middle]
\setupTABLE[c][3][alignmentcharacter={.}, aligncharacter=yes, align=middle]
\setupTABLE[c][4][alignmentcharacter={.}, aligncharacter=yes, align=middle, style=\ss]


\def\firsttableline{
\bTR
\bTD Product name\eTD
\bTD Quantity\eTD
\bTD Unit Price\eTD
\bTD Items price\eTD
\eTR
}

\def\lasttableline{
\bTR
\bTD[nc=3]TOTAL PRICE\eTD
\bTD\eval{\totalprice}\eTD
\eTR
}



\starttext

\setheader 
\setsep{;}  
\opencsvfile{superstore_sales_smalltable.csv}


\title{Using \type{\dorecurse} cycle}

\def\totalprice{0}


\bTABLE[split=yes]
\firsttableline
\dorecurse{\numrows}{
	\setlinepointer{\recurselevel}
	\lineaction
}
\lasttableline
\eTABLE




\title{Using \type{\filelineaction} cycle}

\def\totalprice{0}

\bTABLE[split=yes]
\firsttableline
\filelineaction
\lasttableline
\eTABLE

\stoptext
