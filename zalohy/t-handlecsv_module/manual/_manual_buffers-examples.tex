\startcomponent _manual_buffers-examples



\startbuffer[example.1]

\usemodule[handlecsv]  % this command load module into my source code
\setheader % set a header flag i.e. CSV file has header in first line
\setsep{,} % change settings of default separator (this is ; semicolon) 

\opencsvfile{myfirstcsvexamplefile.csv}

\unexpanded\def\sampleaction{%
\catcode`\#=12
\item \numline -- \csvcell['A',\linepointer]
}


\dorecurse{\numrows}
{\numline - \csvcell['first_name',\recurselevel]
\nextline\nextnumline\crlf
}

\startitemize[n]
\doloopaction{\sampleaction}
\stopitemize

%\filelineaction


\stopbuffer


\startbuffer[example.makronames]

\usemodule[handlecsv]  % this command load module into my source code
\setheader % set a header flag i.e. CSV file has header in first line
\setsep{,} % change settings of default separator (this is ; semicolon) 

\opencsvfile{myfirstcsvexamplefile.csv}

% {\catcode`\#=12
% {\dorecurse{\numcols}{\numline: \xlscolname[\recurselevel] -- \columncontent[\recurselevel]\crlf}}
% }


\def\onerow#1{\readline{#1} {\dorecurse{\numcols}{\lineno: \xlscolname[\recurselevel] -- \columncontent[\recurselevel]\crlf}}}


\resetlinepointer

\onerow{1}

\onerow{5}




% {\catcode`\#=12
% \dorecurse{\numcols}{\numline: \readline \xlscolname[\recurselevel] -- \columncontent[\recurselevel]\crlf}
% }

\stopbuffer



\stopcomponent
