\setupwhitespace[medium]

% \let\numcols\relax
% \let\numrows\relax
% \let\numcols\relax


\starttext
This simple sample document shows possibilities of the handlecsv.lua module.


Imagine, that CSV file \type{myfirstcsvexamplefile.csv} that I want to use for 
my print report has the so-called header (ie. the first line contains the 
names of each columns of the table) and the separator of columns is a 
semicolon.

When I need process this CSV file, then at first I have to load required 
module by command:

\starttyping
\usemodule[handlecsv] % this command load module into my source code
\stoptyping

\usemodule[handlecsv]  % this command load module into my source code
% \usemodule[handlecsv-tools]

The module now needs to know, whether CSV file contain a header and to know 
which delimiter used to divide to separate columns. The default setting of the 
module library setting can be changed to suit my needs.

\starttyping
\setheader % this set a header flag i.e. CSV file has header in first line 
\setsep{,} % change settings of default separator (this is ; semicolon)
\stoptyping

\setheader % set a header flag i.e. CSV file has header in first line 
% compared to \unsetheader or \resetheader synonyms are used for no header file
\setsep{,} % change settings of default separator (this is ; semicolon)
% \unsetsep or \resetsep synonyms change settings of default separator (;)

Now I can open my CSV file for processing by the command: 


\starttyping
\opencsvfile{myfirstcsvexamplefile.csv}
\stoptyping


\opencsvfile{myfirstcsvexamplefile.csv}

After opening the file I can find out a lot of useful information. eg typing of line:

\startbuffer[message1]
The file \csvfilename\ has \numcols columns\  and \numrows\ rows (lines).
\stopbuffer

\typebuffer[message1]

Lists message: 

\getbuffer[message1]

%\printline

%\printall

%\csvreport




% \setheader
% \ifissetheader{true}\else{false}\fi
% 
% 
% \def\rowfrommacro#1{\the\numexpr(#1+0)}




% \unsetheader
% \ifissetheader{true}\else{false}\fi


\unexpanded\def\lineaction{%
%\catcode`\_=12
%\catcode`\%=12
%\catcode`\&=12
\catcode`\#=12
%\catcode`\@=12
%\numline  \linepointer -  \csvcell['A',\the\numexpr(\linepointer+0)]\crlf % OK
%\numline  \linepointer -  \csvcell['A',\rowfrommacro{\linepointer}]\crlf %OK
%\csvcell['A',\rowfrommacro{\linepointer}] OK
%\numline  \linepointer -  \csvcell['A','\linepointer']\crlf
\ifnotemptyline\numline\ xx  \linepointer -  \csvcell['A',\linepointer] \crlf \else XX \numline \fi%OK
}


DORECURSE

\dorecurse{\numrows}{\numline / \linepointer - \csvcell['first_name',\recurselevel]\nextline\nextnumline\crlf}


\filelineaction

XXXXXXXXXXXXXXXXXXXXXXXXXXXXXXXXXXXX

\setheader % set a header flag i.e. CSV file has header in first line 
\setsep{,} % change settings of default separator (this is ; semicolon)
\opencsvfile{myfirstcsvexamplefilewithoutheader.csv}

The filename {\ttbf \csvfilename}\ has \numcols\ columns\  and \numrows\ rows (lines).

\dorecurse{\numrows}{\numline / \linepointer - \csvcell['A',\recurselevel]\nextline\nextnumline\crlf}


%\filelineaction


%\csvreport 
 

První řádek obsahuje hlavičku ()


Stejně pro soubor bez hlavičky



% \starttyping
% \usemodule[handlecsv]  % this command load module into my source code
% 
% \setheader
% \setsep{,}
% 
% 
% 
% \stoptyping



%\notmarkemptylines % TZN všechny řádky jsou považovány za neprázdné (i když v nich není žádná informace)
%\markemptylines % TZN prázdné řádky jsou rozlišovány a jsou označeny  Lépe makro s názvem OZNAČITprázdnéřádky 


\setheader
\setsep{,}
% \unsetsep
\opencsvfile{myfirstcsvexamplefilewithemptylines.csv}


The filename {\ttbf \csvfilename}\ has \numcols\ columns\  and \numrows\ rows (lines).

\dorecurse{\numrows}{\numline / \linepointer - \csvcell['A',\recurselevel]\nextline\nextnumline\crlf}


LUACODE


% -- context(tostring(thirddata.handlecsv.testemptyrow(9)))
% -- context(tostring(thirddata.handlecsv.gNumEmptyRows))

\startluacode
for i = 1, thirddata.handlecsv.gNumRows do
	if thirddata.handlecsv.emptylineevaluation(i) then context(i..'empty\\crlf') else context(i..'nonempty\\crlf') end
end
context(tostring(thirddata.handlecsv.gNumEmptyRows))
\stopluacode


EMPTYTEST

\dorecurse{\numrows}{\readline{\recurselevel}\ifemptyline \recurselevel: EMPTY \else \recurselevel: NONEMPTY \fi\nextrow\crlf}


NOTEMPTYTEST

\dorecurse{\numrows}{\readline{\recurselevel}\ifnotemptyline \recurselevel: NOTEMPTY \else \recurselevel: EMPTY \fi\nextrow\crlf}




FILELINEACTION EMPTY -- NONEMPTY

\setheader
\setsep{,}
% \unsetsep
%\notmarkemptylines
%\markemptylines

\opencsvfile{myfirstcsvexamplefilewithemptylines.csv}


\ifemptylinesmarking
EMPTY LINES ARE MARKING
\else
EMPTY LINES ARE NOT MARKING ()
\fi



\ifemptylinesnotmarking
EMPTY LINES ARE NOT MARKING
\else
EMPTY LINES ARE MARKING
\fi



\unexpanded\def\notemptylineaction{%
\catcode`\#=12
%\catcode`\@=12
\ifnotemptyline\numline\ -- \lineno / \linepointer -  \csvcell['B',\linepointer]\crlf\fi%OK
}

\unexpanded\def\emptylineaction{%
\catcode`\#=12
%\catcode`\@=12
\ifemptyline\numline\ -- \lineno / \linepointer -  nothing\crlf\fi%OK
}

% \resetnumline
% \resetlinepointer
%\filelineaction

%\unsetheader
%\notmarkemptylines
\ifissetheader
 SETHEADER = ON
\doloopaction{\notemptylineaction}
\else
	SETHEADER = OFF
\doloopaction{\emptylineaction}
\fi





\numemptyrows

\numnotemptyrows

\numrows

FILELINEACTION EMPTY -- NONEMPTY FILELINEACTION EMPTY -- NONEMPTY FILELINEACTION EMPTY -- NONEMPTY



xxxxxxxxxxxxxxxxxxxxxxxxxxx
\markemptylines


\indexofemptyline{2}

\indexofnotemptyline{2}

AAAAAAAAAAAAAAAAAAAAAAAA

ALL LINES WITH EVALUATING OF EMPTENIES

\dorecurse{\numrows}{\readline{\recurselevel}\ifnotemptyline \recurselevel -- \indexofnotemptyline{\recurselevel} : \csvcell['A', \indexofnotemptyline{\recurselevel}] \else \recurselevel: EMPTY \fi\nextrow\crlf}

BBBBBBBBBBBBBBBBBBBBBBBBBB

ONLY NONEMPTY LINES

%\dorecurse{\numnotemptyrows}{\readline{\recurselevel}\ifnotemptyline \recurselevel -- \indexofnotemptyline{\recurselevel} : \csvcell['A', \indexofnotemptyline{\recurselevel}]\crlf\fi\nextrow}
\dorecurse{\numnotemptyrows}{\readline{\recurselevel}\recurselevel -- \indexofnotemptyline{\recurselevel} : \csvcell['A', \indexofnotemptyline{\recurselevel}]\crlf\nextrow}

CCCCCCCCCCCCCCCCCCCCCCCCCCCCCC

ONLY EMPTY LINES

%\dorecurse{\numemptyrows}{\readline{\thenumexpr{\indexofemptyline{\recurselevel}}}\ifemptyline\recurselevel -- \indexofemptyline{\recurselevel} : EMPTY\crlf\fi\nextrow}
\dorecurse{\numemptyrows}{\readline{\thenumexpr{\indexofemptyline{\recurselevel}}}\recurselevel -- \indexofemptyline{\recurselevel} : EMPTY\crlf\nextrow}

DDDDDDDDDDDDDDDDDDDDDDDD


\startluacode
context(tostring(thirddata.handlecsv.gMarkingEmptyLines))

for i = 1, thirddata.handlecsv.gNumEmptyRows do
context('EMPTY:'..thirddata.handlecsv.gTableEmptyRows[i]..', ')	
thirddata.handlecsv.indexofemptyline(i)
end

for i = 1, thirddata.handlecsv.gNumRows-thirddata.handlecsv.gNumEmptyRows do
 context('NOEMPTY:'..thirddata.handlecsv.gTableNotEmptyRows[i]..', ')
 thirddata.handlecsv.indexofnotemptyline(i)	
end
\stopluacode






\stoptext







https://en.wikipedia.org/wiki/Comma-separated_values




{\bf\backslash csvfilename} -- name of open CSV file 

{\bf\backslash numcols} -- number of table columns
 
{\bf\backslash numrows} -- number of table lines
 
{\bf\backslash numline} -- number of the currently loaded row (for use in print reports)
 
{\bf\backslash lineno} sernumline -- serial number of the actual loaded line of CSV table 

{\bf\backslash csvreport} -- prints the report on file open 

{\bf\backslash printline} -- lists the current CSV row table in a condensed form
 
{\bf\backslash printall} -- CSV output table in a condensed form 

{\bf\backslash setfiletoscan}{{\it filename}} -- setting of name of CSV file

{\bf\backslash opencsvfile}{{\it filename}} -- open CSV table

{\bf\backslash setheader} -- set a header flag

{\bf\backslash unsetheader} or {\bf\backslash resetheader} -- unset a header flag

 

{\bf\backslash nextrow} -- next row of CSV table (with test of EOF)

{\bf\backslash setsep}{{\it delimiter}} -- set delimiter of columns

{\bf\backslash unsetsep} or {\bf\backslash resetsep} -- unset to default values

{\bf\backslash nextline} or {\bf\backslash nextrow} -- move pointer into next line of opening CSV file


setlinepointer, resetlinepointer

linepointer



{\bf\backslash setld}{{\it delimiter}} -- set left quoter

{\bf\backslash resetld} -- unset left quoter to default values

{\bf\backslash setrd}{{\it delimiter}} -- set right quoter

{\bf\backslash resetrd} -- unset right quoter to default values

{\bf\backslash blinehook} -- begin line hook macro (process before first column value of each row)

{\bf\backslash elinehook} -- end line hook macro (process after last column value of each row)

{\bf\backslash bfilehook} -- begin file hook macro (process before whole file processing)

{\bf\backslash efilehook} -- end file hook macro (process after whole file processing)









\resethooks
\newif\ifissetheader
\newif\ifnotsetheader
\newif\ifEOF
\newif\ifnotEOF
\newif\ifemptyline
\newif\ifnotemptyline
\newif\ifemptylinesmarking
\newif\ifemptylinesnotmarking

\numrows
\numemptyrows
\numnotemptyrows
\numcols
\numline
\nextnumline
\setnumline
\resetnumline
\linepointer
\setlinepointer
\resetlinepointer
\setfiletoscan
\setheader
\resetheader
\notmarkemptylines 
\markemptylines
\setsep
\resetsep
\csvfilename
\nextline,\nextrow
\blinehook
\elinehook
\bfilehook
\efilehook
\bch
\ech
\resethooks
\printline
\printall
\csvreport, \csvreport{} 

\xlsname
\hookxlsname
\macroname
\hookmacroname
 
\readline
\colname[#1]
\indexcolname[#1]
\xlscolname[#1]
\csvcell[#1,#2]
\doloopfromto
\doloopforall, \doloopforall{\action}
\doloopaction, \doloopaction{\action}, \doloopaction{\action}{4}, \doloopaction{\action}{2}{5}
\doloopif{value1}{[compare_operator]}{value2}{macro_for_doing} [compareoperators] <, >, ==(eq), ~=(neq), >=, <=, in, until, while
\doloopifnum
\doloopuntil
\doloopwhile
\filelineaction, \filelineaction{filename.csv} 
\thenumexpr
 


  



Due compatibility:

\let\unsetsep\resetsep
\let\unsetheader\resetheader
\let\lineno\linepointer
\let\sernumline\linepointer
\let\resetlineno\resetlinepointer
\let\resetsernumline\resetlinepointer
\def\nextrow{\readline\nextline}
