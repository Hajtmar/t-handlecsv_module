

% csvfile.txt content:
% first,second,third,fourth
% 1,"2","3","4"
% "a","b","c","d"
% "foo","bar""baz","boogie","xyzzy"
% 



\starttext

\let\numcols\relax
\let\numrows\relax
\let\numcols\relax

\startluacode
	-- gCSVHeader=true -- works fine when compiled any version of ConTeXt
	 gCSVHeader=false -- crashed, when compiled by last version of standalone
	 gColNames={}
	 gColumnNames={} 
	 local inpcsvfile='csvfile.txt'
	 local currentlyprocessedcsvfile = io.loaddata(inpcsvfile)
	 local mycsvsplitter = utilities.parsers.rfc4180splitter{
    	separator = ',',
    	quote = '"',
		}
	if gCSVHeader then
	 gTableRows, gColumnNames = mycsvsplitter(currentlyprocessedcsvfile,true)   
	 inspect(gTableRows) 
	 inspect(gColumnNames)
	else
	  gTableRows = mycsvsplitter(currentlyprocessedcsvfile)
	  inspect(gTableRows)
	  -- inspect(gColumnNames)
	end

	context('TR'..tostring(gTableRows)..'  ')
	context('TC'..tostring(gColumnNames))
	 -- gNumRows=#gTableRows -- Getting number of rows
	 --gNumCols=#ColumnNames -- Getting number of cols
 	 --gNumCols=#gTableRows[1] -- Getting number of columns

	 context.setgvalue("numrows",tostring(gNumRows))
	 context.setgvalue("numcols",tostring(gNumCols))
	 context.setgvalue("filename",tostring(inpcsvfile)) 
\stopluacode


File \filename\ has \numrows\ lines and \numcols\ columns.

\stoptext
