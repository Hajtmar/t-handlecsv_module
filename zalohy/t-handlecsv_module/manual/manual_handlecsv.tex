\startproduct manual_handlecsv

\def\buff#1{%
 	\doifelsebuffer{macro.#1}{\getbuffer[macro.#1]}{\getbuffer[macro.nic]}
}


\def\typeexample[#1]#2#3{% buffname, title, description
\subject{#2}
#3\par
The following source code: 

\blank[big]

\typebuffer[#1]

\blank[2*big]

produces the following output result:

\blank[big]

\startlines
\getbuffer[#1]
\stoplines

\blank[big]

\hairline
}





\component _manual_buffers-macros
\component _manual_buffers-examples



%\starttext

This simple sample document shows possibilities of the handlecsv.lua module.


Imagine, that CSV file \type{myfirstcsvexamplefile.csv} that I want to use for 
my print report has the so-called header (ie. the first line contains the 
names of each columns of the table) and the separator of columns is a 
semicolon.

When I need process this CSV file, then at first I have to load required 
module by command:

\starttyping
\usemodule[handlecsv] % this command load module into my source code
\stoptyping

\usemodule[handlecsv]  % this command load module into my source code


The module now needs to know, whether CSV file contain a header and to know 
which delimiter used to divide to separate columns. The default setting of the 
module library setting can be changed to suit my needs.

\starttyping
\setheader % this set a header flag i.e. CSV file has header in first line 
\setsep{,} % change settings of default separator (this is ; semicolon)
\stoptyping

\setheader % set a header flag i.e. CSV file has header in first line 
% compared to \unsetheader or \resetheader synonyms are used for no header file
\setsep{,} % change settings of default separator (this is ; semicolon)
% \unsetsep or \resetsep synonyms change settings of default separator (;)

Now I can open my CSV file for processing by the command: 


\starttyping
\opencsvfile{myfirstcsvexamplefile.csv}
\stoptyping


\opencsvfile{myfirstcsvexamplefile.csv}

After opening the file I can find out a lot of useful information. eg typing of line:

\startbuffer[message1]
The file \csvfilename\ has \numcols columns\  and \numrows\ rows (lines).
\stopbuffer

\typebuffer[message1]

Lists message: 

\getbuffer[message1]



%\getbuffer[macro.numrows]



%\buff{numrows}

%\typebuffer[example.1]


\title{macro.introductionsmacros}

\typebuffer[macro.introductionsmacros]


\title{macro.infomacros}

\typebuffer[macro.infomacros]


\title{Defining \type{\newifs} conditionals for processing testing}

For ConTeXt testing of header settings are defined two newifs:

\typebuffer[macro.newifs]

\typeexample[macro.newifs.example1]{Examples of using predefined conditionals}{Here is actual settings of these conditionals:}



\title{Predefined macros for column work processing}

There are predefined these macros for column information processing:

\typebuffer[macro.columns]


\typeexample[macro.columns.example1]{Examples of using predefined columns macros}{}



\title{example.makronames}

\getbuffer[example.makronames]



\title{macro.numrows}

\getbuffer[macro.numrows]

%\stoptext
\stopproduct 
