\setupwhitespace[medium]

\let\numcols\relax
\let\numrows\relax
\let\numcols\relax


\starttext
This simple sample document shows possibilities of the handlecsv.lua module.


Imagine, that CSV file \type{myfirstcsvexamplefile.csv} that I want to use for 
my print report has the so-called header (ie. the first line contains the 
names of each columns of the table) and the separator of columns is a 
semicolon.

When I need process this CSV file, then at first I have to load required 
module by command:

\starttyping
\usemodule[handlecsv] % this command load module into my source code
\stoptyping

\usemodule[handlecsv]  % this command load module into my source code
%\usemodule[handlecsv-tools]

The module now needs to know, whether CSV file contain a header and to know 
which delimiter used to divide to separate columns. The default setting of the 
module library setting can be changed to suit my needs.

\starttyping
\setheader % this set a header flag i.e. CSV file has header in first line 
\setsep{,} % change settings of default separator (this is ; semicolon)
\stoptyping

\setheader % set a header flag i.e. CSV file has header in first line 
% compared to \unsetheader or \resetheader synonyms are used for no header file
\setsep{,} % change settings of default separator (this is ; semicolon)
% \unsetsep or \resetsep synonyms change settings of default separator (;)

Now I can open my CSV file for processing by the command: 


\starttyping
\opencsvfile{myfirstcsvexamplefile.csv}
\stoptyping


\opencsvfile{myfirstcsvexamplefile.csv}


After opening the file I can find out a lot of useful information. eg .:

The filename {\ttbf \csvfilename}\ has \numcols columns\  and \numrows\ rows (lines).

% \setheader
% \ifissetheader{true}\else{false}\fi
% 
% 
% \def\rowfrommacro#1{\the\numexpr(#1+0)}




% \unsetheader
% \ifissetheader{true}\else{false}\fi


\unexpanded\def\lineaction{%
%\catcode`\_=12
%\catcode`\%=12
%\catcode`\&=12
\catcode`\#=12
%\catcode`\@=12
%\numline  \linepointer -  \csvcell['A',\the\numexpr(\linepointer+0)]\crlf % OK
%\numline  \linepointer -  \csvcell['A',\rowfrommacro{\linepointer}]\crlf %OK
%\csvcell['A',\rowfrommacro{\linepointer}] OK
%\numline  \linepointer -  \csvcell['A','\linepointer']\crlf
\numline\ xx  \linepointer -  \csvcell['A',\linepointer] \crlf %OK
}


DORECURSE

\dorecurse{\numrows}{\numline / \linepointer - \csvcell['first_name',\recurselevel]\nextline\crlf}


FILELINEACTION

\filelineaction

XXXXXXXXXXXXXXXXXXXXXXXXXXXXXXXXXXXX

\unsetheader % set a header flag i.e. CSV file has header in first line 
\setsep{,} % change settings of default separator (this is ; semicolon)
\opencsvfile{myfirstcsvexamplefilewithoutheader.csv}

The filename {\ttbf \csvfilename}\ has \numcols\ columns\  and \numrows\ rows (lines).

\dorecurse{\numrows}{\numline / \linepointer - \csvcell['A',\recurselevel]\crlf}

xxxxxx

\filelineaction


%\csvreport 
 

První řádek obsahuje hlavičku ()


Stejně pro soubor bez hlavičky



% \starttyping
% \usemodule[handlecsv]  % this command load module into my source code
% 
% \setheader
% \setsep{,}
% 
% 
% 
% \stoptyping

\stoptext



https://en.wikipedia.org/wiki/Comma-separated_values




{\bf\backslash csvfilename} -- name of open CSV file 

{\bf\backslash numcols} -- number of table columns
 
{\bf\backslash numrows} -- number of table lines
 
{\bf\backslash numline} -- number of the currently loaded row (for use in print reports)
 
{\bf\backslash lineno} sernumline -- serial number of the actual loaded line of CSV table 

{\bf\backslash csvreport} -- prints the report on file open 

{\bf\backslash printline} -- lists the current CSV row table in a condensed form
 
{\bf\backslash printall} -- CSV output table in a condensed form 

{\bf\backslash setfiletoscan}{{\it filename}} -- setting of name of CSV file

{\bf\backslash opencsvfile}{{\it filename}} -- open CSV table

{\bf\backslash setheader} -- set a header flag

{\bf\backslash unsetheader} or {\bf\backslash resetheader} -- unset a header flag

 

{\bf\backslash nextrow} -- next row of CSV table (with test of EOF)

{\bf\backslash setsep}{{\it delimiter}} -- set delimiter of columns

{\bf\backslash unsetsep} or {\bf\backslash resetsep} -- unset to default values

{\bf\backslash nextline} or {\bf\backslash nextrow} -- move pointer into next line of opening CSV file


setlinepointer, resetlinepointer

linepointer



{\bf\backslash setld}{{\it delimiter}} -- set left quoter

{\bf\backslash resetld} -- unset left quoter to default values

{\bf\backslash setrd}{{\it delimiter}} -- set right quoter

{\bf\backslash resetrd} -- unset right quoter to default values

{\bf\backslash blinehook} -- begin line hook macro (process before first column value of each row)

{\bf\backslash elinehook} -- end line hook macro (process after last column value of each row)

{\bf\backslash bfilehook} -- begin file hook macro (process before whole file processing)

{\bf\backslash efilehook} -- end file hook macro (process after whole file processing)





-- ConTeXt source:
local string2print=[[
\resethooks
\newif\ifissetheader
\newif\ifnotsetheader
\newif\ifEOF
\newif\ifnotEOF
\newif\ifemptyline %
\newif\ifnotemptyline %


% Macros defining above in source text:
%%% \numrows
%%% \numcols
% \numline
% \linepointer
% \setlinepointer
% \resetlinepointer
% \setfiletoscan
% \setheader
% \resetheader 
% \setsep
% \resetsep
% \csvfilename
% \nextline \nextrow
% \blinehook
% \elinehook
% \bfilehook
% \efilehook
% \bch
% \ech
% \resethooks
% \printline
% \printall
% \csvreport, \csvreport{} 



% Macros defining bottom in source text:
% \opencsvfile, \opencsvfile{} 
% \readline
% \colname[#1]
% \indexcolname[#1]
% \xlscolname[#1]
% \csvcell[#1,#2]
% \doloopfromto{from}{to}{action}
% \doloopforall, \doloopforall{\action}
% \doloopaction, \doloopaction{\action}, \doloopaction{\action}{4}, \doloopaction{\action}{2}{5}
% \doloopif{value1}{[compare_operator]}{value2}{macro_for_doing} % [compareoperators] <, >, ==(eq), ~=(neq), >=, <=, in, until, while
% \doloopifnum
% \doloopuntil#1#2#3{\doloopif{#1}{until}{#2}{#3}}% \doloopuntil{\Trida}{3.A}{\tableaction}  % List all, until the test is not met - then just quit. Repeat until satisfied. If it is not never met, will list all records.
% \doloopwhile#1#2#3{\doloopif{#1}{while}{#2}{#3}}% \doloopwhile{\Trida}{3.A}{\tableaction}  % List when the test is met, other just quit.
% \filelineaction, \filelineaction{filename.csv} 
% 
 


  



% Due compatibility:
\let\unsetsep\resetsep
\let\unsetheader\resetheader
\let\lineno\linepointer
\let\sernumline\linepointer
\let\resetlineno\resetlinepointer
\let\resetsernumline\resetlinepointer
\def\nextrow{\readline\nextline}

