\startbuffer[about_hooks]



% SOURCE CODE:
\usemodule[handlecsv]

% in CSV file are used # characters (i.e. TeX problematic character)
\catcode`\#=12 

% define beginning general hook (\bfilehook ie begin file hook) to insert before processing of CSV file
\def\bfilehook{\bTABLE } 
% define closing general hook (\efilehook ie end file hook) to insert after processing of CSV file
\def\efilehook{\eTABLE }
% define  beginning general hook (\blinehook ie begin line hook) to insert before processing of each separate line
\def\blinehook{\bTR }
% define  closing general hook (\elinehook ie end line hook) to insert after processing of each separate line
\def\elinehook{\eTR }

% define  beginning general hook (\bch i.e. begin column hook) to insert before processing of each separate column
\def\bch{\bgroup\ssa }
% define  closing general hook (\bch i.e. end column hook) to insert after processing of each separate column
\def\ech{\egroup }


% You can also use special columns hooks to marking specific columns items
% When you use the following settings of these hooks:
\def\bchcA{\bgroup\blue\bf }% begin column hook of column A
\def\echcA{\egroup}% end column hook of column A
\def\bchcB{\bgroup\red\ssbfc }% begin column hook of column B
\def\echcB{\egroup}% end column hook of column B
% then you can use a macros \hcA and \hcB (i.e hooked columns A and B)
% explanation: \def\hcA{\bchcA\cA\echcA} etc.


% When you use the following settings of these hooks too:
\def\bchfirstxname{\bgroup\yellow\bf }
\def\echfirstxname{\egroup}
\def\bchlastxname{\bgroup\darkgray\bf }
\def\echlastxname{\egroup}
\def\bchemail{\bgroup\green\it }
\def\echemail{\egroup}
% then you can use a macros \hfirstxname, \hlastxname and \hemail
% explanation: \def\hfirstxname{\bchfirstxname\firstxname\echfirstxname} etc.

\usehooks % it is necessary turn on using hooks while process CSV file
\setheader
\setsep{,} 
% first_name, last_name, company_name, address, city, county, state, zip,
% phone1, phone2, email, web
\opencsvfile{myfirstcsvexamplefile.csv}


\starttext
\title{Example of using hooks}

Note: hooked macros can not be tested in cycles. Use unhooked macros for this.

\blank[big]

\unexpanded\def\tableaction{%
\expanded{%
\bTD
unhooked: \columncontent['first_name']\crlf 
hooked cA: \hcA\crlf 
unhooked: \firstxname\crlf 
hooked firstxname: \hfirstxname
\eTD
\bTD 
unhooked: \columncontent['last_name']\crlf
hooked cB: \hcB\crlf 
unhooked: \lastxname\crlf 
hooked lastxname: \hlastxname
\eTD%
\bTD 
unhooked: \email\crlf
hooked email: \hemail
\eTD%
}%
}%

\doloopfromto{2}{5}{\tableaction}


\blank[big]


\unexpanded\def\othertableaction{\expanded{ \hcA\hcB\hemail}}%
% define  beginning hook to insert before processing of each separate column
\def\bch{\bTD }
% define  closing hook to insert after processing of each separate column
\def\ech{\eTD }

\doloopfromto{2}{5}{\othertableaction}

\startlines
\typebuffer[about_hooks]
\stoplines


\stoptext
\stopbuffer
 

\getbuffer[about_hooks]

