\usemodule[handlecsv]
%\usemodule[handlecsv-tools]

\def\lastname{\dosingleempty\dolastname}

\def\dolastname[#1]%
   {\iffirstargument%
	 		arg. #1
    \else
    	arg. nic
   \fi
}


%\unexpanded\def\lineaction{\numrows \cA, \cB, \cC, \cD, \cE, \cF, \cG, \cH, \cI. \crlf}


\unexpanded\def\vypisobsah{VYPISOBSAH \lineno:\sernumline: %
% for lide-specialheadernames.csv  table. This table has a special (TeX problematic) header
	\ifissetheader\Lastname, \Firstname, \Flag, \MonThu, \ThuPM, \Fri, \Sat, \Sun, \Email. \crlf
	\else\cA, \cB, \cC, \cD, \cE, \cF, \cG, \cH, \cI. \crlf
	\fi
% 	\ifnotsetheader \Lastname, \Firstname, \Flag, \MonThu, \ThuPM, \Fri, \Sat, \Sun, \Email. \crlf
% 	\else \cA, \cB, \cC, \cD, \cE, \cF, \cG, \cH, \cI. \crlf
% 	\fi
}

\unexpanded\def\Vypisobsah{VYPISOBSAH \lineno:\sernumline: %
\ifissetheader\columncontent['Příjmení'] -- \Prijmeni, \columncontent['Jméno'] -- \Jmeno, \columncontent['Příznak'], \columncontent['123'] -- \IIIIII (MonThu), \columncontent['!?*'] -- \xxx (ThuPM), \Fri, \Sat, \Sun, \Email. \crlf
	\else\cA, \cB, \cC, \cD, \cE, \cF, \cG, \cH, \cI. \crlf
	\fi
% 	\ifnotsetheader \Lastname, \Firstname, \Flag, \MonThu, \ThuPM, \Fri, \Sat, \Sun, \Email. \crlf
% 	\else \cA, \cB, \cC, \cD, \cE, \cF, \cG, \cH, \cI. \crlf
% 	\fi
}


\unexpanded\def\lineaction{LINEACTION \lineno:\sernumline: \Vypisobsah}

\unexpanded\def\action{ACTION \lineno:\sernumline: \Vypisobsah}

%\def\Vypisobsah{\cA\crlf}

\setheader
\setsep{,}
% \unsetsep
%\opencsvfile{lide.csv}
%\opencsvfile{lide-withoutheader.csv}
\opencsvfile{lide-specialheadernames.csv}


\starttext

%\csvreport%{lide.csv}
 
For ConTeXt testing of header settings are defined two newifs:

\backslash ifissetheader --> \ifissetheader TRUE \else FALSE\fi

\backslash ifnotsetheader  --> \ifnotsetheader TRUE \else FALSE\fi



\blank[big]


\colname[1]


\type{\colname[2]}: \colname[2]

\type{\xlscolname[5]}: \xlscolname[5]


\type{\indexcolname['Firstname']}: \indexcolname['Firstname']

\type{\indexcolname['cC']}: \indexcolname['cC']

 
 
\type{\columncontent[2]}: \columncontent[2]

\type{\columncontent['Firstname']}: \columncontent['Firstname']

\type{\columncontent['B']}: \columncontent['B']

\type{\columncontent['Firstnam']}: \columncontent['Firstnam'] % not exist this column

\type{\columncontent['š23cv']}: \columncontent['š23cv']  % not exist this column

\type{\columncontent['Příjmení']}: \columncontent['Příjmení']  % not exist this column

\type{\columncontent['Prijmeni']}: \columncontent['Prijmeni']  % not exist this column

\type{\columncontent['Jméno']}: \columncontent['Jméno']  % not exist this column

\type{\columncontent['Jmeno']}: \columncontent['Jmeno']  % not exist this column

\type{\columncontent['123']}: \columncontent['123']  % not exist this column

\type{\columncontent['!?*']}: \columncontent['!?*']  % not exist this column



\type{\numberxlscolname['F']}: \numberxlscolname['F']
 
 

\dorecurse{\numrows}{\recurselevel: \readline{\recurselevel} \columncontent['Příjmení'] -- \columncontent['Jméno'] -- \columncontent['!?*']\crlf } 
 
% \dorecurse{9}{\edef\actualcolname{\colname[\recurselevel]}\recurselevel: \colname[\recurselevel] -- \indexcolname[\actualcolname]\crlf
% }


Macros:

\backslash{numrows}: \numrows

\backslash{numcols}: \numcols

\backslash{csvfilename}:  \csvfilename

\blank[big]


Column names:

\startluacode
	for i=1,thirddata.handlecsv.gNumCols do
		context('colname '..i..': \\backslash '..thirddata.handlecsv.gColumnNames[i]..' (\\backslash c'..thirddata.handlecsv.ar2colnum(i)..')\\crlf')
	end
\stopluacode

Table column names:

\startluacode
context(thirddata.handlecsv.gColumnNames[1]..' [1]\\crlf')
context(thirddata.handlecsv.gColumnNames[2]..' [2]\\crlf')
context(thirddata.handlecsv.gColumnNames[3]..' [3]\\crlf')
\stopluacode

Column names macros: (\backslash colname[i]\ and\ \backslash xlscolname[i])

\dorecurse{10}{\recurselevel: \colname[\recurselevel] and \xlscolname[\recurselevel]\crlf}


\blank[big]




\blank[big]

Table cells:
 
\startluacode
context(thirddata.handlecsv.gTableRows[1][1]..' [first row, first column]\\crlf')
context(thirddata.handlecsv.gTableRows[1][2]..' [first row, second column]\\crlf')
context(thirddata.handlecsv.gTableRows[5][9]..' [fifth row, nineth column]\\crlf')
\stopluacode

\blank[big]

Table cell contents by TeX macro:

row 1, column 1 \csvcell[1,1]

row 2, column 1 \csvcell[2,1]

row 5, column 9 \csvcell[9,5]


'Lastname' column (ie first column) and first row: \backslash csvcell['Lastname',1] --> \csvcell['Lastname',1]
 
'Firstname' column (ie second column) and first row: \backslash csvcell['Firstname',1] --> \csvcell['Firstname',1]
 
'Email' column (ie nineth column) and fifth row: \backslash csvcell['Email',5] --> \csvcell['Email',5]

\backslash csvcell['A',1] --> \csvcell['A',1]

\backslash csvcell['Prijmeni',1] --> \csvcell['Prijmeni',1]

\backslash csvcell['Příjmení',1] --> \csvcell['Příjmení',1] % OK


\backslash csvcell['Jméno',1] --> \csvcell['Jméno',1]  % OK

\backslash csvcell['Příznak',1] --> \csvcell['Příznak',1]  % OK


\backslash csvcell['B',3]  --> \csvcell['B',3]

\backslash csvcell['I',5] --> \csvcell['I',5]


List of ten rows of 'Firstname' column (\backslash dorecurse loop using):

\dorecurse{10}
 {Firstname[\recurselevel] : \csvcell['Firstname',\recurselevel] \crlf}


Undefined and defined cells: 

\backslash csvcell['Is',5] --> \csvcell['Is',5]

\backslash csvcell['I',51] --> \csvcell['I',51]

\backslash csvcell['I',0]  --> \csvcell['I',0] 

\backslash csvcell['I',50] --> \csvcell['I',50]


\def\myrow{13}
\def\mycolumn{'A'}

\backslash csvcell[\mycolumn,\myrow] --> \csvcell[\mycolumn,\myrow]


\def\myrow{31}
\def\mycolumn{'Firstname'}

\backslash csvcell[\mycolumn,\myrow] --> \csvcell[\mycolumn,\myrow]




List of all CSV table:

\bTABLE
\dorecurse{\numrows}{
	\bTR
	\dorecurse{\numcols}{\bTD \csvcell[##1,#1] \eTD}
	\eTR
}
\eTABLE


OR:

\bTABLE
    \dorecurse{\numrows}{\bTR
        \dorecurse{\numcols}{\bTD \csvcell[\currentTABLEcolumn,\currentTABLErow] \eTD}
    \eTR}
\eTABLE
   




\blank[big]

% OK
% \dorecurse{51}
%  {\readline{\recurselevel} \par
% 	\recurselevel: \Vypisobsah
% }


\blank[big]


\dostepwiserecurse{1}{\numcols}{1}{
\csvcell[\recurselevel,0]
}



%DOLOOP FROM - TO:
%
% OK: \doloopfromto{3}{10}{\Vypisobsah}
% 
% OK: \doloopfromto{5}{8}{\lineaction}
% 
% DOLOOP FOR ALL:
% 
% do Lineaction
% 
% OK: \doloopforall
% 
% do specific action
% 
% OK: \doloopforall{\Vypisobsah}
% 
% 
% DOLOOPACTION:
% 
% DO LINEACTION (for all lines):
% 
% OK: \doloopaction % implicit use \lineaction macro
% 
% DO SPECIFIC ACTION (for all lines):
% 
% OK: \doloopaction{\action} % use \action macro for all lines of open CSV file
% 
% DO SPECIFIC ACTION (for first lines):
% 
% OK: \doloopaction{\action}{4} % use \action macro for first 4 lines
% 
% DO SPECIFIC ACTION (for lines from - to specific lines):
% 
% OK: \doloopaction{\action}{2}{5} % use \action macro for lines from 2 to 5
% 
% 
% FILELINEACTION:
% 
% Do for all lines of opening table
% 
% OK: \filelineaction
% 
% 
% Do for all lines of specific table name 
% 
% OK: \filelineaction{lide2.csv}


DOLOOPIF CYCLES:

%\doloopif{value1}{[compare_operator]}{value2}{macro_for_doing} % [compareoperators] <, >, ==(eq), ~=(neq), >=, <=, in, until, while 






AAAAAAAAAAAAAAAAAA

\resetlinepointer
\dorecurse{5}{\recurselevel: \readline{\recurselevel}\Vypisobsah\nextline}

\resetsernumline
\dorecurse{5}{\recurselevel: \readline[\recurselevel]\Vypisobsah\nextline}


\readline[7]

\Vypisobsah

SSSSSSSSSSSSS

\readline

\Vypisobsah

xxxxxxxxxxxxxxx

1
\startluacode
thirddata.handlecsv.readline(1)
\stopluacode
\Vypisobsah

5
\startluacode
thirddata.handlecsv.readline(5)
\stopluacode
\Vypisobsah

0
\startluacode
thirddata.handlecsv.readline(0)
\stopluacode
\Vypisobsah

nic
\startluacode
thirddata.handlecsv.readline(thirddata.handlecsv.gCurrentLinePointer)
\stopluacode
\Vypisobsah



SSSSSSSSSSS

DOLOOP CYCLE:
% \resetlinepointer
% \doloop{ 
%  \linepointer: 
%  \readline
%  \lineaction\nextline
%  \ifnum\linepointer>20\exitloop\fi
% }

\resetlinepointer
\doloop{ 
 \linepointer: 
 \readline
 \lineaction\nextline
 \ifEOF\exitloop\fi
}
 
 
 
 XXX
\setlinepointer{7}
\doloop{%
\ifnum\linepointer<9
 \linepointer: 
 \readline
 \lineaction
 \nextline
 \else
 \exitloop 
\fi
} 


 XXX
\resetlinepointer
\doloop{%
\ifnum\linepointer<9
 \linepointer: 
 \readline
 \lineaction
 \nextline
 \else
 \exitloop 
\fi
} 


% \doloop
%   {Some kind of typesetting punishment \par
%    \ifnum\pageno>5 \exitloop \fi}
   
DOLOOP CYCLE:DOLOOP CYCLE:DOLOOP CYCLE:DOLOOP CYCLE:DOLOOP CYCLE:


nic \readline


0 \readline{0}

100 \readline{100}

6 \readline{6}

100 \readline{100}

50 \readline{50}

51 \readline{51}

'a' \readline{'a'}

a \readline{a}



%\Vypisobsah




\startluacode
function test(opt)
--  if (interfaces.tolist(opt) > 0) 
--%  	then context('kladne') 
--% 	else context('zaporne') 
--%  end
    context("%s",interfaces.tolist(opt))
end

interfaces.definecommand {
    name = "test",
    arguments = {
        { "option", "list" },
    },
    macro = test,
}

\stopluacode

test: \test[5]

test \test .....


\opencsvfile{lide.csv}

And now \backslash filelineaction 

%\doloopfromto{1}{5}{\lineaction}

%\filelineaction


\lastname

\lastname[3]



11111111111111111111111111111111111111111111111111111111111

\def\rowfrommacro#1{\the\numexpr(#1+0)}

\csvcell['A',\rowfrommacro{\numcols}]

\csvcell['A',\linepointer]

\csvcell['A',\numrows]








\stoptext



--~ \startluacode
--~ function test(opt_1, opt_2, arg_1)
--~     context.startnarrower()
--~     context("options 1: %s",interfaces.tolist(opt_1))
--~     context.par()
--~     context("options 2: %s",interfaces.tolist(opt_2))
--~     context.par()
--~     context("argument 1: %s",arg_1)
--~     context.stopnarrower()
--~ end

--~ interfaces.definecommand {
--~     name = "test",
--~     arguments = {
--~         { "option", "list" },
--~         { "option", "hash" },
--~         { "content", "string" },
--~     },
--~     macro = test,
--~ }
--~ \stopluacode

--~ test: \test[1][a=3]{whatever}

--~ \startluacode
--~ local function startmore(opt_1)
--~     context.startnarrower()
--~     context("start more, options: %s",interfaces.tolist(opt_1))
--~     context.startnarrower()
--~ end

--~ local function stopmore(opt_1)
--~     context.stopnarrower()
--~     context("stop more, options: %s",interfaces.tolist(opt_1))
--~     context.stopnarrower()
--~ end

--~ interfaces.definecommand ( "more", {
--~     environment = true,
--~     arguments = {
--~         { "option", "list" },
--~     },
--~     starter = startmore,
--~     stopper = stopmore,
--~ } )
--~ \stopluacode

--~ more: \startmore[1] one \startmore[2] two \stopmore one \stopmore



\startluacode
for i=1,200 do 
 context(i..' - '..thirddata.handlecsv.ar2xls(i)..' - '..thirddata.handlecsv.xls2ar(thirddata.handlecsv.ar2xls(i))..'\\crlf')
end
\stopluacode

