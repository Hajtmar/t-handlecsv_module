\setuppapersize[A4][A4]    

\setuppagenumbering[state=stop]% , alternative=doublesided]
\setuplayout[
	backspace=0mm,  
	leftmargin=0mm, % Nenastavovat!!!
	width=middle, % Nenastavovat!!!
	rightmargin=0mm, % Nenastavovat!!!
	cutspace=0mm,
	topspace=0mm, 
	headerdistance=0mm, % Nenastavovat!!! 
	header=0mm, % Nenastavovat!!!
	height=middle, % Nenastavovat!!!
	footerdistance=0mm, % Nenastavovat!!!
	footer=0mm, % Nenastavovat!!!
	bottomspace=0.01mm, %!!!! Pokud nastavím přímo 0mm, tak se použije pro bottomspace hodnota topspace
	location=middle,
]

\newcount\countofhor
\newcount\countofvert
\newdimen\distancebetweencards
\newdimen\distancecelkem
\newdimen\cardwidth
\newdimen\cardheight



\countofhor=1 % hodnota potřebná pro luaskript, když není nastavena, tiskne se automaticky
\countofvert=4 % hodnota potřebná pro luaskript
\distancebetweencards=1mm


\cardheight=\dimexpr(\makeupheight/\countofvert)
\cardheight=\dimexpr(\cardheight - \dimexpr(0.5mm))
\cardwidth=\dimexpr((\makeupwidth-\dimexpr(\distancebetweencards)*(\countofhor-1))/\countofhor)

\setupcolors[state=start]    % Zapne používání barev.

% Kompilace - minimals
\usemodule[handlecsv] % Kompilace pomocí ConTeXtu ala TeXLive 2011
\setheader
\opencsvfile{tickets.csv}

\defineoverlay[pozadifront][{\externalfigure[empty_ticket.pdf][width=\cardwidth,height=\cardheight, frame=off]}]

\unexpanded\def\lineaction{%
\framed[height=\cardheight, width=\cardwidth,background={pozadifront}, frame=off]{%
\setuppositioning[unit=mm, xoffset=-5mm, yoffset=13mm]
\startpositioning
{\position(0,0){\framed[width=1.5cm, frame=off, align=middle]{\ss\bf\Rada.}}}
{\position(140,0){\framed[width=1.5cm, frame=off, align=middle]{\ss\bf\Rada.}}}
{\position(0,8){\framed[width=1.5cm, frame=off, align=middle]{\ss\bf\Sedadlo}}}
{\position(140,8){\framed[width=1.5cm, frame=off, align=middle]{\ss\bf\Sedadlo}}}
\stoppositioning}
} 



\starttext%
 	\filelineaction{tickets.csv}
\stoptext
