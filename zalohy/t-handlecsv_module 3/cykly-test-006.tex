\usemodule[handlecsv]
%\usemodule[handlecsv-tools]

\unexpanded\def\tab{\expanded{\bTR\bTD\cA - \Jmeno\eTD\bTD\Jmeno\eTD\bTD\Prijmeni\eTD\eTR}}
\unexpanded\def\xtab{\expanded{\bTR\bTD\cA - \Jmeno\eTD\bTD\Jmeno\eTD\bTD\Prijmeni\eTD\eTR}}



\setupcolors[state=start]

%\def\bch{\bgroup\bf\red }
%\def\ech{\egroup}

\def\bch{\bgroup\red\bf }
\def\ech{\egroup}

\starttext


\title{Netabulkové použití} 

\setsep{,}
\setheader
\opencsvfile{lisajdb1a.csv}

%%\printline

%%\printall

%%\csvreport 

%\page


\unexpanded\def\lineaction{ \cA - \Jmeno\ \Prijmeni\par}
\unexpanded\def\akce{Akce \cA - \Jmeno\ \Prijmeni (NumHod: \NumHod)\par}

\let\myakce\akce

%\let\udef\def
%\udef#1#2{\unexpanded\def#1{#2}}
%\udef\myakce{Akce \cA - \Jmeno\ \Prijmeni (NumHod: \NumHod)\par}


<5

\opencsvfile
\doloopif{\Id}{<}{5}{\myakce} %OK

\thinrule



>4

\doloopifnum{\Id}{>}{4}{\akce} %OK

\thinrule

= NumHod

\doloopifnum{\Id}{==}{\NumHod}{\akce} %OK

\thinrule

<= NumHod

\doloopifnum{\Id}{<=}{29}{\akce}

\thinrule

>= NumHod

\doloopifnum{\Id}{>=}{\NumHod}{\akce}

\thinrule


\bTABLE
	\bTR\bTD ID\eTD\bTD Jméno\eTD\bTD Příjmení\eTD\eTR
	
\doloopifnum{\Id}{>=}{5}{\tab}

\eTABLE

\blank[big]

\page

\startitemize[n]



	\sym{1}\par
  \opencsvfile	
	\doloopaction % implicitně používá \lineaction %OK 
	
	\sym{2}\par
  \opencsvfile
	\doloopaction{\akce} %OK
	
	\sym{3}\par
  \opencsvfile
	\doloopaction{\akce}{4} %OK
	
	\sym{4}\par
  \opencsvfile
	\doloopaction{\akce}{2}{5} %OK
	
	\sym{5}\par
  \opencsvfile
	\doloopforall %OK
	
	\sym{6}\par
  \opencsvfile
	\doloopforall{\akce} %OK
	
	
	\sym{7}\par
  \opencsvfile
	\doloopif{Jan}{==}{\Jmeno}{\akce} % OK
  
  \sym{8}\par
  \opencsvfile
	\doloopif{Jan}{~=}{\Jmeno}{\akce} % OK  

	
	\sym{9}\par
  \opencsvfile
	\doloopuntil{\Jmeno}{Jan}{\akce} %OK
	
	\sym{10}\par
  \opencsvfile
	\dorecurse{5}{\akce\readline\nextrow} % OK

	
	\sym{11}\par
  \opencsvfile
	\doloop{\lineaction\readline\nextrow\ifnum\numline>7\exitloop\fi} %OK

	\sym{12}\par
  \opencsvfile
	\doloop{\ifEOF\exitloop\else\lineaction\readline\nextrow\fi} %OK
	
	\sym{13}\par
  \opencsvfile
	\doloop{\lineaction\readline\nextrow\if\Id3\exitloop\fi} %OK

	
	\sym{14}\par
  \opencsvfile
	\filelineaction %OK
	
	\sym{15}\par
  \opencsvfile
	\filelineaction{lisajdb2a.csv} %OK
	
	\sym{16}\par
  \opencsvfile
	\filelineaction{lisajdb1a.csv}{3} %OK
	
	\sym{17}\par
  \opencsvfile
	\filelineaction{lisajdb2a.csv}{3}{6} %OK

	
%	A16\sym{}\par
% \opencsvfile
%  %	\for \px=1 \to 5 \step 1 \do {\akce\readline\nextrow} % OK
  

	\sym{18}\par
  \opencsvfile
	\dostepwiserecurse{1}{6}{1}{\akce\readline\nextrow}
	
  
  
	\sym{19}\par
  \opencsvfile
	\doloopfromto{3}{7}{\akce}	

 
 \stopitemize


%\printline

%\printall

%\csvreport 

\page



\title{Tabulkové použití} 

% Nyní ukázky použití v tabulkovém režimu. Toto použití vyžaduje expanzi uvnitř prostředí \bTABLE ... \eTABLE
% nebo použití jiných maker ... 


\unexpanded\def\lineaction{\expanded{\bTR\bTD\cA\eTD\bTD\Jmeno\eTD\bTD\Prijmeni\eTD\eTR}}
\unexpanded\def\tab{\expanded{\bTR\bTD\cA - \Jmeno\eTD\bTD\Jmeno\eTD\bTD\Prijmeni\eTD\eTR}}
\unexpanded\def\xtab{\expanded{\bTR\bTD\cA - \Jmeno\eTD\bTD\Jmeno\eTD\bTD\Prijmeni\eTD\eTR}}
\unexpanded\def\tableaction{\expanded{\bTR\bTD\cA\eTD\bTD\Jmeno\eTD\bTD\Prijmeni\eTD\eTR}}
\unexpanded\def\xtableaction{\expanded{\bTR\bTD\cA\eTD\bTD\Jmeno\eTD\bTD\Prijmeni\eTD\eTR}}


\setheader
\opencsvfile{lisajdb1a.csv}



\startitemize[n]
\sym{T1}\par
\opencsvfile

\bTABLE
	\bTR\bTD ID\eTD\bTD Jméno\eTD\bTD Příjmení\eTD\eTR
	\doloopaction %OK implicitně používá \lineaction 
						%\doloopaction{\tab} %NOK - bez kompilační chyby, ale vypisuje všechny stejné záznamy
						%\doloopaction{\tab}{4} %NOK - bez kompilační chyby, ale vypisuje všechny stejné záznamy
						%\doloopaction{\tab}{2}{5} %NOK - bez  kompilační chyby, ale vypisuje všechny stejné záznamy  
										%\doloopaction{\xtab} %KIKS - kompilace končí zastavením a chybou (nesnáší \expanded ???)
						%\doloopforall{\tab}  %NOK - bez kompilační chyby, ale vypisuje všechny stejné záznamy
\eTABLE



\sym{T2}\par
\bTABLE
	\bTR\bTD ID\eTD\bTD Jméno\eTD\bTD Příjmení\eTD\eTR
\doloopfromto{3}{7}{\lineaction}
\eTABLE


\sym{T3}\par
\opencsvfile

\bTABLE
	\bTR\bTD ID\eTD\bTD Jméno\eTD\bTD Příjmení\eTD\eTR
	\doloopforall{\tab}  %OK
	\doloopif{Jan}{==}{\Jmeno}{\tab} %NOK - bez  kompilační chyby, ale vypisuje všechny stejné záznamy
\eTABLE


\sym{T4}\par
\opencsvfile

\bTABLE
	\bTR\bTD ID\eTD\bTD Jméno\eTD\bTD Příjmení\eTD\eTR
	\doloopif{Jan}{~=}{\Jmeno}{\tab} %NOK - bez  kompilační chyby, ale vypisuje všechny stejné záznamy
\eTABLE



\sym{T5}\par
\opencsvfile

\bTABLE
	\bTR\bTD ID\eTD\bTD Jméno\eTD\bTD Příjmení\eTD\eTR
	\doloopif{Jan}{==}{\Jmeno}{\tab} %OK	
	\doloopuntil{\Jmeno}{Ondřej}{\tab} %OK
\eTABLE


\sym{T6}\par
\opencsvfile

\bTABLE
	\bTR\bTD ID\eTD\bTD Jméno\eTD\bTD Příjmení\eTD\eTR
	\doloopif{Jan}{~=}{\Jmeno}{\tab} %OK	
\eTABLE


\sym{T7}\par
\opencsvfile

\bTABLE
	\bTR\bTD ID\eTD\bTD Jméno\eTD\bTD Příjmení\eTD\eTR
	\doloopuntil{\Jmeno}{Jan}{\tab} %OK
	\dorecurse{5}{\tab\readline\nextrow} %NOK - bez  kompilační chyby, ale vypisuje všechny stejné záznamy
\eTABLE
	
\sym{T8}\par
\opencsvfile	
\bTABLE
	\bTR\bTD ID\eTD\bTD Jméno\eTD\bTD Příjmení\eTD\eTR
	\dorecurse{5}{\xtab\readline\nextrow} %OK	
\eTABLE


\sym{T9}\par
\opencsvfile	

\bTABLE
	\bTR\bTD ID\eTD\bTD Jméno\eTD\bTD Příjmení\eTD\eTR
	\doloop{\lineaction\readline\nextrow\ifnum\numline>7\exitloop\fi} %OK
\eTABLE

\sym{T10}\par
\opencsvfile

\bTABLE
	\bTR\bTD ID\eTD\bTD Jméno\eTD\bTD Příjmení\eTD\eTR
	\doloop{\xtab\readline\nextrow\ifnum\numline>7\exitloop\fi} %OK	
	\doloop{\tab\readline\nextrow\ifnum\numline>7\exitloop\fi} %NOK - bez  kompilační chyby, ale vypisuje všechny stejné záznamy
	\doloop{\tab\readline\nextrow\ifnum\numline>7\exitloop\fi} %NOK - bez  kompilační chyby, ale vypisuje všechny stejné záznamy
\eTABLE

\sym{T11}\par
\opencsvfile

\bTABLE
	\bTR\bTD ID\eTD\bTD Jméno\eTD\bTD Příjmení\eTD\eTR
	\doloop{\xtab\readline\nextrow\ifnum\numline>7\exitloop\fi} %OK	
\eTABLE

\sym{T12}\par
\opencsvfile

\bTABLE
	\bTR\bTD ID\eTD\bTD Jméno\eTD\bTD Příjmení\eTD\eTR
	\doloop{\ifEOF\exitloop\else\lineaction\readline\nextrow\fi} %OK
\eTABLE

\sym{T13}\par
\opencsvfile

\bTABLE
	\bTR\bTD ID\eTD\bTD Jméno\eTD\bTD Příjmení\eTD\eTR
	\doloop{\lineaction\readline\nextrow\if\Id3\exitloop\fi} %OK
\eTABLE	

\stopitemize


	
\unexpanded\def\lineaction{\expanded{\bTR\bTD\cA\eTD\bTD\Jmeno\eTD\bTD\Prijmeni\eTD\eTR}}	



\startitemize[n]

\sym{T14}\par
\opencsvfile

\bTABLE
	\bTR\bTD ID\eTD\bTD Jméno\eTD\bTD Příjmení\eTD\eTR
	\doloopaction %OK provádí implicitně makro \lineaction pro všechny záznamy otevřeného CSV souboru uvnitř tabulky (netřeba \expanded) 
\eTABLE
	
\sym{T15}\par
\opencsvfile

\bTABLE
	\bTR\bTD ID\eTD\bTD Jméno\eTD\bTD Příjmení\eTD\eTR	
	\doloopaction{\lineaction} %OK stejné jako předchozí ... makro \lineaction zadáno explicitně (netřeba \expanded)
\eTABLE
	
\sym{T16}\par
\opencsvfile

\bTABLE
	\bTR\bTD ID\eTD\bTD Jméno\eTD\bTD Příjmení\eTD\eTR
	\doloopaction{\tab} %OK provede makro \tab pro všechny záznamy otevřeného CSV souboru uvnitř tabulky (netřeba \expanded)
									%\tdoloopaction{\xtab} %KIKS - nefunguje dobře, není chyba kompilace (problém s \expanded???)
\eTABLE

\sym{T17}\par
\opencsvfile

\bTABLE
	\bTR\bTD ID\eTD\bTD Jméno\eTD\bTD Příjmení\eTD\eTR	
	\doloopaction{\tab}{3} %OK provede makro \tab pro záznamy 1-3 otevřeného CSV souboru uvnitř tabulky (netřeba \expanded)
\eTABLE

	
\sym{T18}\par
\opencsvfile

\bTABLE
	\bTR\bTD ID\eTD\bTD Jméno\eTD\bTD Příjmení\eTD\eTR	
	\doloopaction{\tab}{3}{6}  %OK provede makro \tab pro záznamy 3-6 otevřeného CSV souboru uvnitř tabulky (netřeba \expanded)
 \eTABLE

	
\sym{T19}\par
\opencsvfile

\bTABLE
	\bTR\bTD ID\eTD\bTD Jméno\eTD\bTD Příjmení\eTD\eTR 
  \doloopforall %OK ... ekvivalentní \tdoloopaction - udělá \lineaction pro všechny záznamy (netřeba \expanded)
\eTABLE
	
\sym{T20}\par
\opencsvfile

\bTABLE
	\bTR\bTD ID\eTD\bTD Jméno\eTD\bTD Příjmení\eTD\eTR
	\doloopforall{\tab} %OK ... ekvivalentní \tdoloopaction - udělá \tab pro všechny záznamy (netřeba \expanded)
\eTABLE
	
\sym{T21}\par


\bTABLE
	\bTR\bTD ID\eTD\bTD Jméno\eTD\bTD Příjmení\eTD\eTR  
	\doloopif{\Jmeno}{==}{Jan}{\tab} %OK provede makro \tab pro všechny záznamy CSV souboru, které vyhovují podmínce tj. rovnost prvních dvou parametrů (netřeba \expanded)
\eTABLE

\sym{T22}\par


\bTABLE
	\bTR\bTD ID\eTD\bTD Jméno\eTD\bTD Příjmení\eTD\eTR  
	\doloopif{\Jmeno}{~=}{Jan}{\tab} %OK provede makro \tab pro všechny záznamy CSV souboru, které vyhovují podmínce tj. rovnost prvních dvou parametrů (netřeba \expanded)
\eTABLE

	
\sym{T23}\par
\opencsvfile

\bTABLE
	\bTR\bTD ID\eTD\bTD Jméno\eTD\bTD Příjmení\eTD\eTR	
  \doloopuntil{\Jmeno}{Slávek}{\tab} %OK provede makro \tab pro všechny záznamy CSV souboru, které vyhovují podmínce tj. rovnost prvních dvou parametrů (netřeba \expanded)
\eTABLE
	
\sym{T24}\par
\opencsvfile

\bTABLE
	\bTR\bTD ID\eTD\bTD Jméno\eTD\bTD Příjmení\eTD\eTR  
	\filelineaction %OK provede makro \lineaction pro všechny záznamy otevřeného CSV souboru (netřeba \expanded)
\eTABLE
	
\sym{T25}\par
\opencsvfile

\bTABLE
	\bTR\bTD ID\eTD\bTD Jméno\eTD\bTD Příjmení\eTD\eTR
  \filelineaction{lisajdb1a.csv} %OK provede makro \lineaction pro všechny záznamy CSV souboru lisajdb1a.csv (netřeba \expanded)
\eTABLE
	
\sym{T26}\par
\opencsvfile

\bTABLE
	\bTR\bTD ID\eTD\bTD Jméno\eTD\bTD Příjmení\eTD\eTR
  \filelineaction{lisajdb2a.csv}{4} %OK provede makro \lineaction pro záznamy 1-4 CSV souboru lisajdb2a.csv  (netřeba \expanded)
\eTABLE
	
\sym{T27}\par
\opencsvfile

\bTABLE
	\bTR\bTD ID\eTD\bTD Jméno\eTD\bTD Příjmení\eTD\eTR
	\filelineaction{lisajdb2a.csv}{4}{6} %OK provede makro \lineaction pro záznamy 4-6 CSV souboru lisajdb2a.csv  (netřeba \expanded)
\eTABLE
	
\sym{T28}\par
\opencsvfile

\bTABLE
	\bTR\bTD ID\eTD\bTD Jméno\eTD\bTD Příjmení\eTD\eTR
	\filelineaction %OK ... ekvivalentní \tdolooplineaction - provádí \lineaction pro všechny záznamy otevřeného CSV souboru 
\eTABLE
	
\sym{T29}\par
\opencsvfile

\bTABLE
	\bTR\bTD ID\eTD\bTD Jméno\eTD\bTD Příjmení\eTD\eTR
	\filelineaction{lisajdb1a.csv}
\eTABLE
	
\sym{T30}\par
\opencsvfile

\bTABLE
	\bTR\bTD ID\eTD\bTD Jméno\eTD\bTD Příjmení\eTD\eTR
	\filelineaction{lisajdb1a.csv}{4}
\eTABLE
	
\sym{T31}\par
\opencsvfile

\bTABLE
	\bTR\bTD ID\eTD\bTD Jméno\eTD\bTD Příjmení\eTD\eTR
	\filelineaction{lisajdb1a.csv}{3}{7}
\eTABLE
	
\stopitemize


 %\printline
 
 %\printall
 
 %\csvreport 
  
\page 


\opencsvfile{lisajdb2a.csv}

%\printline

%\printall

%\csvreport 



\bTABLE
	\bTR\bTD ID\eTD\bTD Jméno\eTD\bTD Příjmení\eTD\eTR
	\filelineaction{lisajdb2a.csv}{3}{7}
\eTABLE

%\printline

%\printall

%\csvreport 

\startmode[tabulky]
\stopmode

\stoptext 
