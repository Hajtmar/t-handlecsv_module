\mainlanguage[cz]            % Podle typografických pravidel 
\setuppapersize[A4][A4]      % Velikost papíru bude A5 a bude na stránce opět o téže velikosti. 
\setupcolors[state=start]    % Zapne používání barev.
\setuppagenumbering[state=stop]
\setuplayout[width=middle, topspace=1cm, header=1cm, footer=1cm]

%\enablemode[prvnipololeti]
\enablemode[druhepololeti]

\def\puvodniterminklasifikace{22.\ 6.\ 2015}
\def\nahradniterminklasifikace{30.\ 9.\ 2015}
\def\dopiszedne{25.\ 6.\ 2015}





\doifmodeelse{prvnipololeti}
{\def\pololeti{první}}
{\def\pololeti{druhé}}

\usedirectory[c:/1da/ConTeXt/ConTeXt-Tests/my_way/ScanCSV/t-handlecsv_module/]
\usemodule[handlecsv]


\long\def\addto#1#2{\expandafter\def\expandafter#1\expandafter{#1#2}}
\long\def\eaddto#1#2{\edef#1{#1#2}} 

\unexpanded\def\nazevpredmetu#1{\word{\ctxlua{context(thirddata.handlecsv.ParseCSVLine('#1',':')[2])}}}%

\def\neklpredmety{%
\def\NeklPredmety{}
\dostepwiserecurse{24}{58}{1}{\if\csvcell[\recurselevel,\linepointer]N \eaddto\NeklPredmety{\nazevpredmetu{\csvcell[\recurselevel,0]}, }\fi}% 
\dostepwiserecurse{75}{85}{1}{\if\csvcell[\recurselevel,\linepointer]N \eaddto\NeklPredmety{\nazevpredmetu{\csvcell[\recurselevel,0]}, }\fi}% 
}



%\input ../texlibrary/lib-lua.tex
\input lib-lua.tex


\unexpanded\def\lineaction{
\neklpredmety
\ifx\NeklPredmety\empty\addtonumline{-1}\else
\let\Trida\cA
\let\Prijmeni\cE
\let\Jmeno\cF
\let\DatumNarozeni\cG
\let\MistoNarozeni\cH
\let\Pohlavi\cJ
\let\Mesto\cK
\let\PSC\cL
\let\Ulice\cM
\let\ZZPrijmeni\cP
\let\ZZJmeno\cQ
\let\ZZUlice\cR
\let\ZZMesto\cS
\let\ZZPSC\cT
\let\DatNar\DatumNarozeni
\let\Obec\Mesto

%\vodoznakGyZA
\hlavicka
\hrule
\blank[big]

%\adresnistitek		
				
\blank[10mm]

{\tfa Věc: Stanovení náhradního termínu pro hodnocení}
				
\blank[big]	
			
Vážená paní,\crlf
Vážený pane,
			
\blank[big]	
			
Ředitel Gymnázia, Zábřeh, náměstí Osvobození 20 stanovuje náhradní termín pro 
hodnocení níže jmenovaného žáka, který z~důvodu nedostatku podkladů nemohl být 
hodnocen za {\pololeti} pololetí ke dni {\puvodniterminklasifikace}.

\blank[big]

		
\zak
						
\blank[big]
			
Předměty, ze kterých žák nemohl být hodnocen:	{\bf \NeklPredmety} %
			
\blank[big]
			
Náhradní termín pro hodnocení:	{\bf \nahradniterminklasifikace}

\blank[big]

\doifmodeelse{prvnipololeti}
{Nebude-li pro nedostatek podkladů možné žáka hodnotit ani v~náhradním termínu, 
žák nebude za první pololetí hodnocen.}
{Nebude-li žák hodnocen ani v~náhradním termínu, bude podle § 69 odst.~6 zákona 
č.~561/2004 Sb., o~předškolním, základním, středním, vyšším odborném a~jiném 
vzdělávání (školský zákon), v~platném znění, a~§~3 vyhlášky č.~13/2005~Sb.,
o~středním vzdělávání a~vzdělávání v~konzervatoři, v~platném znění, celkové 
hodnocení žáka na vysvědčení vyjádřeno stupněm {\bf neprospěl}.}
			



\blank[3*big]
			
V~Zábřehu dne {\dopiszedne} 					
			
\podpisreditele

\vfill

%Vzal na vědomí dne {\hbox to 4cm{\dotfill}}\hfill {\hbox to 5cm{\dotfill}}\break
%\null\hfill Podpis zákonného zástupce\break

\page
\fi
} 



\starttext

\setheader

\doifmodeelse{prvnipololeti}
{\setfiletoscan{export-1pololeti.csv}}
{\setfiletoscan{export-2pololeti.csv}}
\opencsvfile


% \doloop{
% \readline
% \lineaction
% \exitlooptest
% }


\filelineaction % OK

% \doloopfromto{1}{\numrows}{% OK
% \lineaction
% }

\stoptext
