 


\starttext

This works as expected:

\def\makrowithAAAAcontent{AAAA}

\edef\testparameter#1{%
\startluacode
local parameter='#1'
if parameter=='AAAA' then context('TRUE. ') else context('FALSE. ') end
context('Parameter '..parameter..' has length '..string.len(parameter))
\stopluacode
}%

\type{\makrowithAAAAcontent} -- \makrowithAAAAcontent

Test is \testparameter{BBBB} 

Test is \testparameter{AAAA} % OK

Test is \testparameter{\makrowithAAAAcontent} % OK

But what about with this: ???


% initialize values
\startluacode
A={}
for i=1,50 do A[i]='Record '..tostring(i) end
A[25]='AAAA'
\stopluacode


\def\makrowithAAAAcontent{AAAA}

\def\lastname{\dosingleempty\dolastname}%
\def\dolastname[#1]{\doifsomethingelse{#1}{\ctxlua{context(A[#1])}}{\makrowithAAAAcontent}}%


\type{\lastname} - \lastname

\type{\lastname[2]} - \lastname[2]

\type{\lastname[5]} - \lastname[5]

\type{\lastname[25]} - \lastname[25]





\edef\readandprocessparameters#1#2{%
\startluacode
local parameter1='#1'
if parameter1=='AAAA' then context('Test '..parameter1..' = AAAA is TRUE. ') else context('Test '..parameter1..' = AAAA is FALSE. ') end
context('Parameter 1 = '..parameter1..' has length '..string.len(parameter1))
context('\\crlf')
local parameter2='#2'
if parameter2=='AAAA' then context('Test '..parameter2..' = AAAA is TRUE. ') else context('Test '..parameter2..' = AAAA is FALSE. ') end
context('Parameter 2 = '..parameter2..' has length '..string.len(parameter2))
context('\\crlf')
\stopluacode
}%  


\readandprocessparameters{AAAA}{BBBB}

%Test is \readandprocessparameters{\lastname[25]}{\lastname} % Crashed
 
\readandprocessparameters{\\lastname[25]}{\\lastname}

\stoptext
