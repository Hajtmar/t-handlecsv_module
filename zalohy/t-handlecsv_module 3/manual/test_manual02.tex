% Možnosti získávání údajů z CSV tabulky pomocí maker
% \cA, \cB, .... a odpovídající \Firstname, \Lastname atd...
% \csvcell['',radek] - univerzální přístu k informacím ...
% \columncontent['sloupec']
% \colA, \colB, ... resp \colA[5], \colB[7],....atd...
% \colFirstname, \colFirstname[11]



\def\foo{XXXX}

\usemodule[handlecsv]
% \usemodule[handlecsv-tools]

%\setuppapersize[A0]
%\setuplayout[A0]

\starttext


\setheader % set a header flag i.e. CSV file has header in first line
\setsep{,} % change settings of default separator (this is ; semicolon) 
\opencsvfile{myfirstcsvexamplefile1.csv}
%\opencsvfile{myfirstcsvexamplefilewithoutheader.csv}






 






\startluacode
local pocet=#thirddata.handlecsv.gColumnNames
context('Počet je: ')
context(pocet)

local keyset={}
local n=0

for k,v in pairs(thirddata.handlecsv.gColNames) do
  n=n+1
  keyset[n]=k
end

table.sort(keyset)

for i=1,n do
context(''..keyset[i]..', ')
end


context('\\crlf')

for i=1,#thirddata.handlecsv.gColumnNames do
context('\\backslash '..thirddata.handlecsv.gColumnNames[i]..', ')
end
    
\stopluacode





Column names:

\dorecurse{\numcols}{\colname[\recurselevel] -- \xlscolname[\recurselevel] -- \cxlscolname[\recurselevel] -- \texcolname[\recurselevel]\crlf }


 Available macros:
  
\dorecurse{\numcols}{\backslash columncontent['\colname[\recurselevel]'] -- \xlscolname[\recurselevel] -- \cxlscolname[\recurselevel] -- \texcolname[\recurselevel]\crlf }

sss


\cA, \csname cA\endcsname \csname first_name\endcsname

ss

First names:

\dorecurse{\numrows}{
\columncontent['first_name'] \nextrow \crlf
}

Column contents:

\dorecurse{\numcols}{
\texcolname[\recurselevel]
\edef\namea{'\texcolname[\recurselevel]'}
\columncontent[\namea]
\columncontent[\recurselevel] \nextrow \crlf
}


Zajímavý obrat:

\dorecurse{\numcols}{
Name of column \recurselevel is \colname[\recurselevel]. Content of this column is \columncontent[\recurselevel]. Content this column by other metod is \edef\texname{'\texcolname[\recurselevel]'} -- \columncontent[\texname]. Serial number of column is \indexcolname[\texname].\crlf 
}



Stepping

\columncontent['firstxname'] \nextrow \crlf
\columncontent['first_name'] \nextrow \crlf
\columncontent['firstxname'] \nextrow \crlf
\columncontent['cA'] \nextrow \crlf
\columncontent['A'] \nextrow \crlf


Other Steppings:

\setheader % set a header flag i.e. CSV file has header in first line
\setsep{,} % change settings of default separator (this is ; semicolon) 
\opencsvfile{myfirstcsvexamplefile1.csv}
%\opencsvfile{myfirstcsvexamplefilewithoutheader.csv}

\readline

\cA -- \firstxname

\colA -- \colA[11] -- \colfirstxname -- \colfirstxname[11] \columncontent[4]\nextrow\crlf
\colA -- \colA[11] -- \colfirstxname -- \colfirstxname[11] \columncontent[4] \nextrow\crlf
\colA -- \colA[11] -- \colfirstxname -- \colfirstxname[11] \columncontent[4]\nextrow\crlf

\cA -- \firstxname

Cykly:

\dorecurse{\numrows}{%  
	rl: \recurselevel\setlinepointer{\recurselevel}: 
	1. \colfirstxname[\recurselevel] --
	2. \firstxname -- 
	3. \colcity[\recurselevel] --	
	4. \city --  
	5. \columncontent[5] -- 
	6. \columncontent['city']
	\crlf}


\unexpanded\def\lineaction{
  2. \firstxname -- 	
	4. \city --  
	5. \columncontent[5] -- 
	6. \columncontent['city']
\crlf
}


FLA:

\filelineaction





\title{Tabulkové použití} 

% Nyní ukázky použití v tabulkovém režimu. Toto použití vyžaduje expanzi uvnitř prostředí \bTABLE ... \eTABLE
% nebo použití jiných maker ... 


\unexpanded\def\lineaction{\expanded{\bTR\bTD\cA\eTD\bTD\colJmeno\eTD\bTD\Prijmeni\eTD\eTR}}
\unexpanded\def\tab{\expanded{\bTR\bTD\cA - \colJmeno\eTD\bTD\colJmeno\eTD\bTD\Prijmeni\eTD\eTR}}
\unexpanded\def\xtab{\expanded{\bTR\bTD\cA - \colJmeno\eTD\bTD\colJmeno\eTD\bTD\Prijmeni\eTD\eTR}}
\unexpanded\def\tableaction{\expanded{\bTR\bTD\cA\eTD\bTD\colJmeno\eTD\bTD\Prijmeni\eTD\eTR}}
\unexpanded\def\xtableaction{\expanded{\bTR\bTD\cA\eTD\bTD\colJmeno\eTD\bTD\Prijmeni\eTD\eTR}}


\setheader
\opencsvfile{../lisajdb1a.csv}



\startitemize[n]
\sym{T1}\par
\opencsvfile

\bTABLE
	\bTR\bTD ID\eTD\bTD Jméno\eTD\bTD Příjmení\eTD\eTR
	\doloopaction %OK implicitně používá \lineaction 
						%\doloopaction{\tab} %NOK - bez kompilační chyby, ale vypisuje všechny stejné záznamy
						%\doloopaction{\tab}{4} %NOK - bez kompilační chyby, ale vypisuje všechny stejné záznamy
						%\doloopaction{\tab}{2}{5} %NOK - bez  kompilační chyby, ale vypisuje všechny stejné záznamy  
										%\doloopaction{\xtab} %KIKS - kompilace končí zastavením a chybou (nesnáší \expanded ???)
						%\doloopforall{\tab}  %NOK - bez kompilační chyby, ale vypisuje všechny stejné záznamy
\eTABLE



\sym{T2}\par
\bTABLE
	\bTR\bTD ID\eTD\bTD Jméno\eTD\bTD Příjmení\eTD\eTR
\doloopfromto{3}{7}{\lineaction}
\eTABLE


\sym{T3}\par
\opencsvfile

\bTABLE
	\bTR\bTD ID\eTD\bTD Jméno\eTD\bTD Příjmení\eTD\eTR
	\doloopforall{\tab}  %OK
	\doloopif{Jan}{==}{\Jmeno}{\tab} %NOK - bez  kompilační chyby, ale vypisuje všechny stejné záznamy
\eTABLE


\sym{T4}\par
\opencsvfile

\bTABLE
	\bTR\bTD ID\eTD\bTD Jméno\eTD\bTD Příjmení\eTD\eTR
	\doloopif{Jan}{~=}{\Jmeno}{\tab} %NOK - bez  kompilační chyby, ale vypisuje všechny stejné záznamy
\eTABLE



\sym{T5}\par
\opencsvfile

\bTABLE
	\bTR\bTD ID\eTD\bTD Jméno\eTD\bTD Příjmení\eTD\eTR
	\doloopif{Jan}{==}{\Jmeno}{\tab} %OK	
	\doloopuntil{\Jmeno}{Ondřej}{\tab} %OK
\eTABLE


\sym{T6}\par
\opencsvfile

\bTABLE
	\bTR\bTD ID\eTD\bTD Jméno\eTD\bTD Příjmení\eTD\eTR
	\doloopif{Jan}{~=}{\Jmeno}{\tab} %OK	
\eTABLE


\sym{T7}\par
\opencsvfile

\bTABLE
	\bTR\bTD ID\eTD\bTD Jméno\eTD\bTD Příjmení\eTD\eTR
	\doloopuntil{\Jmeno}{Jan}{\tab} %OK
	\dorecurse{5}{\tab\nextrow} %NOK - bez  kompilační chyby, ale vypisuje všechny stejné záznamy
\eTABLE
	
\sym{T8}\par
\opencsvfile	
\bTABLE
	\bTR\bTD ID\eTD\bTD Jméno\eTD\bTD Příjmení\eTD\eTR
	\dorecurse{5}{\xtab\nextrow} %OK	
\eTABLE


\sym{T9}\par
\opencsvfile	

\bTABLE
	\bTR\bTD ID\eTD\bTD Jméno\eTD\bTD Příjmení\eTD\eTR
	\doloop{\lineaction\nextrow\ifnum\numline>7\exitloop\fi} %OK
\eTABLE

\sym{T10}\par
\opencsvfile

\bTABLE
	\bTR\bTD ID\eTD\bTD Jméno\eTD\bTD Příjmení\eTD\eTR
	\doloop{\xtab\nextrow\ifnum\numline>7\exitloop\fi} %OK	
	\doloop{\tab\nextrow\ifnum\numline>7\exitloop\fi} %NOK - bez  kompilační chyby, ale vypisuje všechny stejné záznamy
	\doloop{\tab\nextrow\ifnum\numline>7\exitloop\fi} %NOK - bez  kompilační chyby, ale vypisuje všechny stejné záznamy
\eTABLE

\sym{T11}\par
\opencsvfile

\bTABLE
	\bTR\bTD ID\eTD\bTD Jméno\eTD\bTD Příjmení\eTD\eTR
	\doloop{\xtab\nextrow\ifnum\numline>7\exitloop\fi} %OK	
\eTABLE

\sym{T12}\par
\opencsvfile

\bTABLE
	\bTR\bTD ID\eTD\bTD Jméno\eTD\bTD Příjmení\eTD\eTR
	\doloop{\ifEOF\exitloop\else\lineaction\nextrow\fi} %OK
\eTABLE

\sym{T13}\par
\opencsvfile

\bTABLE
	\bTR\bTD ID\eTD\bTD Jméno\eTD\bTD Příjmení\eTD\eTR
	\doloop{\lineaction\nextrow\if\Id3\exitloop\fi} %OK
\eTABLE	

\stopitemize


	
\unexpanded\def\lineaction{\expanded{\bTR\bTD\cA\eTD\bTD\colJmeno\eTD\bTD\Prijmeni\eTD\eTR}}	



\startitemize[n]

\sym{T14}\par
\opencsvfile

\bTABLE
	\bTR\bTD ID\eTD\bTD Jméno\eTD\bTD Příjmení\eTD\eTR
	\doloopaction %OK provádí implicitně makro \lineaction pro všechny záznamy otevřeného CSV souboru uvnitř tabulky (netřeba \expanded) 
\eTABLE
	
\sym{T15}\par
\opencsvfile

\bTABLE
	\bTR\bTD ID\eTD\bTD Jméno\eTD\bTD Příjmení\eTD\eTR	
	\doloopaction{\lineaction} %OK stejné jako předchozí ... makro \lineaction zadáno explicitně (netřeba \expanded)
\eTABLE
	
\sym{T16}\par
\opencsvfile

\bTABLE
	\bTR\bTD ID\eTD\bTD Jméno\eTD\bTD Příjmení\eTD\eTR
	\doloopaction{\tab} %OK provede makro \tab pro všechny záznamy otevřeného CSV souboru uvnitř tabulky (netřeba \expanded)
									%\tdoloopaction{\xtab} %KIKS - nefunguje dobře, není chyba kompilace (problém s \expanded???)
\eTABLE

\sym{T17}\par
\opencsvfile

\bTABLE
	\bTR\bTD ID\eTD\bTD Jméno\eTD\bTD Příjmení\eTD\eTR	
	\doloopaction{\tab}{3} %OK provede makro \tab pro záznamy 1-3 otevřeného CSV souboru uvnitř tabulky (netřeba \expanded)
\eTABLE

	
\sym{T18}\par
\opencsvfile

\bTABLE
	\bTR\bTD ID\eTD\bTD Jméno\eTD\bTD Příjmení\eTD\eTR	
	\doloopaction{\tab}{3}{6}  %OK provede makro \tab pro záznamy 3-6 otevřeného CSV souboru uvnitř tabulky (netřeba \expanded)
 \eTABLE

	
\sym{T19}\par
\opencsvfile

\bTABLE
	\bTR\bTD ID\eTD\bTD Jméno\eTD\bTD Příjmení\eTD\eTR 
  \doloopforall %OK ... ekvivalentní \tdoloopaction - udělá \lineaction pro všechny záznamy (netřeba \expanded)
\eTABLE
	
\sym{T20}\par
\opencsvfile

\bTABLE
	\bTR\bTD ID\eTD\bTD Jméno\eTD\bTD Příjmení\eTD\eTR
	\doloopforall{\tab} %OK ... ekvivalentní \tdoloopaction - udělá \tab pro všechny záznamy (netřeba \expanded)
\eTABLE
	
\sym{T21}\par


\bTABLE
	\bTR\bTD ID\eTD\bTD Jméno\eTD\bTD Příjmení\eTD\eTR  
	\doloopif{\Jmeno}{==}{Jan}{\tab} %OK provede makro \tab pro všechny záznamy CSV souboru, které vyhovují podmínce tj. rovnost prvních dvou parametrů (netřeba \expanded)
\eTABLE

\sym{T22}\par


\bTABLE
	\bTR\bTD ID\eTD\bTD Jméno\eTD\bTD Příjmení\eTD\eTR  
	\doloopif{\Jmeno}{~=}{Jan}{\tab} %OK provede makro \tab pro všechny záznamy CSV souboru, které vyhovují podmínce tj. rovnost prvních dvou parametrů (netřeba \expanded)
\eTABLE

	
\sym{T23}\par
\opencsvfile

\bTABLE
	\bTR\bTD ID\eTD\bTD Jméno\eTD\bTD Příjmení\eTD\eTR	
  \doloopuntil{\Jmeno}{Slávek}{\tab} %OK provede makro \tab pro všechny záznamy CSV souboru, které vyhovují podmínce tj. rovnost prvních dvou parametrů (netřeba \expanded)
\eTABLE
	
\sym{T24}\par
\opencsvfile

\bTABLE
	\bTR\bTD ID\eTD\bTD Jméno\eTD\bTD Příjmení\eTD\eTR  
	\filelineaction %OK provede makro \lineaction pro všechny záznamy otevřeného CSV souboru (netřeba \expanded)
\eTABLE
	
\sym{T25}\par
\opencsvfile

\bTABLE
	\bTR\bTD ID\eTD\bTD Jméno\eTD\bTD Příjmení\eTD\eTR
  \filelineaction{../lisajdb1a.csv} %OK provede makro \lineaction pro všechny záznamy CSV souboru lisajdb1a.csv (netřeba \expanded)
\eTABLE
	
\sym{T26}\par
\opencsvfile

\bTABLE
	\bTR\bTD ID\eTD\bTD Jméno\eTD\bTD Příjmení\eTD\eTR
  \filelineaction{../lisajdb2a.csv}{4} %OK provede makro \lineaction pro záznamy 1-4 CSV souboru lisajdb2a.csv  (netřeba \expanded)
\eTABLE
	
\sym{T27}\par
\opencsvfile

\bTABLE
	\bTR\bTD ID\eTD\bTD Jméno\eTD\bTD Příjmení\eTD\eTR
	\filelineaction{../lisajdb2a.csv}{4}{6} %OK provede makro \lineaction pro záznamy 4-6 CSV souboru lisajdb2a.csv  (netřeba \expanded)
\eTABLE
	
\sym{T28}\par
\opencsvfile

\bTABLE
	\bTR\bTD ID\eTD\bTD Jméno\eTD\bTD Příjmení\eTD\eTR
	\filelineaction %OK ... ekvivalentní \tdolooplineaction - provádí \lineaction pro všechny záznamy otevřeného CSV souboru 
\eTABLE
	
\sym{T29}\par
\opencsvfile

\bTABLE
	\bTR\bTD ID\eTD\bTD Jméno\eTD\bTD Příjmení\eTD\eTR
	\filelineaction{../lisajdb1a.csv}
\eTABLE
	
\sym{T30}\par
\opencsvfile

\bTABLE
	\bTR\bTD ID\eTD\bTD Jméno\eTD\bTD Příjmení\eTD\eTR
	\filelineaction{../lisajdb1a.csv}{4}
\eTABLE
	
\sym{T31}\par
\opencsvfile

\bTABLE
	\bTR\bTD ID\eTD\bTD Jméno\eTD\bTD Příjmení\eTD\eTR
	\filelineaction{../lisajdb1a.csv}{3}{7}
\eTABLE
	
\stopitemize


 %\printline
 
 %\printall
 
 %\csvreport 
  
\page 


\opencsvfile{../lisajdb2a.csv}

%\printline

%\printall

%\csvreport 



\bTABLE
	\bTR\bTD ID\eTD\bTD Jméno\eTD\bTD Příjmení\eTD\eTR
	\filelineaction{../lisajdb2a.csv}{3}{7}
\eTABLE



\stoptext




\startluacode
    interfaces.setmacro("foo","bar")
\stopluacode

\foo

\start
\startluacode
    interfaces.setmacro("foo","foo","global")
\stopluacode
\stop

\foo
