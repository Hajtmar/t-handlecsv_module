% Možnosti získávání údajů z CSV tabulky pomocí maker
% \cA, \cB, .... a odpovídající \Firstname, \Lastname atd...
% \csvcell['',radek] - univerzální přístu k informacím ...
% \columncontent['sloupec']
% \colA, \colB, ... resp \colA[5], \colB[7],....atd...
% \colFirstname, \colFirstname[11]




\usemodule[handlecsv]
% \usemodule[handlecsv-tools]

%\setuppapersize[A0]
%\setuplayout[A0]

\starttext


\setheader % set a header flag i.e. CSV file has header in first line
\setsep{,} % change settings of default separator (this is ; semicolon) 
\opencsvfile{myfirstcsvexamplefile.csv}
%\opencsvfile{myfirstcsvexamplefilewithoutheader.csv}







\catcode`\#=12 %CSV file contains # characters (i.e. TeX problematic character)



\startluacode
local pocet=#thirddata.handlecsv.gColumnNames
context('Počet je: ')
context(pocet)

local keyset={}
local n=0

for k,v in pairs(thirddata.handlecsv.gColNames) do
  n=n+1
  keyset[n]=k
end

table.sort(keyset)

for i=1,n do
context(''..keyset[i]..', ')
end


context('\\crlf')

for i=1,#thirddata.handlecsv.gColumnNames do
context('\\backslash '..thirddata.handlecsv.gColumnNames[i]..', ')
end
    
\stopluacode





Column names:

\dorecurse{\numcols}{\colname[\recurselevel] -- \xlscolname[\recurselevel] -- \cxlscolname[\recurselevel] -- \texcolname[\recurselevel]\crlf }


 Available macros:
  
\dorecurse{\numcols}{\backslash columncontent['\colname[\recurselevel]'] -- \xlscolname[\recurselevel] -- \cxlscolname[\recurselevel] -- \texcolname[\recurselevel]\crlf }

sss


\cA, \csname cA\endcsname \csname first_name\endcsname

ss

First names:

\dorecurse{\numrows}{
\columncontent['first_name'] \nextrow \crlf
}

Column contents:

\dorecurse{\numcols}{
\texcolname[\recurselevel]
\edef\namea{'\texcolname[\recurselevel]'}
\columncontent[\namea]
\columncontent[\recurselevel] \nextrow \crlf
}


Zajímavý obrat:

\dorecurse{\numcols}{
Name of column \recurselevel is \colname[\recurselevel]. Content of this column is \columncontent[\recurselevel]. Content this column by other metod is \edef\texname{'\texcolname[\recurselevel]'} -- \columncontent[\texname]. Serial number of column is \indexcolname[\texname].\crlf 
}


Stepping

\columncontent['firstxname'] \nextrow \crlf
\columncontent['first_name'] \nextrow \crlf
\columncontent['firstxname'] \nextrow \crlf
\columncontent['cA'] \nextrow \crlf
\columncontent['A'] \nextrow \crlf


xxxxxxxxxxxxxxxxxxxxxxxxxxxxxxxxxxxxxxxxxxxxxxxxxxxxxxxxxxxxxxxxxxxxxxxxxxxxxxxx

\type{\numrows}: \numrows

\type{\numemptyrows}: \numemptyrows

\type{\numnotemptyrows}: \numnotemptyrows

\type{\numcols}: \numcols

\type{\csvfilename}: \csvfilename

% \type{\csvreport}: 
% 
% \csvreport
% 
% \csvfilename
% 
% 
% \type{\csvreport{myfirstcsvexamplefilewithemptylines.csv}}: 
% 
% \csvreport{myfirstcsvexamplefilewithemptylines.csv} 
% 
% 
% \type{\printline}: 
% 
% \printline
% 
% \type{\printall}: 
% 
% {\catcode`\#=12 \printall}
% 



\csvcell[2,1]

\csvcell[2,1]



% \colname[#1]
% \indexcolname[#1]
% \xlscolname[#1]
% \numberxlscolname[#1]





\stoptext
