\usemodule[annotation]


\defineannotation
  [result]
  [alternative=command,
   command=\resultCommand]

\define[2]\resultCommand
  {\result{#2}}

\defineframed[result][width=local,offset=3mm,frame=on,align=yes,background=color,backgroundcolor=yellow]

\setupcolor[x11]

\def\typeexampleandresult[#1]#2{

\blank[2*big]

{\em For example following source code:}

\startframedtext[width=\textwidth, frame=on, align=yes, background=color, backgroundcolor=lightgray]
{
\switchtobodyfont[#2]
\typebuffer[#1]
}
\stopframedtext

\blank[big]

{\em \dots\ produces the following output result:}

\blank[big]

\getbuffer[#1]

}


\startbuffer[example1-cell coordinates]
\usemodule[handlecsv]

\setheader %first row contents header with columns  names
\setsep{,} %columns separators
\opencsvfile{myfirstcsvexamplefile.csv} %open CSV table

\startresult
Content of \type{\csvcell[1,1]} (i.e. 1st colmn, 1st row) is {\bf \csvcell[1,1]}. \crlf
Content of \type{\csvcell[2,3]} (i.e. 2nd colmn, 3rd row) is {\bf \csvcell[2,3]}. \crlf
Content of \type{\csvcell[9,5]} is {\bf \csvcell[9,5]}. \crlf
Header item \type{\csvcell[3,0]} has content {\ttbf \csvcell[3,0]}.
\stopresult 
\stopbuffer




\startbuffer[example2-cell coordinates]
\startresult
\type{\csvcell['A',1]} --> {\bf \csvcell['A',1]} \crlf
\type{\csvcell['cA',1]} --> {\bf \csvcell['cA',1]} (library authorized version)\crlf
\type{\csvcell['B',3]}  --> {\bf \csvcell['B',3]} \crlf
\type{\csvcell['cI',5]} --> {\bf \csvcell['cI',5]} \crlf
Header item \type{\csvcell['J',0]} --> {\bf \csvcell['J',0]} \crlf
Out of range \type{\csvcell['Q',157]} --> {\bf \csvcell['Q',157]}
\stopresult
\stopbuffer




\startbuffer[example3-cell coordinates]
\startresult
Exact notation \type{\csvcell['first_name',1]} --> {\bf \csvcell['first_name',1]} \crlf
\TeX\ variant name \type{\csvcell['firstxname',1]} --> {\bf \csvcell['firstxname',1]} \crlf
\stopresult
\stopbuffer




\startbuffer[example4-cell coordinates]
\startresult
Exact notation \type{\csvcell['email',5]} --> {\bf \csvcell['email',5]} \crlf
Exact notation \type{\csvcell['phone2',10]} --> {\bf \csvcell['phone2',10]} \crlf
\TeX\ variant name \type{\csvcell['phoneII',10]} --> {\bf \csvcell['phoneII',10]} \crlf
\type{\csvcell['NonExistentColumn',8]} --> {\bf \csvcell['NonExistentColumn',8]}
\stopresult
\stopbuffer



\startbuffer[example5-cell coordinates]
\startresult%
\def\myrow{13}
\def\mycol{7}
\def\mynextrow{7}
\def\myxlscolumn{'A'}
\def\mycxlscolumn{'cC'}
\def\mycolumnname{'email'}

\type{\csvcell[\mycol,\myrow]} --> {\bf \csvcell[\mycol,\myrow]} \crlf
\type{\csvcell[\myxlscolumn,\mynextrow]} --> {\bf \csvcell[\myxlscolumn,\mynextrow]} \crlf
\type{\csvcell[\mycxlscolumn,\myrow]} --> {\bf \csvcell[\mycxlscolumn,\myrow]} \crlf
\type{\csvcell[\mycolumnname,\mynextrow]} --> {\bf \csvcell[\mycolumnname,\mynextrow]}
\stopresult
\stopbuffer



\startbuffer[example6-cell coordinates]
\startresult
\dorecurse{5}{\csvcell['first_name',\recurselevel], }
\stopresult
\stopbuffer



\startbuffer[example7-cell coordinates]
\startresult
Header items are \dorecurse{\numcols}{{\ttbf\csvcell[\recurselevel,0]}, }
\stopresult
\stopbuffer


\startbuffer[example8-cell coordinates]
\setuppapersize[A3,landscape][A3,landscape]
{\catcode`\#=12 %CSV file contains # characters (i.e. TeX problematic character)
\switchtobodyfont[10pt]
\setupTABLE[background=color,backgroundcolor=yellow]
\setupTABLE[row][first][background=color,backgroundcolor=lightblue]
\bTABLE[offset=2pt, split=yes]
 \dorecurse{\numexpr(\numrows+1)}
 {\bTR
   \dorecurse{\numcols}
	 {\bTD \csvcell[\currentTABLEcolumn,\currentTABLErow-1] \eTD}
 \eTR}
\eTABLE}
\stopbuffer


\startbuffer[examplex-cell coordinates]
\startresult
\stopresult
\stopbuffer

\startbuffer[examplex-cell coordinates]
\startresult
\stopresult
\stopbuffer

\startbuffer[examplex-cell coordinates]
\startresult
\stopresult
\stopbuffer
