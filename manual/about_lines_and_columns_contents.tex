


\setuppapersize[A4,portrait][A4,portrait]

\setuplayout[width=middle, height=middle, backspace=1in, cutspace=1in, margins=0pt]

\input about_cells_contents-buffers



\starttext
\title{Disclosure of the data in the lines and columns of CSV table}

%\startcolumns[n=2, distance=2mm, tolerance=stretch, balance=yes]
After loading CSV table into module (when use \type{\opencsvfile} command) the 
CSV data are put into the library macros for practical using. Datas in macros 
are available to ConTeXt and can be browsed “manually“ or by using the 
standard cycles or predefined cycles of library.

Processing method and creating of usable macros depends on whether it's header 
of CSV table or not. If the first row of the table contains the so-called 
header of CSV file, then column names will be used as names of macros. In 
addition, macros are created next macros, whose names match the names of cells 
in a spreadsheet (eg. Excel). 

The contents of individual cells can be obtained using macro 
\type{\csvcell[<}{\it column}\type{>,<}{\it row}\type{>]} or using any macros 
that refer to a specific column, while line number of table is determined 
using a pointer (linepointer). Let me show the following simple example.


To begin, suppose that we will process the CSV table 
{\ss myfirstcsvexamplefile.csv}, the column delimiter is comma (\type{,}) and in 
first line is header with these items: {\tt "first_name"}, {\tt "last_name"}, 
{\tt "company_name"}, {\tt "address"}, {\tt "city"}, {\tt "county"}, 
{\tt "state"}, {\tt "zip"}, {\tt "phone1"}, {\tt "phone2"}, {\tt "email"}, 
{\tt "web"}.

%\stopcolumns


\title{Examples of usage}

\subject{Contents cells via coordinates}


















\stoptext

 
