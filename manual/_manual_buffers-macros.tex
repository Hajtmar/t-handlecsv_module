\startcomponent  _manual_buffers-macros

% \numline
% \nextnumline
% \setnumline
% \addtonumline
% \resetnumline
% \linepointer
% \setlinepointer
% \resetlinepointer

% \notmarkemptylines % Thus, all the rows are considered non-empty (although there is not necessary any information in them) 
% \markemptylines % completely blank lines are marked and can be for example excluded from processing.
% \nextline \nextrow
% \blinehook
% \elinehook
% \bfilehook
% \efilehook
% \bch
% \ech
% \readline
% \doloopfromto{from}{to}{action}
% \doloopforall, \doloopforall{\action}
% \doloopaction, \doloopaction{\action}, \doloopaction{\action}{4}, \doloopaction{\action}{2}{5}
% \doloopif{value1}{[compare_operator]}{value2}{macro_for_doing} % [compareoperators] <, >, ==(eq), ~=(neq), >=, <=, in, until, while
% \doloopifnum
% \doloopuntil#1#2#3{\doloopif{#1}{until}{#2}{#3}}% \doloopuntil{\Trida}{3.A}{\tableaction}  % List all, until the test is not met - then just quit. Repeat until satisfied. If it is not never met, will list all records.
% \doloopwhile#1#2#3{\doloopif{#1}{while}{#2}{#3}}% \doloopwhile{\Trida}{3.A}{\tableaction}  % List when the test is met, other just quit.
% \filelineaction, \filelineaction{filename.csv} 
% \thenumexpr





% Makra pro nastavení před zpracováním 
\startbuffer[macro.introductionsmacros]
\resethooks
\setheader
\unsetheader (or syn. \resetheader)
\setsep{;}
\unsetsep (or syn. \resetsep)
\setfiletoscan{filename.csv}
\opencsvfile
\opencsvfile{filename.csv}
\stopbuffer




% Makra pro předávání parametrů: 
\startbuffer[macro.infomacros]
\numrows
\numemptyrows
\numnotemptyrows
\numcols
\csvfilename
% následující jsou součástí t-handlecsv-tools knihovny
%\csvreport
%\csvreport{filename.csv} 
%\printline
%\printall

\stopbuffer




% Makra pro práci s buňkami 
\startbuffer[macro.columns]
\colname[number]
\xlscolname[number]
\numberxlscolname['XLSColName']
\indexcolname['ColName']
\indexcolname['XLSColName']
\indexcolname['cXLSColName']
\columncontent[number]
\columncontent['ColName']
\columncontent['XLSColName']
\columncontent['cXLSColName']
\stopbuffer


\startbuffer[macro.columns.example1]
1. \type{\colname[2]}: \colname[2]
2. \type{\xlscolname[5]}: \xlscolname[5]
3. \type{\numberxlscolname['F']}: \numberxlscolname['F']
4. \type{\indexcolname['Firstname']}: \indexcolname['Firstname']
5. \type{\indexcolname['B']}: \indexcolname['B']
6. \type{\indexcolname['cC']}: \indexcolname['cC']
7. \type{\columncontent[2]}: \columncontent[2]
8. \type{\columncontent['A']}: \columncontent['A']
9. \type{\columncontent['cA']}: \columncontent['cA']
10. \type{\columncontent['Firstname']}: \columncontent['Firstname']
11. \type{\columncontent['firstname']}: \columncontent['firstname']
12. \type{\columncontent['123']}: \columncontent['123']
13. \type{\columncontent['!?*']}: \columncontent['!?*']
\stopbuffer




\startbuffer[macro.cells]
\csvcell[column,row]
\stopbuffer





% Definice TeXových logických vyhodnocovacích maker  
\startbuffer[macro.newifs]
\ifissetheader
\ifnotsetheader
\ifEOF
\ifnotEOF
\ifemptyline
\ifnotemptyline
\ifemptylinesmarking
\ifemptylinesnotmarking

\stopbuffer


\startbuffer[macro.newifs.example1]
1. \type{issetheader} is \ifissetheader TRUE \else FALSE\fi
2. \type{notsetheader} is \ifnotsetheader TRUE \else FALSE\fi
3. \type{EOF} is \ifEOF TRUE \else FALSE\fi
4. \type{notEOF}  is \ifnotEOF TRUE \else FALSE\fi
5. \type{emptyline} is \ifemptyline TRUE \else FALSE\fi
6. \type{notemptyline} is \ifnotemptyline TRUE \else FALSE\fi
7. \type{emptylinesmarking}  is \ifemptylinesmarking TRUE \else FALSE\fi
8. \type{emptylinesnotmarking}  is \ifemptylinesnotmarking TRUE \else FALSE\fi

%  We can use macros \markemptylines and \notmarkemptylines to set values: 
\markemptylines
9. After \type{\markemptylines} settings is \type{emptylinesmarking}  set to \ifemptylinesmarking TRUE \else FALSE\fi 
10. and \type{emptylinesnotmarking}  is \ifemptylinesnotmarking TRUE \else FALSE\fi
\notmarkemptylines
11. After \type{\notmarkemptylines} settings is \type{emptylinesmarking} set to \ifemptylinesmarking TRUE \else FALSE\fi
12. and \type{emptylinesnotmarking} is \ifemptylinesnotmarking TRUE \else FALSE\fi
\resetmarkemptylines
13. After \type{\resetmarkemptylines} settings is \type{emptylinesmarking} set to \ifemptylinesmarking TRUE \else FALSE\fi
14. and \type{emptylinesnotmarking} is \ifemptylinesnotmarking TRUE \else FALSE\fi
\stopbuffer





\startbuffer[macro.numrows]
\input tufte
\stopbuffer



\stopcomponent
